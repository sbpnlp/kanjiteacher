%%Create new if environment
%TODO: nimm das hier wieder rein:
%\newif\ifpdf\ifx\pdfoutput\undefined\pdffalse\else\pdfoutput=1\pdftrue\fi

%\documentclass[a4paper]{report}
%\documentclass[a4paper,pdftex]{report}
%\documentclass[a4paper]{article}
%\usepackage{amssymb,amsmath}

%\usepackage{gb4e}
%\usepackage{avm+}
% \bibliographystyle{h-physrev3}
%\usepackage{a4wide}
%\hyphenation{spe-ci-fied dis-tri-bu-ted am-bi-fix-al pa-ra-digm}

% \avmfont{\sc}
% \avmoptions{sorted, active}
% \avmvalfont{\rm}
% \avmsortfont{\scriptsize\it}
% {\regAvmFonts\avmoptions{center}

- Why this section? 
  The purpose of this section is 
  It would be off purpose, if 
- What goes into this section?
  The main content of this section is 
  * if describing a problem: why is the problem relevant.
  * if describing a solution to a problem: what alternatives were
    there to solve it, why was this solution chosen? 
    what made it the best choice? was it the optimal solution?
- How will this section be structured and organised?
  The organisational structure of the section 
- In what style will it be written?
  The style of writing will be 
- Next action - what to write first?
  The next part to write is

%xxx - Why this section? 
%xxx   The purpose of this section is 
%xxx   It would be off purpose, if 
%xxx - What goes into this section?
%xxx   The main content of this section is 
%xxx   * if describing a problem: why is the problem relevant.
%xxx   * if describing a solution to a problem: what alternatives were
%xxx     there to solve it, why was this solution chosen? 
%xxx     what made it the best choice? was it the optimal solution?
%xxx - How will this section be structured and organised?
%xxx   The organisational structure of the section 
%xxx - In what style will it be written?
%xxx   The style of writing will be 
%xxx - Next action - what to write first?
%xxx   The next part to write is


%Enter source code in latex documents:

http://www.theofel.de/archives/2004/10/source_code_in.html

\usepackage{listings} 
\lstset{numbers=left, numberstyle=\tiny, numbersep=5pt} 
\lstset{language=Perl} 

1.   Als Code-Schnippsel im Flie�text
\lstinline|print "hello world"|
2. Als eigenst�ndigen Sourcecode
\begin{lstlisting}[caption=Beispielcode]{Name}
  print "hello world";
\end{lstlisting}
3. Einbetten einer externen Source-Code-Datei
\lstinputlisting[frame=single,label=beispielcode,caption=Ein Beispiel]{beispiel.pl}




Bleibe schon in der ersten Zeile stecken: \\
Warum sind die Punkte mit \( 1,exp\) indiziert?
Also
\( P_{1,exp}, \ldots , P_{n,exp} \)
und dann bei \( x \) und \(y \) genauso:
\( (x_{i,exp}; y_{i,exp} ) \)

Was ist \( exp \)? Wofuer steht es und was bedeutet es mathematisch?

Und dann genauso \( cal \)?

Ach, die \(a\) sind also noch unbekannt...
Ok, dann ist also \( y_{i,cal} \) auch noch unbekannt, unabhaengig davon,
was \(cal\) eigentlich ist. Ok, soweit komme ich mit...
Aber jetzt: Woher kommen auf einmal die \( \delta \)?
 \begin{quote}
 \(
  \frac{\delta}{\delta a_0}
 \)
 \end{quote}
Bzw. ist das ueberhaupt ein Delta? Kenne das Symbol nicht - sieht irgendwie
nach Delta aus - aber woher kommt es und was macht es da?
Wenn es eine Relle Zahl symbolisiert, kuerzt es sich doch auch heraus,
oder? Ersetze ich das Symbol mal gegen ein normales \( x \),
kriege ich:
 \begin{quote}
 \(
  \frac{x}{x a_0}
 \)
 \end{quote}
Wenn ich das kuerze, kommt 
 \begin{quote}
 \(
  \frac{1}{a_0}
 \)
 \end{quote}
heraus. Also, wozu das ganze? Aufloesung des Quadrates und das Reinziehen in
die Summe akzeptiere ich einfach mal - aha, so geht das also...
Aber ich komme gerade darauf, dass es vielleicht doch noch nicht das ist,
was ich zu berechnen versuche... Ich weiss es nicht, weil ich es nicht verstehe...

