%%% Local Variables: 
%%% mode: latex
%%% TeX-master: "../KanjiHWR"
%%% End: 

\chapter{Handwriting Recognition Engine}
\label{chap:handwritingrecognitionengine}

- Why this section? 
  The purpose of this section is to describe the software module of the 
  HWR engine.
  Any important of the software should be described. This is the core of the
  HWR part of the software and therefore needs description.
  It would be off purpose, if non-technical aspects of the HWR Engine would be 
  described.
- What goes into this section?
  The main content of this section is the complete functionality of the 
  HWR engine.
  It should become apparent how the HWR engine works.
  * if describing a problem: why is the problem relevant.
    The problem of HWR is relevant, because it is the crucial novelty of the
    application. The combination of a Kanji learning tool and a HWR is a new 
    thing. It enables user to practice writing Kanji. (maybe too high-level for
    this section, move to intro / motivation section).
  * if describing a solution to a problem: 
    what alternatives were there to solve it?    
    Instead of a HWR engine, the user could either not have a HWR engine and
    write on paper or not practice writing the kanji.
    why was this solution chosen? 
    because it creates an additional benefit, that was not available before.
    what made it the best choice? 
    the fact that it writing on paper is impractical and not practicing does
    not lead to the desired result.

    was it the optimal solution? given there is only not practicing and
    practicing on paper: yes.

- How will this section be structured and organised?
  The organisational structure of the section is already layed out in the
  sections and subsections.
- In what style will it be written?
  The style of writing will be technical. A technical description, using
  pseudocode.
- Next action - what to write first?
  The next part to write is the answers to the questions of all subsections.

%xxx always: what alternatives were there?

\section{Data Capturing}
\label{sec:hwre:datacapturing}
this should deal with how the data are captured during the process
mouse coordinates and stuff

- Why this section? 
  The purpose of this section is to describe the data capturing process.
  It would be off purpose, if it describes to much around - the devices, the
  transmission. This should focus on the capturing as such.
  It should not go into great deail.
  
- What goes into this section?
  The main content of this section is the technical data capturing.
  A description of how to
  programmatically generate a pen trajectory from reading mouse coordinates.
  
  * if describing a problem: why is the problem relevant.
    The relevance of the problem is evident: Without the user's writing,
    there would not be any handwriting recognition.

  * if describing a solution to a problem: what alternatives were
    there to solve it, why was this solution chosen? 
    The alternatives are not massively relevant here, since they come down
    to using different input devices, which is discussed elsewhere.

    what made it the best choice? 
    the fact that this is just how it is done.
    was it the optimal solution?
    yes.

- How will this section be structured and organised?
  The organisational structure of the section begins with describing the mobile 
  input GUI from a technical level.
  Then the class in the background that actually captures the mouse events.
  the capturing of the mouse events.
  the storing of the data points in lists.
 
- In what style will it be written?
  The style of writing will be technical. using pseudocode if required.
  however the algorithm is not relevant here.

- Next action - what to write first?
  The next part to write is the one about the GUI
  also the substructuring needs to be done as required.



\section{Data Format}
\label{sec:hwre:dataformat}

- Why this section? 
  The purpose of this section is to give the reader an idea of how the captured
  data is structured. That means concretely the character format, describing
  the HW trajectories. This is necessary in order to be able to explain the
  Funktionsweise of the HWR engine in detail.

  It would be off purpose, if the section contained a description of the 
  lexical data, which is described in a different section.

- What goes into this section?
  The main content of this section is what is stated under purpose:
  A description of how the captured data is structured.

  * if describing a problem: why is the problem relevant.
    The task of handwriting recognition is essentially a task working with data.
    Therefore the structure of the data is one of the crucial points.
    
  * if describing a solution to a problem: what alternatives were
    there to solve it, why was this solution chosen? 

    The solution was chosen because of its clarity, simplicity and 
    accessibility, also because of its interoperability with potential 
    other systems.
    what made it the best choice? was it the optimal solution?
    For the purpose followed with the system - yes, it was the optimal solution,
    because it does not require much effort to parse the stored data
    and it contains all necessary information in a structured manner.
    
- How will this section be structured and organised?
  The organisational structure of the section contains:

  A general description of the XML format - why it was used as opposed to
  other possible formats like unipen and inkml.
  show requirements!
  show process of how I came to current data formats.
  include character models 1-4 on pages. - however, not everything in one go,
  but rather in the individual sections if possible.
  in the end, a radical has its own format that is unchanged, even
  if internal structure of a stroke is changed.
  (unipen is only text based, inkml does not help here, but the system allows
  for exchange of the custom format with those)

  data format of and representation of 
  point, stroke, box, radical, character s. 20-23, 25f

- In what style will it be written?
  The style of writing will be technical - a description of the XML format.
  Pointer to code sample in Appendix.

- Next action - what to write first?
  The next part to write is the actual subsection structure of this section.

\section{Database}
\label{sec:hwre:database}

- Why this section? 
  The purpose of this section is to explain the structure and production of 
  the database.   That includes both pen trajectory and lexical data.
  That is a relevant information, because the system provides these information,
  therefore they must come from somewhere. In order to explain some of the
  error recognition processes it is necessary to have these information first.

  It would be off purpose, if anything else goes into this section.
  The xml format is described elsewhere, the lexical charater DB comes from 
  jim breen (cite his paper) nothing else needs be in here.

- What goes into this section?
  The main content of this section is a description of the character database.
  The jim breen stuff can be pointed to, but the production process of the 
  self made data should be described.

  * if describing a problem: why is the problem relevant.
    The problem is relevant, because without a character database, there won't
    be any recognition - the system wouldn't know anything.

  * if describing a solution to a problem: what alternatives were
    there to solve it, why was this solution chosen? 

    Alternatives - see alternatives of data format!
    Technical alternatives - proper DB instead of flat-file possible,
    but not necessary. It is actually not that much data,
    even if all characters are included.
   
    what made it the best choice? was it the optimal solution?
    the fact that the alternatives do not offer a better solution.

- How will this section be structured and organised?
  The organisational structure of the section will contain
  1. description of the technical format of the character data base
     two flat files - indexed via the actual kanji character in unicode
     format, since it is unique. this is a character-centred application!
     how are they accessible? what format?
     this can be seen more generic as 'the lexicon', not necessarily
     the linguistic information about the characters, 
     but rather the whole lexicon, including the point sequences.

  2. discussion of alternatives: relational DB or only one flat file
     two flat files are better, because of updates and other-language versions
     of the jim breen lexicon. leave it unchanged! otherwise you'd need a
     converter!

  3. Description of the production of the lexicon.
     it was not just taken from j.b. but it was intervowen?! (verflochten) 
     with the trajectories. where did I get these from? 
     how many chars are in the two dictionaries?

- In what style will it be written?
  The style of writing will be a technical description of how it is done.

- Next action - what to write first?
  The next part to write is to hash out the actual subsections.

\section{Recognition Architecture}
\label{sec:hwre:recognitionarchitecture}

- Why this section? 
  The purpose of this section is to give the reader an overview of the whole
  recognition system. the recognition system! not the whole system, 
  not the learning part.

  It would be off purpose if other stuff would be described.
  stick to the main focus - architecture of the pure recognition.
  If would be off purpose if it was too detailed.
  The actual recognition process is discussed in subsequent sections!
  don't go into too much detail.

- What goes into this section?
  The main content of this section is an overview of the recognition 
  architecture. This should not be confused with the overview of the
  whole Kanji Coach system. it is more detailed and focused.

  One big graphic shows the overview of the different modules.
  
  * if describing a problem: why is the problem relevant.
    not really a problem - every system has an architecture.
   
  * if describing a solution to a problem: what alternatives were
    there to solve it, why was this solution chosen? 

    this solution was chosen after review of around 50 HWR systems.
    the discussion of those can be found in chapter \ref{chap:onlinehwr}.
    this section only uses pointers to that chapter.

    what made it the best choice?
    certainly not the lacking of alternatives, but the mix between 
    possibilities to programmatically interact with the recognition process
    and performance of the process.

    was it the optimal solution?
    yes and no.
    it depends on the focus.
    big discussion! WHY this architecture?
    roughly following X and Y and Z,
    however, not creating an optimal handwriting recognition engine.
    research suggests that HMM models are more useful etc.
    however - the type of error recognition desired,
    requires a structural system.
    it is only new in research (find that one paper 
    that does that bloody thing) to blend HMM and structural models
    in a hybrid HWR system, simimlar to hybrid MT systems.
    this is interesting for outlook - point there to idea
    HMM vs structural vs error recognition

- How will this section be structured and organised?
  The organisational structure of the section will be as follows:
  1. architecture: graphic and explanatory description
  -> Modules and parts of HWR, create graphic.

  s. 18 zeichen, punkt usw. UML diagramme.
     
  2. brief discussion about alternative architectures

- In what style will it be written?
  The style of writing will be technical
- Next action - what to write first?
  The next part to write is to hash out the substructure in subsections.

\section{Stroke Recognition Process}
\label{sec:hwre:strokerecognitionprocess}

- Why this section? 
  The purpose of this section is to describe the recognition process of a strokes
  This is necessary, because it is a crucial part of the recognition process
  as such, to describe which is the purpose of this chapter

  It would be off purpose, if the radical or character recognition process 
  would be described, too.
  It would be off purpose, if the stroke recognition process would not be 
  layed out in its entirety.

- What goes into this section?
  The main content of this section is the details of the stroke recognition
  process. The question how it is done is central and crucial.

  * if describing a problem: why is the problem relevant.
    stroke recognition is not relevant as such, it becomes relevant because
    of the recognition approach chosen, which is similar to the one of CITE

  * if describing a solution to a problem: what alternatives were
    there to solve it, why was this solution chosen? 
    there are alternatives, many of which are possible. mathematical methods 
    and formats. since roughly following \shortciteANP{Nakagawa2008}'s
    \citeyear{Nakagawa2008} approach - the discussion of the different 
    approaches can be left to the handwriting recognition section and the 
    literature, however - I needed to come up with a 'good mix' of 
    recognition quality and the possibility to extract detailed knowledge 
    from the recognitiont process.
     
    what made it the best choice? 
    certainly not the lacking of alternatives, but the mix between 
    possibilities to programmatically interact with the recognition process
    and performance of the process.
    see - main section! not too much repetition.

    was it the optimal solution?

    given the circumstances - yes, because it fulfilled the criteria 
    (to be layed out)
    from a pure HWR perspective - no, sinces other systems perform better on the
    pure handwriting task.
    from a hwr in a learning environment perspective - yes, because this
    system gives us the ability to accompany the recognition process in detail
    

- How will this section be structured and organised?
  The organisational structure of the section will contain

  1. from the captured point lists to something cooler, what are we actually
     doing? from point to 'stroke'.

  2. 'normalisation', including 'boxing' and then 'scaling'

  3. curve handling. what are we doing at the edgy points?

  4. alternative solution: time warping. doesn't need anything of the
     abovementioned cool stuff, just one alorithm.
     show algorithm and give a rough, rough, rough explanation,
     maybe even without algorithm and point to papers that explain DTW.

- In what style will it be written?
  The style of writing will be technical and mathematical.
  if something is calculated, show maths behind rather than algorithm.

- Next action - what to write first?
  The next part to write is the detailed content requirements of each subsection.

xxx: see santosh2009 for mathematical stuff: nice description of what I'm doing

\subsection{Advanced Point Lists}
\label{sec:hwre:advancedpointlists}
%xxx this chapter used to be called:
%xxx \subsection{From Simple Point List to Boxed Representation}

what happens to the points?
nothing, really - the magic happens when normalisation and the other stuff
starts.
why this section? what's the purpose? oh, right - angles and vectors instead
of simple points. from one point to the next, or rather from on point to
ten points down the line, to get a rougher direction.
vectors make it interesting. 
impacts on curve handling! gradient and stuff can be measured in the 
vector representation (even without any boxes)
making the point list a cool mathematical object!
show code samples in pseudocode if necessary.
report about the cool stuff.

what's the similarity measure for
points and strokes?
show requirements.
what alternatives were there to consider?

\subsection{Normalisation}
\label{sec:hwre:normalisation}

what is N?
why do N?
show requirements.
how is N performed here?
why is it performed like that?

\subsubsection{Boxing}
\label{sec:hwre:boxing}
how is boxing done?
show requirements.
what alternatives were there to consider?
is it useful to have a similarity measure for bounding boxes?
yes! but why? explain!
size of the boxes! - think of characters that only have two strokes.

\subsubsection{Scaling}
\label{sec:hwre:scaling}

s. 42-45
how is scaling done?
show requirements.
what alternatives were there to consider?

\subsection{Curve Handling}
\label{sec:hwre:curvehandling}

S 14, 16, 17
how is curver handling done?
show requirements.
what alternatives were there to consider?

stroke matching with angles instead of point position.
s. 24

\subsection{Dynamic Time Warping}
\label{sec:hwre:dynamictimewarping}

what's the similarity measure for
points and strokes?
show requirements.
what alternatives were there to consider?

s. 51
how is dynamic time warping done here?
pointer to papers or hwr - chapter, don't explain DTW here.
show requirements
why DTW?
what alternatives were there to consider?
none - it is the alternative.
to all the other stuff I've been doing.
however, what about 3D time warping?


\section{Radical Recognition Process}
\label{sec:hwre:radicalrecognitionprocess}

- Why this section? 
  The purpose of this section is to explain the Radical recognition process in
  detail. It is a crucial part of the structural recognition process.
  Therefore the purpose is evident - see purpose of chapter.

  It would be off purpose, if there'd be too much around,
  focus on the RRP only, nothing around.

- What goes into this section?
  The main content of this section is the radical recognition.
  That contains how the model is matched to the input.
  Incrementality of recognition, using one stroke, two strokes,
  three strokes and so on...

  * if describing a problem: why is the problem relevant.
    it is a part of the structural recognition process!

  * if describing a solution to a problem: what alternatives were
    there to solve it, why was this solution chosen? 
    again - part of the whole process
    what made it the best choice? was it the optimal solution?
- How will this section be structured and organised?
  The organisational structure of the section 
- In what style will it be written?
  The style of writing will be 
- Next action - what to write first?
  The next part to write is


what's the similarity measure for
radicals?
show requirements.
what alternatives were there to consider?

what about stroke number and stroke sequence?
why deal with it? can be dealt with? 
how? why this way, why not another way?
generally, how is radical recognition done?
show requirements.
what alternatives were there to consider?

\section{Character Recognition Process}
\label{sec:hwre:characterrecognitionprocess}

- Why this section? 
  The purpose of this section is 
  It would be off purpose, if 
- What goes into this section?
  The main content of this section is 
  * if describing a problem: why is the problem relevant.
  * if describing a solution to a problem: what alternatives were
    there to solve it, why was this solution chosen? 
    what made it the best choice? was it the optimal solution?
- How will this section be structured and organised?
  The organisational structure of the section 
- In what style will it be written?
  The style of writing will be 
- Next action - what to write first?
  The next part to write is


what's the similarity measure for
characters?
show requirements.
s. 24 pseudocode
s. 9/10 pixelwolke vs. reihenfolge

\section{Error Handling}
\label{sec:hwre:errorhandling}

- Why this section? 
  The purpose of this section is 
  It would be off purpose, if 
- What goes into this section?
  The main content of this section is 
  * if describing a problem: why is the problem relevant.
  * if describing a solution to a problem: what alternatives were
    there to solve it, why was this solution chosen? 
    what made it the best choice? was it the optimal solution?
- How will this section be structured and organised?
  The organisational structure of the section 
- In what style will it be written?
  The style of writing will be 
- Next action - what to write first?
  The next part to write is



see section \ref{sec:concept:sourcesoferror} in chapter 
\ref{chap:conceptualdesignofkanjicoach} for possible sources of error

\subsection{Error Recognition}
\label{sec:hwre:errorrecognition}



why this section? to demonstrate own achievements of error recognition.
the reader should know how it is done technically.

what goes into this section? the aspects of finding errors. finding errors
is not a straightforward trivial task - whenever something does not match
it is an error - doesn't work like that. instead, 
firstly, it needs to be made sure that it actually is an error.
meaning - not a recognition error, but a user error.
secondly, the type of error needs be identified.
see section \ref{sec:concept:sourcesoferror} (or handwritten page 58)
for sources of error.

how will this section be written?
technical - first describe how the error recognition integrates into the
recognition process, then how errors are identified.


\subsection{Error Processing}
\label{sec:hwre:errorprocessing}

%focus on technical aspects



why this section? 
actually the 'handling' or 'processing' aspect could be 
described in the recognition section \ref{sec:hwre:errorrecognition} as well.
so this section is only for a better overview, for document structure, 
thematically they are the same section. thus they are put together under
Error Handling \ref{sec:hwre:errorhandling}.

what goes into this section?

