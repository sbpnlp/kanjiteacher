%%% Local Variables: 
%%% mode: latex
%%% TeX-master: "../KanjiHWR"
%%% End: 

\chapter{Handwriting Recognition Engine}
\label{chap:handwritingrecognitionengine}
The sections of this chapter are more the result of a brainstorming 
than a  proper thought-through chapter design.
xxx: see santosh2009 for mathematical stuff: nice description of what I'm doing

why this section? 
what goes into this section?
  if describing a problem: why is the problem relevant.
  if describing a solution to a problem: what alternatives were
  there to solve it, why was this solution chosen? what made it the best
  choice? was it the optimal solution?
how will this section be structured and organised?
in what style will it be written?
next action - what to write first?


%xxx always: what alternatives were there?

\section{Data Capturing}
\label{sec:hwre:datacapturing}
this should deal with how the data are captured during the process
mouse coordinates and stuff

why this section? 
what goes into this section?
  if describing a problem: why is the problem relevant.
  if describing a solution to a problem: what alternatives were
  there to solve it, why was this solution chosen? what made it the best
  choice? was it the optimal solution?
how will this section be structured and organised?
in what style will it be written?
next action - what to write first?

\section{Data Format}
\label{sec:hwre:dataformat}
how is the data structured? radicals, strokes, characters, xml-format
representation of point, stroke, box, radical, character
s. 20-23, 25f
include character models 1-4 on pages

show process of how I came to current data formats.
show requirements.
what alternatives were there to consider?

why this section? 
what goes into this section?
  if describing a problem: why is the problem relevant.
  if describing a solution to a problem: what alternatives were
  there to solve it, why was this solution chosen? what made it the best
  choice? was it the optimal solution?
how will this section be structured and organised?
in what style will it be written?
next action - what to write first?

\section{Database}
\label{sec:hwre:database}
this can be seen more generic as 'the lexicon', not necessarily
the linguistic information about the characters, but rather the whole
lexicon, including the point sequences.
where did I get it from? how many chars are in there?
how are they accessible? what format?
Jim Breen

why this section? 
what goes into this section?
  if describing a problem: why is the problem relevant.
  if describing a solution to a problem: what alternatives were
  there to solve it, why was this solution chosen? what made it the best
  choice? was it the optimal solution?
how will this section be structured and organised?
in what style will it be written?
next action - what to write first?


\section{Recognition Architecture}
\label{sec:hwre:recognitionarchitecture}

-> Modules and parts of HWR, create graphic.

s. 18 zeichen, punkt usw. UML diagramme.

why this section? 
what goes into this section?
  if describing a problem: why is the problem relevant.
  if describing a solution to a problem: what alternatives were
  there to solve it, why was this solution chosen? what made it the best
  choice? was it the optimal solution?
how will this section be structured and organised?
in what style will it be written?
next action - what to write first?

\section{Stroke Recognition Process}
\label{sec:hwre:strokerecognitionprocess}

why this section? 
what goes into this section?
  if describing a problem: why is the problem relevant.
  if describing a solution to a problem: what alternatives were
  there to solve it, why was this solution chosen? what made it the best
  choice? was it the optimal solution?
how will this section be structured and organised?
in what style will it be written?
next action - what to write first?

\subsection{Advanced Point Lists}
\label{sec:hwre:advancedpointlists}
%xxx this chapter used to be called:
%xxx \subsection{From Simple Point List to Boxed Representation}

why this section? 
what goes into this section?
  if describing a problem: why is the problem relevant.
  if describing a solution to a problem: what alternatives were
  there to solve it, why was this solution chosen? what made it the best
  choice? was it the optimal solution?
how will this section be structured and organised?
in what style will it be written?
next action - what to write first?

what's the similarity measure for
points and strokes?
show requirements.
what alternatives were there to consider?

\subsection{Normalisation}
\label{sec:hwre:normalisation}

why this section? 
what goes into this section?
  if describing a problem: why is the problem relevant.
  if describing a solution to a problem: what alternatives were
  there to solve it, why was this solution chosen? what made it the best
  choice? was it the optimal solution?
how will this section be structured and organised?
in what style will it be written?
next action - what to write first?

\subsection{Boxing}
\label{sec:hwre:boxing}
how is boxing done?
show requirements.
what alternatives were there to consider?
is it useful to have a similarity measure for bounding boxes?
yes! but why? explain!

why this section? 
what goes into this section?
  if describing a problem: why is the problem relevant.
  if describing a solution to a problem: what alternatives were
  there to solve it, why was this solution chosen? what made it the best
  choice? was it the optimal solution?
how will this section be structured and organised?
in what style will it be written?
next action - what to write first?

\subsubsection{Scaling}
\label{sec:hwre:scaling}

why this section? 
what goes into this section?
  if describing a problem: why is the problem relevant.
  if describing a solution to a problem: what alternatives were
  there to solve it, why was this solution chosen? what made it the best
  choice? was it the optimal solution?
how will this section be structured and organised?
in what style will it be written?
next action - what to write first?

s. 42-45
how is scaling done?
show requirements.
what alternatives were there to consider?

\subsection{Curve Handling}
\label{sec:hwre:curvehandling}

why this section? 
what goes into this section?
  if describing a problem: why is the problem relevant.
  if describing a solution to a problem: what alternatives were
  there to solve it, why was this solution chosen? what made it the best
  choice? was it the optimal solution?
how will this section be structured and organised?
in what style will it be written?
next action - what to write first?

S 14, 16, 17
how is curver handling done?
show requirements.
what alternatives were there to consider?

stroke matching with angles instead of point position.
s. 24

\subsection{Dynamic Time Warping}
\label{sec:hwre:dynamictimewarping}

why this section? 
what goes into this section?
  if describing a problem: why is the problem relevant.
  if describing a solution to a problem: what alternatives were
  there to solve it, why was this solution chosen? what made it the best
  choice? was it the optimal solution?
how will this section be structured and organised?
in what style will it be written?
next action - what to write first?

what's the similarity measure for
points and strokes?
show requirements.
what alternatives were there to consider?

s. 51
how is dynamic time warping done here?
pointer to papers or hwr - chapter, don't explain DTW here.
show requirements
why DTW?
what alternatives were there to consider?
none - it is the alternative.
to all the other stuff I've been doing.
however, what about 3D time warping?


\section{Radical Recognition Process}
\label{sec:hwre:radicalrecognitionprocess}

why this section? 
what goes into this section?
  if describing a problem: why is the problem relevant.
  if describing a solution to a problem: what alternatives were
  there to solve it, why was this solution chosen? what made it the best
  choice? was it the optimal solution?
how will this section be structured and organised?
in what style will it be written?
next action - what to write first?

what's the similarity measure for
radicals?
show requirements.
what alternatives were there to consider?

what about stroke number and stroke sequence?
why deal with it? can be dealt with? 
how? why this way, why not another way?
generally, how is radical recognition done?
show requirements.
what alternatives were there to consider?

\section{Character Recognition Process}
\label{sec:hwre:characterrecognitionprocess}

why this section? 
what goes into this section?
  if describing a problem: why is the problem relevant.
  if describing a solution to a problem: what alternatives were
  there to solve it, why was this solution chosen? what made it the best
  choice? was it the optimal solution?
how will this section be structured and organised?
in what style will it be written?
next action - what to write first?

what's the similarity measure for
characters?
show requirements.
s. 24 pseudocode
s. 9/10 pixelwolke vs. reihenfolge

\section{Error Handling}
\label{sec:hwre:errorhandling}

%xxx focus on technical aspects
why this section? 
what goes into this section?
  if describing a problem: why is the problem relevant.
  if describing a solution to a problem: what alternatives were
  there to solve it, why was this solution chosen? what made it the best
  choice? was it the optimal solution?
how will this section be structured and organised?
in what style will it be written?
next action - what to write first?


see section \ref{sec:concept:sourcesoferror} in chapter 
\ref{chap:conceptualdesignofkanjicoach} for possible sources of error

\subsection{Error Recognition}
\label{sec:hwre:errorrecognition}

why this section? 
what goes into this section?
  if describing a problem: why is the problem relevant.
  if describing a solution to a problem: what alternatives were
  there to solve it, why was this solution chosen? what made it the best
  choice? was it the optimal solution?
how will this section be structured and organised?
in what style will it be written?
next action - what to write first?

why this section? to demonstrate own achievements of error recognition.
the reader should know how it is done technically.

what goes into this section? the aspects of finding errors. finding errors
is not a straightforward trivial task - whenever something does not match
it is an error - doesn't work like that. instead, 
firstly, it needs to be made sure that it actually is an error.
meaning - not a recognition error, but a user error.
secondly, the type of error needs be identified.
see section \ref{sec:concept:sourcesoferror} (or handwritten page 58)
for sources of error.

how will this section be written?
technical - first describe how the error recognition integrates into the
recognition process, then how errors are identified.


\subsection{Error Processing}
\label{sec:hwre:errorprocessing}

%focus on technical aspects

why this section? 
what goes into this section?
  if describing a problem: why is the problem relevant.
  if describing a solution to a problem: what alternatives were
  there to solve it, why was this solution chosen? what made it the best
  choice? was it the optimal solution?
how will this section be structured and organised?
in what style will it be written?
next action - what to write first?


why this section? 
actually the 'handling' or 'processing' aspect could be 
described in the recognition section \ref{sec:hwre:errorrecognition} as well.
so this section is only for a better overview, for document structure, 
thematically they are the same section. thus they are put together under
Error Handling \ref{sec:hwre:errorhandling}.

what goes into this section?



