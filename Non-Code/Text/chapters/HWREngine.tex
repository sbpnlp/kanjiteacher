%%% Local Variables: 
%%% mode: latex
%%% TeX-master: "../KanjiHWR"
%%% End: 

\chapter{Handwriting Recognition Engine}
\label{chap:handwritingrecognitionengine}
The sections of this chapter are more the result of a brainstorming 
than a  proper thought-through chapter design.
xxx: see santosh2009 for mathematical stuff: nice description of what I'm doing

\section{Capturing Data}
this should deal with how the data are captured during the process
mouse coordinates and stuff

\section{Data Format}
how is the data structured? radicals, strokes, characters, xml-format

\section{Database}
where did I get it from? how many chars are in there?
how are they accessible? what format?

\section{Recognition Architecture}

\section{Stroke recognition process}
\subsection{From point list to vectors}
\subsection{Handling curves}
\subsection{Handling all that other stuff that requires some math}

\section{Radical recognition process}
\section{Character recognition process}

\section{Error recognition}
\subsection{How to deal with typical errors when writing Japanese}
\subsubsection{Error recognition}
focus on technical aspects
\subsubsection{Error handling}
focus on technical aspects

\section{HWR applied to e-learning of Japanese Kanji}
\subsection{Integration of HWR into e-learning app}
educational aspects / the e-learning view
\subsection{Error handling}
educational aspects / the e-learning view
