%%% Local Variables: 
%%% mode: latex
%%% TeX-master: "../KanjiHWR"
%%% End: 

\chapter{Handwriting Recognition Engine}
\label{chap:handwritingrecognitionengine}

%- Why this section? 
%   The purpose of this section is to describe the software module of the 
%   HWR engine.
%   Any important of the software should be described. This is the core of the
%   HWR part of the software and therefore needs description.
%   It would be off purpose, if non-technical aspects of the HWR Engine would be 
%   described.
% - What goes into this section?
%   The main content of this section is the complete functionality of the 
%   HWR engine.
%   It should become apparent how the HWR engine works.
%   * if describing a problem: why is the problem relevant.
%     The problem of HWR is relevant, because it is the crucial novelty of the
%     application. The combination of a Kanji learning tool and a HWR is a new 
%     thing. It enables user to practice writing Kanji. (maybe too high-level for
%     this section, move to intro / motivation section).
%   * if describing a solution to a problem: 
%     what alternatives were there to solve it?    
%     Instead of a HWR engine, the user could either not have a HWR engine and
%     write on paper or not practice writing the kanji.
%     why was this solution chosen? 
%     because it creates an additional benefit, that was not available before.
%     what made it the best choice? 
%     the fact that it writing on paper is impractical and not practicing does
%     not lead to the desired result.

%     was it the optimal solution? given there is only not practicing and
%     practicing on paper: yes.

% - How will this section be structured and organised?
%   The organisational structure of the section is already layed out in the
%   sections and subsections.
% - In what style will it be written?
%   The style of writing will be technical. A technical description, using
%   pseudocode.
% - Next action - what to write first?
%   The next part to write is the general introduction to this chapter

%xxx always: what alternatives were there?

\section{Data Capturing}
\label{sec:hwre:datacapturing}

Each handwriting recognition process begins with the data capturing.
The user's handwriting must be captured and fed into the system. The data 
capturing is therefore a crucial part of the whole process. In this system 
a GUI is used for the capturing of the pen movements on a writing surface.

%this should deal with how the data are captured during the process
%mouse coordinates and stuff

%- Why this section? 
%  The purpose of this section is to describe the data capturing process.
%  It would be off purpose, if it describes to much around - the devices, the
%  transmission. This should focus on the capturing as such.
%  It should not go into great deail.
%  
%- What goes into this section?
%  The main content of this section is the technical data capturing.
%  A description of how to
%  programmatically generate a pen trajectory from reading mouse coordinates.
%  
%  * if describing a problem: why is the problem relevant.
%    The relevance of the problem is evident: Without the user's writing,
%    there would not be any handwriting recognition.
%
%  * if describing a solution to a problem: what alternatives were
%    there to solve it, why was this solution chosen? 
%    The alternatives are not massively relevant here, since they come down
%    to using different input devices, which is discussed elsewhere.
%
%    what made it the best choice? 
%    the fact that this is just how it is done.
%    was it the optimal solution?
%    yes.
%
%- How will this section be structured and organised?
%  The organisational structure of the section begins with describing the mobile 
%  input GUI from a technical level.
%  Then the class in the background that actually captures the mouse events.
%  the capturing of the mouse events.
%  the storing of the data points in lists.
% 
%- In what style will it be written?
%  The style of writing will be technical. using pseudocode if required.
%  however the algorithm is not relevant here.
%
%- Next action - what to write first?
%  The next part to write is the one about the GUI
%  also the substructuring needs to be done as required. DONE

\subsection{Writing Surface} %the mobile GUI
\label{sec:hwre:writingsurface}
%NEXT ACTION - describe writing surface and it's GUI

The writing surface module, the view is split into two parts. The writing 
surface GUI and the writing surface background module. The technical design of 
the data input GUI is described in 
section~\ref{sec:arch:handwritingdatainputview}.

\subsubsection{Writing Surface GUI}
\label{sec:hwre:writingsurfacegui}
The GUI works with pen-down and pen-up events. It has a cross in the middle in 
order to partition the writing surface the same way, character practicing paper 
sheets are usually partitioned. The GUI class is listening to \emph{pen-down}, 
\emph{pen-up} and \emph{pen-move} events. These are the equivalent in the mobile
world for regular mouse-down, mouse-up and mouse-move events. However, there is 
one difference - the pen-move event can only be captured between a pen-down and a
pen-up event. An earlier conceptual idea for the HWR engine included using the
mouse-move events during the input of a character between the strokes. This 
could not be realised, however, because the series of pen-move events can only 
be captured when there has been a previous pen-down event and no pen-up event 
yet. The GUI captures the events and passes the point coordinates on to 
the background class.

\subsubsection{Writing Surfaces Background Module}
\label{sec:hwre:writingsurfacebackground}

When a pen-down event is detected, the background class of the GUI starts 
listening for the pen movement. All point coordinates and the time of their 
capturing are stored in two separate lists with the same indeces.
An alternative solution would have been to store a number of instances of a 
custom-made encapsulated Point class including the time stamp in one list only, 
however, using two separate lists resulted in increased speed.
Therfore, the point coordinates are stored in the frameworks Point class that 
does not account for time stamps. Therefore a separate list is used for the 
timestamps only.

The background module of the writing surface mainly administrates the capturing 
of the pen trajectory and sends it to the recognition module as soon as a 
stroke is finished. Therefore the system does not receive a separate signal 
indicating that the drawing of the character is finished. That design creates 
as segmentation problem that is solved with the \emph{clear} button and the 
\emph{clear} message. The segmentation of characters is left undetermined - only 
the beginning of it is determined through the \emph{clear} message that is sent 
to the moblie view from the controller - or the clear message that is sent to 
the controller because the user clicked the \emph{reset} button.
After a stroke is finished the according point and timestamp sequences are 
passed to the main part of the HWR engine. 

\section{Data Format}
\label{sec:hwre:dataformat}

% - Why this section? 
%   The purpose of this section is to give the reader an idea of how the captured
%   data is structured. That means concretely the character format, describing
%   the HW trajectories. This is necessary in order to be able to explain the
%   mode of operation of the HWR engine in detail.

%   It would be off purpose, if the section contained a description of the 
%   lexical data, which is described in a different section.

% - What goes into this section?
%   The main content of this section is what is stated under purpose:
%   A description of how the captured data is structured.

%   * if describing a problem: why is the problem relevant.
%     The task of handwriting recognition is essentially a task 
%     working with data.
%     Therefore the structure of the data is one of the crucial points.
    
%   * if describing a solution to a problem: what alternatives were
%     there to solve it, why was this solution chosen? 

%     The solution was chosen because of its clarity, simplicity and 
%     accessibility, also because of its interoperability with potential 
%     other systems.
%     what made it the best choice? was it the optimal solution?
%     For the purpose followed with the system - yes, it was the optimal solution,
%     because it does not require much effort to parse the stored data
%     and it contains all necessary information in a structured manner.
    
% - How will this section be structured and organised?
%   The organisational structure of the section contains:

%   A general description of the XML format - why it was used as opposed to
%   other possible formats like unipen and inkml.
%   show requirements!
%   show process of how I came to current data formats.
%   include character models 1-4 on pages. - however, not everything in one go,
%   but rather in the individual sections if possible.
%   in the end, a radical has its own format that is unchanged, even
%   if internal structure of a stroke is changed.
%   (unipen is only text based, inkml does not help here, but the system allows
%   for exchange of the custom format with those)

%   data format of and representation of 
%   point, stroke, box, radical, character s. 20-23, 25f

% - In what style will it be written?
%   The style of writing will be technical - a description of the XML format.
%   Pointer to code sample in Appendix.

% - Next action - what to write first?
%   The next part to write is the actual subsection 
%   structure of this section. DONE
%   Think about requirements and write them down.


\subsection{Requirements of the Data Format}
\label{sec:hwre:requirementsofdataformat}

%show requirements!
%think about differences between static data format and runtime data format.

The data format for the recognition process underlies a number of requirements.
Firstly, it has to be stated that two main data formats are necessary. 
One for the actual recognition process as a data structure in the RAM.
Secondly, there needs be a storage format for the data base that represents
the handwriting of a character, radical, stroke, point list or simple point.
The requirements for both data structures are:
\begin{itemize}
  \item \textbf{Expressiveness}: The data structure should be able to 
        fully represent a complete character and all sub-elements that belong 
        to it. That requirement is due to the structural approach to handwriting
        recognition that is the basis for the error handling.

  \item \textbf{Well-definedness}: The information structure should be 
        unambigous, two different characters must have a different structure.
\end{itemize}
The storage structure has some additional requirements:
\begin{itemize}
  \item \textbf{Accessibility}: The data should be formatted in a way that a 
        human editor can access the data, but it should also be prepared for 
        programmatical access.
  \item \textbf{Well-formedness}: There should be a way to check if the data is 
        well-formed
  \item \textbf{Parsabilty}: It should be possible to create the run-time data 
        structure from the storage data structure.
\end{itemize}
The run-time data structure should in addition provide
\begin{itemize}
  \item \textbf{Serialisability}: It must be possible to create the storage 
        data structure from the run-time data structure.
\end{itemize}
Any combination of two formats used should meet the above requirements, 
at best in a uniform way. That is, any combination of storage format and 
run-time format should ideally be compatible without the need of a complex
data format converter.

%xxx how much do I really have to say about the unipen format?
%xxx does it really not meet the requirements?

% Both:
% \emph{expressiveness}
% \emph{well-definedness}

% Storage:
% \emph{accessibility}
% \emph{well-formedness}
% \emph{parsabilty}

% Run-time:
% \emph{serialisability}

\subsection{Existing Handwriting Formats}
\label{sec:hwre:existingformats}

There are some existing formats for the description of handwriting. Their 
intended use is for handwriting recognition systems. The formats that will be
reviewed in this section are the \emph{Microsoft ISF} format in 
section~(\ref{sec:hwre:msisfformat}), which is a binary format.
The other formats are text based and human-readable. In this short review,
the formats will be checked against the requirements defined in 
section~(\ref{sec:hwre:requirementsofdataformat}). We will consider 
\emph{InkML} in section~(\ref{sec:hwre:inkml}), 
\emph{UNIPEN} in section~(\ref{sec:hwre:standardunipenformat}), 
\emph{hwDataset} in section~(\ref{sec:hwre:hwdataset}) and 
\emph{UPX} in section~(\ref{sec:hwre:unipenformatupx}). 
The central question is, if one of these formats meets the above requirements in 
a sufficient way. This is unclear, because the most of the formats have been 
originally developed mainly for alphabetic scripts or for usage scenarios 
different from the one proposed in this chapter.

\subsubsection{The Microsoft Ink Serialized Format (\emph{ISF})}
\label{sec:hwre:msisfformat}

The \emph{Ink Serialized Format} (ISF) format is owned by the 
Microsoft Corporation. However, it is not a proprietary format, the 
specification is freely available. Since the format is purely binary, 
it can not meet one of the requirements for the storage data structure:
It is not feasible for a human to read and write files in that format,
therefore ISF does not provide the necessary 
\emph{accessibility}~\shortcite{Microsoft2007}. The format could still be used 
as a run-time data structure, if it meets the requested criteria.
The requirement of \emph{serialisability} is generally met. A binary format 
allows for fast and reliable programmatical access to the data, it can be stored
in binary files on a data storage medium. The problem that accrues from this 
type of serialisation is that the storage format, the format used in a file or 
database, would again not be human-editable and therefore fail to meet the 
\emph{accessibility} requirement. 
In order to provide a serialisation that results in a human-readable format, 
a format converter would be needed. Another problem originates from the 
ISF format specification: \\
It accounts only for the description of mouse coordinates or pen trajectories, 
but does not offer any structures for linguistic information. 
It fails to meet the ideal, a unified format for linguistic and graphic 
information, as it is lacking some degree of \emph{expressiveness}.

\subsubsection{InkML}
\label{sec:hwre:inkml}

InkML is a markup language for describing electronic pen trajectories. 
Hence the name \emph{ink}. \emph{ML} stands for markup language, as is customary 
for XML-based and SGML-based languages. The specification is currently in a 
draft status, maintained by the World Wide Web Consortium 
(W3C)~\shortcite{W3C2006}. 
Judged from the viewpoint of the defined requirements, InkML is a candidate
for the storage format. As an XML format it automatically fulfils the
criterion of \emph{well-formedness}: Any piece of InkML code can be evaluated 
with common techniques, since InkML has a well-defined scheme description.

The main elements in the simplest form of InkML are the \textbf{<ink>} and 
the \textbf{<trace>} element. These elements allow for a very simple form 
of a pen trajectory, basically a flat list of point coordinates. The <ink> 
element is the root element of any InkML code. The <trace> element holds
point coordinates.
\begin{xmlcode}
  \begin{lstlisting}[emph={trace},emphstyle=\textbf,caption={Demonstration of the \emph{trace} tag},label=lst:TraceTag]
    <ink>
      <trace>
        282 45, 281 59, 284 73, 285 87, 287 101, 288 115, 290 129,
        291 143, 294 157, 294 171, 294 185, 296 199, 300 213
      </trace>
    </ink>
  \end{lstlisting}
\end{xmlcode}
Listing~(\ref{lst:TraceTag}) covers the general gist of the InkML format.
Despite being an XML format, it is a flat format, each value pair, 
separated by commas, represents one point in a coordinate notation.

It is possible to add \emph{time} information to the trace. In order to do so,
a time channel needs to be defined. Listing~(\ref{lst:TimeChannel}) shows the 
definition and use of a time channel. The example shows a time channel whose 
values for a given point are the relative to the timestamp refered to to by 
\emph{\#ts001}~\shortcite{W3C2006}. Below there are two \emph{timestamp} tags.
The first one has the ID \emph{ts001} and is referred to by both the
time channel and the second time stamp that defines a time offset to time stamp
\emph{ts001} with the \emph{timeOffset} attribute.
The possibility to define time channels enables the InkML format to hold 
information about the time at which a sample point has been taken.
This is useful for the recognition mechanism used in this application,
because it compares point coordinates as well as time stamps.
\begin{xmlcode}
  \begin{lstlisting}[emph={timestamp,channel},
                     emphstyle=\color{blue}\textbf,
                     emph={[2]name,type,units,respectTo,time,
                              timeOffset,timestampRef},
                     emphstyle={[2]\color{red}},
                     caption={Demonstration of the \emph{time channel}},
                     label=lst:TimeChannel]
    <channel name="T" 
             type="integer"
             units="ms"
             respectTo="#ts001"/>

    <timestamp xml:id="ts001" 
               time="1265122414000"/>
    <timestamp xml:id="ts002" 
               timeOffset="600000" 
               timestampRef="#ts001"/>
  \end{lstlisting}
\end{xmlcode}
The InkML data structure fulfills several of the requirements necessary for a 
storage data structure. InkML is 
\begin{itemize}
  \item \emph{well-defined}: Any point sequence that has different coordinate
        values than another point sequence can be distinguished from the other.
        
  \item \emph{accessible}: As an XML format it can be handled programmatically,
        but is also human-readable and can be edited by a human with a simple
        interface like a text editor.

  \item \emph{well-formed}: As an XML format it can be validated according to 
        it's specification.

  \item \emph{parsabale}: As an XML format, InkML does not need to be parsed with
        custom-made methods, since most modern high-level languages offer
        an access method to XML tree structures.
\end{itemize}
However, InkML is lacking some \emph{expressiveness}. Only pen trajectories and
their time stamps can be expressed in the InkML format. InkML is not designed 
to hold any of the other information compulsory for the structural handwriting
recognition. In the case of character recognition, it needs to be able to 
account for more than coordinate points and time stamps. There needs to be a 
way to encode structural information about the characters and sub-elements of 
the characters. InkML does not to be a sufficient format for the given task.

\subsubsection{The Standard UNIPEN Format}
\label{sec:hwre:standardunipenformat}

The standard UNIPEN format specification is a file format definition for a flat
text file. It contains tags in dot-notation. For example the tag \emph{.COMMENT}
means that the following free text should be ignored, the tag \emph{.KEYWORD} 
is used to define a new keyword, while the tag \emph{RESERVED} states that
the text that comes after that tag is a reserved word within the UNIPEN format.
The format is self-defined with the three keywords above. Any new keyword is
defined with \emph{.KEYWORD}~\shortcite{Guyon1994,Unipen}.

As a data structure for the purpose of a handwriting recognition, UNIPEN can 
serve as a storage format, which is the primary purpose for the existence of the
format. The UNIPEN format accounts for both pen trajectories as well as 
information about the characters that have been written.
\begin{texcode}
  \begin{lstlisting}[emph={COMMENT,COORD,SEGMENT},emphstyle=\textbf,caption={Demonstration of the \emph{UNIPEN} format},label=lst:unipen]
.COMMENT #################################################################### 
.COORD   X Y

.SEGMENT TEXT 235:0-297:9 OK "Kurosu Masaaki"
.SEGMENT CHARACTER 235:0-255:9 OK  "JISx3975_'Kuro'"

.PEN_DOWN 
  486 -1456 
  488 -1454 
  490 -1452 
  488 -1450 
  488 -1450 
  486 -1452 
  480 -1456 
  474 -1466 
  464 -1480 
  452 -1492 
  440 -1506 
  428 -1524
.PEN_UP 
  406 -1556 
  394 -1574 
  384 -1590 
  374 -1602
.PEN_DOWN 
  \end{lstlisting}
\end{texcode}
Listing~(\ref{lst:unipen}) is an excerpt from a data file created 
by~\shortcite{Unipen}. It shows an example of a UNIPEN data structure.
The \emph{.COORD} tag defines the structure of the pen coordinates. 
The \emph{.SEGMENT} tag with the \emph{TEXT} modifier informs about the
full text of the next number of segments. The \emph{.SEGMENT} tag with the
\emph{CHARACTER} modifier informs about the character that is represented
with the sequence of pen coordinates. The other information given is the
start and end position within the file and the character code in JIS encoding.
The \emph{.PEN\_DOWN} tag can be understood as a flag. All pen coordinates 
following this tag have been captured as \emph{pen down} coordinates.
The tag \emph{.PEN\_UP} sets the opposite flag: The coordinates stated after 
this tag were captured while the pen was not touching the writing surface.
The last \emph{.PEN\_DOWN} tag is the beginning of a new coordinate sequence
for the next stroke. From a requirements point of view the UNIPEN provides a
high standard, as it meets a great number of requirements for the storage 
structure. \\
The UNIPEN format is:
\begin{itemize}
  \item partially \textbf{expressive}: UNIPEN can represent a complete character
        and the pen trajectory along with it. It does however, not account for
        capturing the time information. At a constant sample rate the time stamps
        of the individual points could theoratically be calculated, 
        if time stamp of the first point is encoded in the meta information. 
        However, that solution is not optimal, because it needs constant
        sampling, even between a \emph{pen-up} and a \emph{pen-down} event.
        There are input devices that do not offer those coordinates and the 
        recognition system presented here does not rely on them for that reason.
        The examples given are encoded in JIS, but it seems possible and feasible
        to use the format with unicode encoding.

  \item \textbf{well-defined}: The UNIPEN is unambigous, two different 
        characters do have a different data structure, because the JIS code 
        can serve as an ID.

  \item \textbf{accessibile}: The data resides in flat text files, designed for 
        human editing. It is accessble programmatically, too, but has a weakness
        on that requirement compared to an XML-based format.

  \item \textbf{parsable}: The flat file format can be parsed easily. Due to its
        procedural nature it is also possible to create a run-time data structure
        that is conceptually based on the storage structure and therefore
        serialisable.
\end{itemize}
The main weakness of the flat format file is the lack of \emph{well-formedness}.
There is no automatic way to check a document against a predefined format 
specification. It can not be assessed if a file is well-formed.
Additionally, the expressive power of the UNIPEN format would be seriously 
challenged if it should provide for sub-structures of characters.
In order to do so, it would be necessary to create new tag definitions for the 
Radicals of the Kanji. \shortciteANP{Agrawal2005} \citeyear{Agrawal2005} 
describe some more shortcomings of the UNIPEN format. It can be concluded that 
the standard UNIPEN format is not suitable for the task given. 

\subsubsection{Handwriting Dataset (\emph{hwDataset})}
\label{sec:hwre:hwdataset}

The \emph{Handwriting Dataset} (hwDataset) is an XML format that is a complement
to InkML. It is inspired by the UNIPEN format. The \emph{hwDataset} format 
attempts to close the gap between pure ink data and the annotations that are 
needed for handwriting recognition~\shortcite{Bhaskarabhatla2004}. 
The format contains three main parts. The \emph{datasetInfo}, 
the \emph{datasetDefs} and the \emph{hwData}. The datasetInfo element holds
any metadata, like \emph{name}, \emph{category} and the like.
The dataSetDefs encompasses information about \emph{data sources}, different
\emph{writers} and their features like \emph{handedness}, \emph{gender}, 
\emph{age}. The \emph{hwData} element in the XML code organises the 
handwriting data hiearchically. Each hierarchy level contains one or 
more \emph{hwTrace} elements. The hwTrace element refers to an InkML file, 
containing the actual handwriting data. A detailed description of the hwDataset 
format can be found in~\shortcite{Bhaskarabhatla2004}. 

The format provides an XML scheme description and thus meets the 
\emph{well-formedness} requirement. It is \emph{well-defined}, 
because it accounts for unique structures. The \emph{parsabilty} is not an 
issue for any XML format, since methods for accessing XML trees are provided 
by modern programming libraries.
The \emph{accessibility} is provided, the format has a clearly defined structure
that can easily be understood by humans.
The hwDataset format can structure many desirable information for the handwriting
learning system proposed in this chapter. 
From an \emph{expressiveness} viewpoint it can create substructures for 
different parts of a sequence of pen trajectories. The \emph{hierarchy level} (H)
of a trace accounts for the expressive power necessary. The H(n) elements 
of the hwData elements are designed to hold meaningful names like 
\emph{PARAGRAPH} or \emph{WORD}. It may be possible to use the format even to 
denote different parts of the same character. There is no reason why one should
not introduce an H(n) definition for \emph{RADICAL} or \emph{GRAPHEM}.

\subsubsection{The UNIPEN XML Format (\emph{UPX})}
\label{sec:hwre:unipenformatupx}

The format \emph{Unipen for XML} (UPX) can be seen as an XML version of the
UNIPEN format. The UNIPEN standard described in 
section~(\ref{sec:hwre:standardunipenformat}) does not bear any resemblance
with the InkML format shown in section~(\ref{sec:hwre:inkml}).
Both formats fulfil a number of the requirements presented in 
section~(\ref{sec:hwre:requirementsofdataformat}), but both formats fail to 
fulfil them all. The hwDataset format described in 
section~(\ref{sec:hwre:hwdataset}) solves many of the problems UNIPEN faces,
as it brings together a superset of the expressiveness of UNIPEN with
the well-defined InkML format in a uniform way.
According to~\shortciteANP{Agrawal2005} \citeyear{Agrawal2005} hwDataset
was designed to support new data collection. The motivation to create another
format came about from the fact that existing UNIPEN resources should be made
available in a modern XML-based format that unifies InkML and UNIPEN.
hwDataset comes very close to this aim, but does not necessarily account for a 
conversion of UNIPEN resources. \shortciteANP{Agrawal2005} \citeyear{Agrawal2005}
attempt to create a format that allows for a well-defined conversion from UNIPEN.
The UNIPEN foundation considers UPX as the new de facto standard format for 
storing annotated databases of online pen input~\shortcite{UnipenConsortium2006}.

UPX is a multi file format. It resembles hwDataset in the way that it comprises 
both meta information and data annotations as well as pure InkML data in 
separate files. Within the UPX \emph{hwData} element there can be any number
of \emph{hLevel} elements in order to define the hierarchy of the trajectory.
Listing~(\ref{lst:hLevelinupx}) shows an example 
after~\shortcite{UnipenConsortium2006}. In the listing, the \emph{hLevel} tag
is demonstrated. The word \emph{sexy} is described hierarchically.
Firstly, in the outermost \emph{hLevel} tag, the \emph{level} attribute specifies
a \emph{WORD}. Then the inner \emph{hLevel} elements specify a lower level
of hierarchy, the characters. Note that not all characters are defined
individualy in this example.
The \emph{hwTraces} elements surround a number of \emph{traceView} elements that
in turn point to the corresponding positions in an InkML file that holds the 
pen trajectory. The meta data information that can be stored in UPX has a 
similar expressive power as the hwDataset format.
\shortciteANP{Madhvanath2006} \citeyear{Madhvanath2006} give a full format 
description of the UPX format in their technical report.

\begin{xmlcode}
  \begin{lstlisting}[emph={hLevel,hwTraces},
                     emphstyle={\color{blue}\textbf},
                     emph={[2]id,level},
                     emphstyle={[2]\color{red}},
                     caption={Demonstration of the \emph{hLevel} tag in UPX},
                     label=lst:hLevelinupx]
    <hLevel id="WORD_0" level="WORD">
      <label id="WORD_0">
        <alternate rank="1" score="1"/>sexy</alternate>
      </label>
      <hwTraces> 
        <inkml:traceView traceRef="w0629.inkml#dataSet_0_traces" 
                         from="22" 
                         to="24"/>
      </hwTraces>

      <hLevel id="CHAR_0" level="CHAR">
        <label id="CHAR_0">
          <alternate rank="1" score="1"/>e</alternate>
        </label>
        <hwTraces> 
          <inkml:traceView traceRef="w0629.inkml#dataSet_0_traces" 
                           from="22:91" 
                           to="22:161"/>
        </hwTraces>
      </hLevel>

      <hLevel id="CHAR_1" level="CHAR">
        <label id="CHAR_1">
          <alternate rank="1" score="1"/>y</alternate>
        </label>
        <hwTraces> 
          <inkml:traceView traceRef="w0629.inkml#dataSet_0_traces" 
                           from="24:61" 
                           to="24:162"/>
        </hwTraces>
      </hLevel>
    </hLevel>
  \end{lstlisting}
\end{xmlcode}

It seems as if the UPX format was able to fulfil all requirements.
The UPX format fulfils the requirements of
\begin{itemize}
  \item \emph{Well-definedness}: As an XML structure providing IDs and names for 
        each element and substructure, the format is well-defined.

  \item \emph{Accessibility}: As an XML format it is both human and machine
        readable and editable.

  \item \emph{Well-formedness}: As an XML format with a defined scheme it can
        be validated for its syntactic features.

  \item \emph{Parsabilty}: As an XML format it can be easily parsed into a
        tree structure.

  \item \emph{Expressiveness}: The format provides the missing link between the
        standard UNIPEN format and the InkML format. Additionally, it can 
        describe hierarchical structures.
\end{itemize}
Since any XML format translates directly into a tree structure, the data 
structure for the run-time does not need to be defined separately.
Conversing pen trajectories in InkML format into a data stream has been 
studied~\shortcite{Keshari2008}. These techniques however, are not necessary 
for the proposed system. The data exchange is provided as technique available
within the framework without any conversion (see 
section~\ref{sec:arch:webservice}).

\subsection{Data Format Description}
\label{sec:hwre:dataformatdescription}

  % A general description of the XML format - why it was used that way,
  % given the requirements

  % data format of and representation of 
  % point, stroke, box, radical, character s. 20-23, 25f

  % show process of how I came to current data formats.
  % include character models 1-4 on pages. - however, not everything in one go,
  % but rather in the individual sections if possible.
  % in the end, a radical has its own format that is unchanged, even
  % if internal structure of a stroke is changed.
  % (unipen is only text based, inkml does not help here, but the system allows
  % for exchange of the custom format with those)

The data format description could trivially be described with the UPX 
description. In this case however, UPX needs to be adapted to the special needs.
This section describes the adaption of UPX to the requirements of an on-line
handwriting recognition for Kanji characters in a learning environment.
The section describes both the file format and the run-time data structure.

\subsubsection{Point Data Format}
\label{sec:hwre:pointdataformat}

In UPX, points are not represented directly, but in an outsourced InkML file,
where they are part of the InkML standard pen trajectory definition. In InkML
files, points are stored in traces as specified in the InkML format description.
\begin{xmlcode}
  \begin{lstlisting}[emph={traceFormat,channel,timestamp},
                     emphstyle=\color{blue}\textbf,
                     emph={[2]name,type,units,respectTo,id,time},
                     emphstyle={[2]\color{red}},
                     caption={Definition of the trace format},
                     label=lst:traceformat]
    <timestamp xml:id="ts001" 
               time="1265122414000"/>
    <traceFormat xml:id="kanjiStrokeTrace">
      <channel name="X" type="decimal" />
      <channel name="Y" type="decimal" />
      <channel name="T" type="integer"
                        units="ms"
                        respectTo="#ts001" />
    </traceFormat>
  \end{lstlisting}
\end{xmlcode}
Listing~(\ref{lst:traceformat}) shows the trace format description as used in
the system. The format consists of three channels that are given in ordered 
sequence. There are no intermittent channels. The first two channels are \emph{X}
and \emph{Y} for the pen coordinates. The third channel is \emph{T} for a time 
stamp. The pen coordinates can be given as \emph{decimal} numbers, while the
time stamp (in milliseconds) must be an \emph{integer}.
The \emph{respectTo} attribute in the time channel indicates that all values
must be interpreted as offsets to the time stamp with the id \emph{ts001}.

An actual trace would then contain three values, each value triple separated by
a comma. The trace format definition ensures the correct interpreation of the 
values.
\begin{xmlcode}
  \begin{lstlisting}[emph={trace,channel},
                     emphstyle=\color{blue}\textbf,
                     emph={[2]name,type,units,respectTo,id},
                     emphstyle={[2]\color{red}},
                     caption={A sample trace},
                     label=lst:traceinterpreation]
    <trace id="id123abc">
      45 76 0, 3 5 100, 4 -8 200, 4 -5 300,
      9 -2 400, ...
    </trace>
  \end{lstlisting}
\end{xmlcode}
Listing~(\ref{lst:traceinterpreation}) shows a trace. Given the file contains
the trace format definition shown in listing~(\ref{lst:traceformat}), the trace 
in listing~(\ref{lst:traceinterpreation}) is interpreted in 
table~(\ref{table:sampletraceinterpretation}). 

\begin{table}[htbp]
  \begin{tabular}{|l||l|l|l|p{200pt}|}
    \hline
    Trace & Channel X & Channel Y & Channel T & Comment \\
    \hline
    45 76 0 & 45 & 76  & 1265122414000 & The first trace initialises 
                                         the values. The time stamp value is
                                         read from timestamp \emph{ts001}. \\
    \hline
    3 5 100 & 48 & 81 &  1265122414100 & The delta values are added to the
                                         existing data. \\
    \hline
    4 -8 200 & 52 & 73 &  1265122414200 & The time stamp values are relative to
                                          \emph{ts001}, not to the last time 
                                          stamp, unlike the pen coordinate 
                                          deltas that are added to the current 
                                          value trace by trace. \\
    \hline
    4 -5 300 & 56 & 68 &  1265122414300 & Negative values are treated as regular
                                          delta values. \\
    \hline
    9 -2 400 & 65 & 66 &  1265122414400 & The values change as expected. \\
    \hline
  \end{tabular}
\caption{Sample trace interpreation}
\label{table:sampletraceinterpretation}
\end{table}

\paragraph{The run-time data structure} for a point holds the coordinate values 
and an integer value for the time stamp. The conversion between the run-time 
instance and the trace format is trivial and needs no description.

\subsubsection{Stroke Data Format}
\label{sec:hwre:strokedataformat}

There is no explicit format for a stroke in the storage structure.
A stroke is simply the trace between a pen-down event and the following pen-up
event. All the points captured between form one stroke.
Thus, in the InkML file format a stroke is simply a trace.
If a greater number of strokes comes together, each one of them will be a trace
and they can be groupe together as a tracegroup element in InkML.
It would be possible to create a separate level for strokes, as demonstrated
in listing~(). However, since every \emph{hLevel} element for a stroke would 
contain exactly one \emph{hwTrace} element, it seem redundant to do so.
Listing~(\ref{lst:strokeasindividualelement}) shows a potential data structure
for a stroke. As stated befere, this data structure is redundant in the way, 
that an \emph{hLevel} definition would contain only exactly one trace.
If a trace is interpreted as a stroke, the additional \emph{hLevel} for strokes
becomes unnecessary.
xxx check this part for nonsense xxx yyy xxx yyy
\begin{xmlcode}
  \begin{lstlisting}[emph={hLevel,hwTraces,label,alternative},
                     emphstyle=\color{blue}\textbf,
                     emph={[2]level,labelSrcRef,labelType,traceRef,from,to},
                     emphstyle={[2]\color{red}},
                     caption={A sample of how a stroke could be represented},
                     label=lst:strokeasindividualelement]
    <hLevel level="stroke" id="STROKE1">
    <!-- The first stroke in this Radical of handwriting. -->
      <label id="STROKE1label" labelSrcRef="labelref_DW" labelType="truth">
        <alternate>[unicode value]</alternate>
      </label>
      <hwTraces>
        <inkml:traceView traceRef="/example.inkml" from="12" to="12"/>
      </hwTraces>
    </hLevel>
  \end{lstlisting}
\end{xmlcode}

\paragraph{The run-time data structure} for a stroke is generated from a trace.
The actual data structure is purely a list of point instances. 
Any other (numerical) values are computed from the point list. 
Section~(\ref{sec:hwre:strokerecognitionprocess}) goes into details of what 
methods the data structure provides.

\subsubsection{Radical Data Formant}
\label{sec:hwre:radicaldataformat}

The Radical data format is a data structure that works with the \emph{hLevel}
element for description.
Conceptually, a Radical is a hierarchical element in the composition of a Kanji
character. With the UPX format specification in hand it is fairly 
straightforward to define an appropriate data structure.
Listing~(\ref{lst:radicaldataformat}) shows a UPX data structure for a radical
of a Kanji. The Radical \cjk{木} consits of four strokes.
Since strokes are not represented explicitely in the data format,
there are four \emph{traveView} elements under the \emph{hwTraces}.

xxx check this part for nonsense xxx yyy xxx yyy
\begin{xmlcode}
  \begin{lstlisting}[emph={hLevel,hwTraces,label,alternative},
                     emphstyle=\color{blue}\textbf,
                     emph={[2]level,labelSrcRef,labelType,traceRef,from,to},
                     emphstyle={[2]\color{red}},
                     caption={A sample of how a stroke could be represented},
                     label=lst:radicaldataformat]
    <hLevel level="radical" id="RADICAL75">
    <!-- The first Radical in this Kanji character of handwriting. -->
      <label id="RADICAL75label" labelSrcRef="labelref_DW" labelType="truth">
        <alternate>[unicode value, if applicable]</alternate>
      </label>
      <hwTraces>
        <inkml:traceView traceRef="/example.inkml" from="12" to="12"/>
        <inkml:traceView traceRef="/example.inkml" from="13" to="15"/>
        <inkml:traceView traceRef="/example.inkml" from="16" to="19"/>
        <inkml:traceView traceRef="/example.inkml" from="20" to="23"/>
      </hwTraces>
    </hLevel>
  \end{lstlisting}
\end{xmlcode}

\paragraph{The run-time data structure} of a Radical is a collection of strokes.
All values for recognition of a Radical are computed from the strokes
by using different permutations of the strokes to match an entry from the
database.

\subsubsection{Character Data Format}
\label{sec:hwre:characterdataformat}

The character data format is an ordered list of radicals, according to their
standard sequence of writing within the character. In 
listing~(\ref{lst:charaterdataformat}) the character \cjk{東} is shown.
It consists of the radicals \cjk{日} and \cjk{木}. The structure is totally
straightforward and exploits the advantages of the UPX format.
The outermost \emph{hLevel} element is used for the character and additional
descripitons, the two inner \emph{hLevel} elements are used for the two
Radicals of that characters. Each \emph{traceView} in the Radicals represents
a stroke. Thus, it can be seen that both Radicals constist of four strokes each.
xxx check this part for nonsense xxx yyy xxx yyy
\begin{xmlcode}
  \begin{lstlisting}[emph={hLevel,hwTraces,label,alternative},
                     emphstyle=\color{blue}\textbf,
                     emph={[2]level,labelSrcRef,labelType,traceRef,from,to},
                     emphstyle={[2]\color{red}},
                     caption={A sample of how a stroke could be represented},
                     label=lst:charaterdataformat]
    <hLevel level="character" id="CHARACTER71">
    <!-- The character 71 consists of the Radicals 72 and 75 -->
      <label id="CHARACTER71label" labelSrcRef="labelref_DW" labelType="truth">
        <alternate>U+6771</alternate>
      </label>

      <hLevel level="radical" id="RADICAL72">
      <!-- The first Radical in this Kanji character of handwriting. -->
        <label id="RADICAL72label" labelSrcRef="labelref_DW" labelType="truth">
          <alternate>U+65E5</alternate>
        </label>
        <hwTraces>
          <inkml:traceView traceRef="/example.inkml" from="0" to="3"/>
          <inkml:traceView traceRef="/example.inkml" from="4" to="7"/>
          <inkml:traceView traceRef="/example.inkml" from="8" to="9"/>
          <inkml:traceView traceRef="/example.inkml" from="10" to="11"/>
        </hwTraces>
      </hLevel>

      <hLevel level="radical" id="RADICAL75">
      <!-- The second Radical in this Kanji character of handwriting. -->
        <label id="RADICAL75label" labelSrcRef="labelref_DW" labelType="truth">
          <alternate>U+6728</alternate>
        </label>
        <hwTraces>
          <inkml:traceView traceRef="/example.inkml" from="12" to="12"/>
          <inkml:traceView traceRef="/example.inkml" from="13" to="15"/>
          <inkml:traceView traceRef="/example.inkml" from="16" to="19"/>
          <inkml:traceView traceRef="/example.inkml" from="20" to="23"/>
        </hwTraces>
      </hLevel>
    </hLevel>
  \end{lstlisting}
\end{xmlcode}

\section{Database}
\label{sec:hwre:database}

% - Why this section? 
%   The purpose of this section is to explain the structure and production of 
%   the database.   That includes both pen trajectory and lexical data.
%   That is a relevant information, because the system provides these information,
%   therefore they must come from somewhere. In order to explain some of the
%   error recognition processes it is necessary to have these information first.

%   It would be off purpose, if anything else goes into this section.
%   The xml format is described elsewhere, the lexical charater DB comes from 
%   jim breen (cite his paper) nothing else needs be in here.

% - What goes into this section?
%   The main content of this section is a description of the character database.
%   The jim breen stuff can be pointed to, but the production process of the 
%   self made data should be described.

%   * if describing a problem: why is the problem relevant.
%     The problem is relevant, because without a character database, there won't
%     be any recognition - the system wouldn't know anything.

%   * if describing a solution to a problem: what alternatives were
%     there to solve it, why was this solution chosen? 

%     Alternatives - see alternatives of data format!
%     Technical alternatives - proper DB instead of flat-file possible,
%     but not necessary. It is actually not that much data,
%     even if all characters are included.
   
%     what made it the best choice? was it the optimal solution?
%     the fact that the alternatives do not offer a better solution.

% - How will this section be structured and organised?
%   The organisational structure of the section will contain
%   1. description of the technical format of the character data base
%      two flat files - indexed via the actual kanji character in unicode
%      format, since it is unique. this is a character-centred application!
%      how are they accessible? what format?
%      this can be seen more generic as 'the lexicon', not necessarily
%      the linguistic information about the characters, 
%      but rather the whole lexicon, including the point sequences.

%   2. discussion of alternatives: relational DB or only one flat file
%      two flat files are better, because of updates and other-language versions
%      of the jim breen lexicon. leave it unchanged! otherwise you'd need a
%      converter!

%   3. Description of the production of the lexicon.
%      it was not just taken from j.b. but it was intervowen?! (verflochten) 
%      with the trajectories. where did I get these from? 
%      how many chars are in the two dictionaries?

% - In what style will it be written?
%   The style of writing will be a technical description of how it is done.

% - Next action - what to write first?
%   The next part to write is to hash out the actual subsections.

\yyy this is the actual next section!

\section{Recognition Architecture}
\label{sec:hwre:recognitionarchitecture}

% - Why this section? 
%   The purpose of this section is to give the reader an overview of the whole
%   recognition system. the recognition system! not the whole system, 
%   not the learning part.

%   It would be off purpose if other stuff would be described.
%   stick to the main focus - architecture of the pure recognition.
%   If would be off purpose if it was too detailed.
%   The actual recognition process is discussed in subsequent sections!
%   don't go into too much detail.

% - What goes into this section?
%   The main content of this section is an overview of the recognition 
%   architecture. This should not be confused with the overview of the
%   whole Kanji Coach system. it is more detailed and focused.

%   One big graphic shows the overview of the different modules.
  
%   * if describing a problem: why is the problem relevant.
%     not really a problem - every system has an architecture.
   
%   * if describing a solution to a problem: what alternatives were
%     there to solve it, why was this solution chosen? 

%     this solution was chosen after review of around 50 HWR systems.
%     the discussion of those can be found in chapter \ref{chap:onlinehwr}.
%     this section only uses pointers to that chapter.

%     what made it the best choice?
%     certainly not the lacking of alternatives, but the mix between 
%     possibilities to programmatically interact with the recognition process
%     and performance of the process.

%     was it the optimal solution?
%     yes and no.
%     it depends on the focus.
%     big discussion! WHY this architecture?
%     roughly following X and Y and Z,
%     however, not creating an optimal handwriting recognition engine.
%     research suggests that HMM models are more useful etc.
%     however - the type of error recognition desired,
%     requires a structural system.
%     it is only new in research (find that one paper 
%     that does that bloody thing) to blend HMM and structural models
%     in a hybrid HWR system, simimlar to hybrid MT systems.
%     this is interesting for outlook - point there to idea
%     HMM vs structural vs error recognition

% - How will this section be structured and organised?
%   The organisational structure of the section will be as follows:
%   1. architecture: graphic and explanatory description
%   -> Modules and parts of HWR, create graphic.

%   s. 18 zeichen, punkt usw. UML diagramme.
     
%   2. brief discussion about alternative architectures

% - In what style will it be written?
%   The style of writing will be technical
% - Next action - what to write first?
%   The next part to write is to hash out the substructure in subsections.

\section{Stroke Recognition Process}
\label{sec:hwre:strokerecognitionprocess}

% - Why this section? 
%   The purpose of this section is to describe the recognition process of a strokes
%   This is necessary, because it is a crucial part of the recognition process
%   as such, to describe which is the purpose of this chapter

%   It would be off purpose, if the radical or character recognition process 
%   would be described, too.
%   It would be off purpose, if the stroke recognition process would not be 
%   layed out in its entirety.

% - What goes into this section?
%   The main content of this section is the details of the stroke recognition
%   process. The question how it is done is central and crucial.

%   * if describing a problem: why is the problem relevant.
%     stroke recognition is not relevant as such, it becomes relevant because
%     of the recognition approach chosen, which is similar to the one of CITE

%   * if describing a solution to a problem: what alternatives were
%     there to solve it, why was this solution chosen? 
%     there are alternatives, many of which are possible. mathematical methods 
%     and formats. since roughly following \shortciteANP{Nakagawa2008}'s
%     \citeyear{Nakagawa2008} approach - the discussion of the different 
%     approaches can be left to the handwriting recognition section and the 
%     literature, however - I needed to come up with a 'good mix' of 
%     recognition quality and the possibility to extract detailed knowledge 
%     from the recognitiont process.
     
%     what made it the best choice? 
%     certainly not the lacking of alternatives, but the mix between 
%     possibilities to programmatically interact with the recognition process
%     and performance of the process.
%     see - main section! not too much repetition.

%     was it the optimal solution?

%     given the circumstances - yes, because it fulfilled the criteria 
%     (to be layed out)
%     from a pure HWR perspective - no, sinces other systems perform better on the
%     pure handwriting task.
%     from a hwr in a learning environment perspective - yes, because this
%     system gives us the ability to accompany the recognition process in detail
    

% - How will this section be structured and organised?
%   The organisational structure of the section will contain

%   1. from the captured point lists to something cooler, what are we actually
%      doing? from point to 'stroke'.

%   2. 'normalisation', including 'boxing' and then 'scaling'

%   3. curve handling. what are we doing at the edgy points?

%   4. alternative solution: time warping. doesn't need anything of the
%      abovementioned cool stuff, just one alorithm.
%      show algorithm and give a rough, rough, rough explanation,
%      maybe even without algorithm and point to papers that explain DTW.

% - In what style will it be written?
%   The style of writing will be technical and mathematical.
%   if something is calculated, show maths behind rather than algorithm.

% - Next action - what to write first?
%   The next part to write is the detailed content requirements of each subsection.

% xxx: see santosh2009 for mathematical stuff: nice description of what I'm doing

\subsection{Advanced Point Lists}
\label{sec:hwre:advancedpointlists}
%xxx this chapter used to be called:
%xxx \subsection{From Simple Point List to Boxed Representation}

what happens to the points?
nothing, really - the magic happens when normalisation and the other stuff
starts.
why this section? what's the purpose? oh, right - angles and vectors instead
of simple points. from one point to the next, or rather from on point to
ten points down the line, to get a rougher direction.
vectors make it interesting. 
impacts on curve handling! gradient and stuff can be measured in the 
vector representation (even without any boxes)
making the point list a cool mathematical object!
show code samples in pseudocode if necessary.
report about the cool stuff.

what's the similarity measure for
points and strokes?
show requirements.
what alternatives were there to consider?

\subsection{Normalisation}
\label{sec:hwre:normalisation}

what is N?
why do N?
show requirements.
how is N performed here?
why is it performed like that?

\subsubsection{Boxing}
\label{sec:hwre:boxing}
how is boxing done?
show requirements.
what alternatives were there to consider?
is it useful to have a similarity measure for bounding boxes?
yes! but why? explain!
size of the boxes! - think of characters that only have two strokes.

\subsubsection{Scaling}
\label{sec:hwre:scaling}

s. 42-45
how is scaling done?
show requirements.
what alternatives were there to consider?

\subsection{Curve Handling}
\label{sec:hwre:curvehandling}

S 14, 16, 17
how is curver handling done?
show requirements.
what alternatives were there to consider?

stroke matching with angles instead of point position.
s. 24

\subsection{Dynamic Time Warping}
\label{sec:hwre:dynamictimewarping}

what's the similarity measure for
points and strokes?
show requirements.
what alternatives were there to consider?

s. 51
how is dynamic time warping done here?
pointer to papers or hwr - chapter, don't explain DTW here.
show requirements
why DTW?
what alternatives were there to consider?
none - it is the alternative.
to all the other stuff I've been doing.
however, what about 3D time warping?


\section{Radical Recognition Process}
\label{sec:hwre:radicalrecognitionprocess}

% - Why this section? 
%   The purpose of this section is to explain the Radical recognition process in
%   detail. It is a crucial part of the structural recognition process.
%   Therefore the purpose is evident - see purpose of chapter.

%   It would be off purpose, if there'd be too much around,
%   focus on the RRP only, nothing around.

% - What goes into this section?
%   The main content of this section is the radical recognition.
%   That contains how the model is matched to the input.
%   Incrementality of recognition, using one stroke, two strokes,
%   three strokes and so on...

%   * if describing a problem: why is the problem relevant.
%     it is a part of the structural recognition process!

%   * if describing a solution to a problem: what alternatives were
%     there to solve it, why was this solution chosen? 
%     again - part of the structural recognition process that allows for
%     interaction with the user concerning error recognition

%     what made it the best choice? was it the optimal solution?
%     here, simply the lacking of alternatives. if the recognition process
%     does not distinguish between 'shapes' and 'radicals' the error
%     handling engine would not be able to inform the user.
%     (xxx: turn this one around into a positive sentence. 
%     only through distinguishing... :xxx)
    
% - How will this section be structured and organised?
%   The organisational structure of the section will be as follows:

%     1. Representation of a radical as a stroke sequence
%     2. stroke matching, reordered stroke matching in order to find 
%        alternative radicals or radical variants, depending on the 
%        number of strokes involved in the process: stroke number, stroke sequence.

% - In what style will it be written?
%   The style of writing will be formal, showing the general algorithm.

% - Next action - what to write first?
%   The next part to write is the general algorithm as pseudocode.

% what's the similarity measure for
% radicals?
% show requirements.
% what alternatives were there to consider?

% what about stroke number and stroke sequence?
% why deal with it? can be dealt with? 
% how? why this way, why not another way?
% generally, how is radical recognition done?
% show requirements.
% what alternatives were there to consider?

\section{Character Recognition Process}
\label{sec:hwre:characterrecognitionprocess}

% - Why this section? 
%   The purpose of this section is to distinguish the radical recognition process
%   from the character recognition process.
  
%   It would be off purpose, if too much of the radical recognition process would
%   be repeaeted. 

% - What goes into this section?
%   The main content of this section is to show the actual character recognition
%   process based on the radical recognition. 
%   it should show the plan to find the appropriate characters, given a number of
%   alternative radicals.
%   s. 9/10 pixelwolke vs. reihenfolge

%   * if describing a problem: why is the problem relevant.
%     the problem is not just relevant it is the final problem to solve
%     before solving the main problem of HWR.
%   * if describing a solution to a problem: what alternatives were
%     there to solve it, 
%     many different alternatives - see discussion of HWR, however,
%     the approach chosen here was the structural one/
%     why was this solution chosen? 
%     because of the possibility to interact with the recognition process
%     and reading partial results.
%     what made it the best choice? was it the optimal solution?
%     the other HWR approaches do not allow for the partial structural recognition
%     with linguistic components of characters.

% - How will this section be structured and organised?
%   The organisational structure of the section will contain
%   1. description of the process from  radical list to a character.
%   2. what if there are alternative radical lists?
%   3. what would've been an alternative solution?
%      directly going from exactly one radical list to exactly one chararcter.
%      s. 9/10 pixelwolke vs. reihenfolge

% - In what style will it be written?
%   The style of writing will be technical

% - Next action - what to write first?
%   The next part to write is to put the subparts into proper subsections.

% what's the similarity measure for
% characters?
% show requirements.
% s. 24 pseudocode

\section{Error Handling}
\label{sec:hwre:errorhandling}

% - Why this section? 
%   The purpose of this section is to demonstrate a crucial novelty of this 
%   HWR engline, the error handling module.
  
%   It would be off purpose, if it had to explain too much of the recognition 
%   process as such - that should've been done in other chapters.
  
% - What goes into this section?
%   The main content of this section is the 'way of an error'.
%   what happens if the character that is bound to be recognised just doesn't fit?
%   so there's an stroke that's not long enough and it's position is not
%   high enough within the character.

%   * if describing a problem: why is the problem relevant.
%     the problem is relevant for a learning application.
%     the purpose of the application is to teach the characters with an automated
%     recognition engine - one the central ''A.I.'' of this application
%     is the automated error recognition.

%   * if describing a solution to a problem: 
%     what alternatives were there to solve it, 
    
%     the alternative to the chosen process was a less flexible approach.
%     just measuring the absolute match - in size, shape pixel by pixel
%     (again pixelwolke) would've been of a lower quality.
%     compare skritter - it does not provide the high quality feedback.

%     why was this solution chosen? 
%     because the design of the application required high-quality feedback,
%     on a linguistic level.
    
%     what made it the best choice? 
%     because it was the only one that met the requirement of providing 
%     high-qualitiy feed-back.
    
%     was it the optimal solution?
%     who knows? - looks like it now, but evaluation will show.
%     depends on user experience. 

% - How will this section be structured and organised?
%   The organisational structure of the section 
%    1. error recognition
%    2. The Way of an error after recognition (error processing)
%    3. 
% - In what style will it be written?
%   The style of writing will be a technical description - probably no algorithms

% - Next action - what to write first?
%   The next part to write is to describe the error recognition.

see section \ref{sec:concept:sourcesoferror} in chapter 
\ref{chap:conceptualdesignofkanjicoach} for possible sources of error

\subsection{Error Recognition}
\label{sec:hwre:errorrecognition}

why this section? to demonstrate own achievements of error recognition.
the reader should know how it is done technically.

what goes into this section? the aspects of finding errors. finding errors
is not a straightforward trivial task - whenever something does not match
it is an error - doesn't work like that. instead, 
firstly, it needs to be made sure that it actually is an error.
meaning - not a recognition error, but a user error.
secondly, the type of error needs be identified.
see section \ref{sec:concept:sourcesoferror} (or handwritten page 58)
for sources of error.

how will this section be written?
technical - first describe how the error recognition integrates into the
recognition process, then how errors are identified.


\subsection{Error Processing}
\label{sec:hwre:errorprocessing}

%focus on technical aspects

why this section? 
actually the 'handling' or 'processing' aspect could be 
described in the recognition section \ref{sec:hwre:errorrecognition} as well.
so this section is only for a better overview, for document structure, 
thematically they are the same section. thus they are put together under
Error Handling \ref{sec:hwre:errorhandling}.

what goes into this section?

