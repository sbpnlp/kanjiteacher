%%% Local Variables: 
%%% mode: latex
%%% TeX-master: "../KanjiHWR"
%%% End: 
\chapter{Conclusions}
\label{chap:conclusions}

%- Why this section? 
%  The purpose of this section is 
%  It would be off purpose, if 
%- What goes into this section?
%  The main content of this section is 
%  * if describing a problem: why is the problem relevant.
%  * if describing a solution to a problem: what alternatives were
%    there to solve it, why was this solution chosen? 
%    what made it the best choice? was it the optimal solution?
%- How will this section be structured and organised?
%  The organisational structure of the section 
%- In what style will it be written?
%  The style of writing will be 
%- Next action - what to write first?
%  The next part to write is


%siehe s. 47 oben.
%ich habe erschaffen.
%arbeitshypothese - lernen - kann kaum evaluiert werden.

%siehe introduction:
%was ist neu, was ist geil?
%vergleiche mit motivation?
%inwieweit ist motivation erfuellt?
%warum haben wir dass gemacht? was ist das ergebnis?

\section{Recapitulation}
\label{sec:conclusion:recapitulation}

The goal of this work was the creation of an \emph{analytical} handwriting
recognition engine for Kanji. A subordinated goal was to study the new 
possibilities such an system component offers. This has been exemplified for
a sample e-learning application that includes excercises for handwritten input.
Only because of the analytical handwriting recognition, the e-learning 
application can give the learner feedback about the entered character.

This thesis contains a number of topics that do not have a common academic 
background. The prerequsites for building an analytical handwriting recognition
and integrate it into an e-learning environment form a common ground for the 
topics. The thesis started with the basics of Japanese handwriting. The Japanese
writing system with its three scripts is fairly different from alphabetic 
writing systems. The structure of the Japanese character system has been 
introduced in chapter~\ref{chap:japanasescript}.
Chapter~\ref{chap:onlinehwr} contains a literature review of the topic 
\emph{handwriting recognition systems}. The general techniques and models
for handwriting recognition are presented as well as the special techniques
for online character recognition and the specifics of Japanese and Chinese 
character recognition.
Seemingly unrelated to the previous chapters, but required for building an
e-learning application, chapter~\ref{chap:elearning} presents the general
methods an techniques for e-learning applications, especially e-learning
of languages. These chapters build the foundation of the following chapters.

The following two chapters are concerned with the design of the prototype
that has been developed during the course of the thesis.
Chapter~\ref{chap:conceptualdesignofKanjicoach} takes an abstract viewpoint 
and uses the conceptual insight of the foundational chapters for the 
design of the application. Chapter~\ref{chap:technicaldesign} presents the 
technical design choices that implement the concepts developed in the
precedent chapter.

The core of the recognition engine is layed out in 
chapter~\ref{chap:handwritingrecognitionengine}. This chapter contains topics
that are part of the technical design. However, the viewpoint in this chapter
is more detailed and magnifies a specific part of the application.
The evaluation of the analytical handwriting recognition has been
presented in chapter~\ref{chap:implementationevaluation} and resulted in 
recognition rates below the ones of non-analytical systems. 

\section{Discussion}
\label{sec:conclusion:discussion}


That seems due to
the fact that the detailed analysis increases the recognition options.

The integration of NLP and e-learning relies on the analytical handwriting
recognition. Since characters are split into sub units, the generated feedback 
about the input correctness can guide a user of the e-learning application 
precisely to the error. This is an advantage over a binary statement 
\emph{correct} or \emph{incorrect} that is directly based on recognition success.



