%%% Local Variables: 
%%% mode: latex
%%% TeX-master: "../KanjiHWR"
%%% End: 
\chapter{Conclusions}
\label{chap:conclusions}

%- Why this section? 
%  The purpose of this section is 
%  It would be off purpose, if 
%- What goes into this section?
%  The main content of this section is 
%  * if describing a problem: why is the problem relevant.
%  * if describing a solution to a problem: what alternatives were
%    there to solve it, why was this solution chosen? 
%    what made it the best choice? was it the optimal solution?
%- How will this section be structured and organised?
%  The organisational structure of the section 
%- In what style will it be written?
%  The style of writing will be 
%- Next action - what to write first?
%  The next part to write is


%siehe s. 47 oben.
%ich habe erschaffen.
%arbeitshypothese - lernen - kann kaum evaluiert werden.

%siehe introduction:
%was ist neu, was ist geil?
%vergleiche mit motivation?
%inwieweit ist motivation erfuellt?
%warum haben wir dass gemacht? was ist das ergebnis?

\section{Recapitulation}
\label{sec:conclusion:recapitulation}

The goal of this work was the development of an \emph{analytical} handwriting
recognition engine for Kanji. A subordinated aim was to study the new 
possibilities such a system component offers. This has been exemplified for
a sample e-learning application that includes excercises for handwritten input.
The analytical handwriting recognition provides the e-learning 
application with the ability to give the learner feedback about an 
entered character.

This thesis contains a number of topics that do not share a common academic 
background. The prerequisites for building an analytical handwriting recognition
and integrate it into an e-learning environment form a common ground for the 
topics. The thesis starts with the basics of Japanese handwriting. The Japanese
writing system with its three scripts is fairly different from alphabetic 
writing systems. The structure of the Japanese character system has been 
introduced in chapter~\ref{chap:japanasescript}.
Chapter~\ref{chap:onlinehwr} contains a literature review of the topic 
\emph{handwriting recognition systems}. The general techniques and models
for handwriting recognition are presented as well as the special techniques
for online character recognition and the specifics of Japanese and Chinese 
character recognition.
Seemingly unrelated to the previous chapters, but required for building an
e-learning application, chapter~\ref{chap:elearning} presents the general
methods and techniques for implementing e-learning applications, especially 
for e-learning of languages. This first part of the thesis builds the 
foundation of the following chapters.

The next two chapters are concerned with the design of the prototype
that has been developed during the course of the thesis. The findings 
of these chapters are directly geared to the goals of the thesis.
Chapter~\ref{chap:conceptualdesignofKanjicoach} takes an abstract viewpoint 
and incorporate the conceptual insight of the foundational chapters for the 
high-level design of the application. The knowledge about the 
Japanese character structure and the educational methodology of e-learning
are used for the conceptual design.
Chapter~\ref{chap:technicaldesign} presents the 
technical design choices that implement the concepts developed in the
precedent chapter. 

The core of the recognition engine is layed out in 
chapter~\ref{chap:handwritingrecognitionengine}. This chapter contains topics
that are part of the technical design. However, the viewpoint in this chapter
is more detailed and magnifies a specific part of the application. The core
recognition engine uses a combination of different pattern matching techniques
in order to perform an analytical character recognition, including error
recognition.

The evaluation of the analytical handwriting recognition has been
presented in chapter~\ref{chap:implementationevaluation}. The recognition rates
were below the rates of non-analytical systems, but the systems provides 
additional analytical information that can be used in different contexts.

\section{Discussion}
\label{sec:conclusion:discussion}

In this work an analytical handwriting recognition engine has been developed.
The motivation behind that was not just to develop another recognition system
for Japanese characters, but to create a system that is oriented towards
weak artificial intelligence and analysis the structures to a certain degree
of \emph{understanding}. Another motivational aspect was to look into a
different type of user interface for e-learning applications.

A direct comparison to other efforts in the field of Japanese handwriting 
recognition is not entirely possible because of the additional analytical steps
performed by the prototype.
The sheer recognition rates are lower than the recognition rates of systems
that perform a pure and optimised handwriting recognition.
That seems due to the fact that the detailed analysis increases the 
recognition options.
Partial character recognition of substructures of Japanese characters
has similar recognition results compared to state-of-the-art recognition 
systems.

The attempt to integrate NLP and e-learning was successful. The central part of
the e-learning system relies on the analytical handwriting recognition. 
When a user enteres a character, the character is split into sub units, 
the generated feedback about the input correctness can guide a user precisely 
to the error. This feature is advatagous compared to the binary statement 
\emph{correct} or \emph{incorrect} that other recognition systems provide.
The success of the learning component could not be evaluated because that would
have been outside the scope of the thesis. However, it is plausible that writing
Kanji and receiving feedback is an appropriate method to learn writing
Kanji. A direct comparison to other e-learning systems was not possible as this
is the only e-learning system know to the author that provides feedback to
handwritten input. 

