%%% Local Variables: 
%%% mode: latex
%%% TeX-master: "../KanjiHWR"
%%% End: 

\title{An On-Line Japanese Handwriting Recognition System integrated into an E-Learning Environment for Kanji}
\author{Steven B. Poggel\\
steven.poggel@gmail.com}
% \thanks{danksagung }

%\begin{titlepage}
%\begin{center}
%% Upper part of the page
%%\includegraphics[width=0.15\textwidth]{./logo}\\[1cm]
%\textsc{\LARGE University of Beer}\\[1.5cm]
%\textsc{\Large Final year project}\\[0.5cm]
%% Title
%\HRule \\[0.4cm]
%{ \huge \bfseries Lager brewing techniques}\\[0.4cm]
%\HRule \\[1.5cm]
%% Author and supervisor
%\begin{minipage}{0.4\textwidth}
%\begin{flushleft} \large
%\emph{Author:}\\
%John \textsc{Smith}
%\end{flushleft}
%\end{minipage}
%\begin{minipage}{0.4\textwidth}
%\begin{flushright} \large
%\emph{Supervisor:} \\
%Dr. Mark \textsc{Brown}
%\end{flushright}
%\end{minipage}
%\vfill 
%% Bottom of the page
%{\large \today}
%\end{center}
%\end{titlepage}

%%
% Deckblatt
%


%xxx - Why this section? 
%xxx   The purpose of this section is 
%xxx   It would be off purpose, if 
%xxx - What goes into this section?
%xxx   The main content of this section is 
%xxx   * if describing a problem: why is the problem relevant.
%xxx   * if describing a solution to a problem: what alternatives were
%xxx     there to solve it, why was this solution chosen? 
%xxx     what made it the best choice? was it the optimal solution?
%xxx - How will this section be structured and organised?
%xxx   The organisational structure of the section 
%xxx - In what style will it be written?
%xxx   The style of writing will be 
%xxx - Next action - what to write first?
%xxx   The next part to write is

\maketitle


%\begin{center}
%\textbf{DFKI} \\
%IUI \\
%Prof. Wolfgang Wahlster \\
%Department of Computational Linguistics\\
%Saarland University
%\end{center}

%%%xxx an empty page, because the TOC shouldn't begin at a backside of a page
%\thispagestyle{empty}
%\newpage
%\textsc{ }
%\thispagestyle{empty}
%\newpage


%%
% Inhalt
%

%%
% Table of Contents
%

%\thispagestyle{empty}
\pagestyle{empty}%myheadings}

%\markboth{linker kopf}{rechter kopf}
%\markright{Kochbuch für \LaTeX\hfill\today\hfill}

\tableofcontents



%%
%Another empty page
%

%%%xxx an empty page, because the TOC shouldn't begin at a backside of a page
%\thispagestyle{empty}
%\textsc{ }
%\newpage


%%
%Another empty page
%
%%%xxx an empty page, because the TOC shouldn't begin at a backside of a page
%\thispagestyle{empty}
%\newpage

%%
% Dokumentinhalt, Text
%



