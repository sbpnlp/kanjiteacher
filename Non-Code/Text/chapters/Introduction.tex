%%% Local Variables: 
%%% mode: latex
%%% TeX-master: "../KanjiHWR"
%%% End: 

\chapter{Introduction}
\label{chap:introduction}
\pagestyle{headings}

\section{Motivation}
\label{sec:intro:motivation}

%xxx wiss. fragestellung
%wie sieht hwr in lernumgebung aus?
% requirements s. 19

%xxx - Why this section? 
%xxx   The purpose of this section is 
%xxx   It would be off purpose, if 
%xxx - What goes into this section?
%xxx   The main content of this section is 
%xxx   * if describing a problem: why is the problem relevant.
%xxx   * if describing a solution to a problem: what alternatives were
%xxx     there to solve it, why was this solution chosen? 
%xxx     what made it the best choice? was it the optimal solution?
%xxx - How will this section be structured and organised?
%xxx   The organisational structure of the section 
%xxx - In what style will it be written?
%xxx   The style of writing will be 
%xxx - Next action - what to write first?
%xxx   The next part to write is

%xxx motivation: ausprobieren, welche moeglichkeiten sich aus handschriften-
%erkennung ergeben. S. 15 kernfragen herausarbeiten und warum diese relevant sind
% dafuer notwendig eine abbildbare modellierung, die auf papier nicht existiert.
%ist nicht dasselbe wie introduction, die in das thema einfuehrt.

%was ist daran neu? was ist daran cool? warum wollen wir das machen?
%bringt das was?

In the history of Computational Linguistics there have been a several 
attempts to integrate natural language processing techniques with 
existing technologies. This work is one just like that.
Concretely, we will try to create a handwriting recognition for Japanese Kanji. That seems interesting, because Kanji is an iconographic writing system, thus handwriting recognition (HWR) can follow different patterns than in alphabetical writing systems like latin.

Studying Japanese language is a complex task, because a new learner has to get used to a new vocabulary that - coming from a European language - has very little in common with the vocabulary of his mother tongue, unlike in European languages where quite often there are several intersections. The learner also needs to learn a new grammar system. Broadly speaking, most of the central European languages follow a subject-verb-object (SVO) structure. Japanese follows a subject-object-verb (SOV) structure therefore creating additional difficulty, comparable with German subclause structures that are a source of error for learners of German. 
Yet, the most notable difference for a language learner with a central European mother tongue is of course the writing system. The Japanese writing system uses three different scripts. The Kana scripts Hiragana and Katakana are syllabic, each character represents a syllable. Each syllable consists of either a vowel, a consonant and a vowel, or a consonant cluster and a vowel. Hiragana and Katakana represent roughly the same set of syllables and both have around 40-50 characters that can be modified with diacritics and thus yield additional syllable representations. Therefore, these scripts are a hurdle, but relatively unproblematic, due to their limitation in number of characters. Besides, they look quite distinct, so there is the problem of confusing one character with another, but this is limited to a relatively short period of learning those characters.

Kanji, however, is an iconographic writing system that has around 2000 characters, which are built up of around 200 subunits called 'radicals'. So one part of the complexity lies in the number of characters. The other part of the complexity lies in the general concept of representing an idea or concept with a character instead of representing the phonemes of the spoken language with graphemes in connection with some language specific pronunciation rules. Another difficulty lies in connecting the characters with their pronunciations. Most characters have multiple pronunciations and for a language learner, studying Japanese vocabulary is a double or triple task compared to languages using a Latin or at least an alphabetic writing system. Therefore, the two tasks of learning the Kanji and studying the vocabulary together can epitomise a very high learning curve. A subordinated issue connected to that is that quite often subjectively 'simple' vocabulary comes with complex Kanji.
Some e-learning applications have taken on that issue by creating a learning environment in which a learner can connect learning vocabulary with studying the Kanji.

xxx: see santosh2009: the statement of need

\subsection{Integrating NLP and e-learning}
In this project, we would like to approach the issue of studying Kanji in an e-learning application. The novelty about it is a handwriting recognition that gives the learner the ability to actually practise writing the Kanji, instead of the rather limited multiple choice recognition that most other applications use.


\subsection{Another subsection with a yet unknown title}

\section{A CJK environment}

Rather than selecting a CJK font as the main document typeface, you might want to define a CJK environment for text fragments used in the midst of a document using a normal Roman font. This allows me to say 
\verb|\begin{CJK}|\cjk{東光}\verb|\end{CJK}| to generate \cjk{東光}, without putting the whole paragraph into the Far Eastern font. Or I could define a command that takes the CJK text as an argument, so that \verb|\cjk{|\cjk{北京}\verb|}| produces \cjk{北京}. It's that easy!

\section{Running text}

%\setromanfont{MS Mincho}
%\XeTeXlinebreaklocale "ja" % actually, even "en" will work for default CJK breaking
%\XeTeXlinebreakskip = 0pt plus 1pt % allow slight stretchability for justification
%\XeTeXlinebreakpenalty = 100 % slight preference for breaks at spaces if available
\begin{CJK}
コンピューターは、本質的には数字しか扱うことができません。コンピューターは、文字や記号などのそれぞれに番号を割り振ることによって扱えるようにします。ユニコードが出来るまでは、これらの番号を割り振る仕組みが何百種類も存在しました。どの一つをとっても、十分な文字を含んではいませんでした。例えば、欧州連合一つを見ても、そのすべての言語をカバーするためには、いくつかの異なる符号化の仕組みが必要でした。英語のような一つの言語に限っても、一つだけの符号化の仕組みでは、一般的に使われるすべての文字、句読点、技術的な記号などを扱うには不十分でした。


これらの符号化の仕組みは、相互に矛盾するものでもありました。二つの異なる符号化の仕組みが、二つの異なる文字に同一の番号を付けることもできるし、同じ文字に異なる番号を付けることもできるのです。どのようなコンピューターも(特にサーバーは)多くの異なった符号化の仕組みをサポートする必要があります。たとえデータが異なる符号化の仕組みやプラットフォームを通過しても、いつどこでデータが乱れるか分からない危険を冒すことのなるのです。

\end{CJK}