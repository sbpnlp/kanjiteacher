%%% Local Variables: 
%%% mode: latex
%%% TeX-master: "../KanjiHWR"
%%% End: 

\chapter{Introduction}
\label{chap:introduction}
\pagestyle{headings} %this is a general setting for the document, 
                     %except the unnumbered chapters. That's why it's here.

\section{Motivation}
\label{sec:intro:motivation}

%xxx wiss. fragestellung
%wie sieht hwr in lernumgebung aus?
% requirements s. 19

%xxx - Why this section? 
%xxx   The purpose of this section is 
%xxx   It would be off purpose, if 
%xxx - What goes into this section?
%xxx   The main content of this section is 
%xxx   * if describing a problem: why is the problem relevant.
%xxx   * if describing a solution to a problem: what alternatives were
%xxx     there to solve it, why was this solution chosen? 
%xxx     what made it the best choice? was it the optimal solution?
%xxx - How will this section be structured and organised?
%xxx   The organisational structure of the section 
%xxx - In what style will it be written?
%xxx   The style of writing will be 
%xxx - Next action - what to write first?
%xxx   The next part to write is

%xxx motivation: ausprobieren, welche moeglichkeiten sich aus handschriften-
%erkennung ergeben. S. 15 kernfragen herausarbeiten und warum diese relevant sind
% dafuer notwendig eine abbildbare modellierung, die auf papier nicht existiert.
%ist nicht dasselbe wie introduction, die in das thema einfuehrt.

%was ist daran neu? was ist daran cool? warum wollen wir das machen?
%bringt das was?

In the history of Computational Linguistics there have been a several 
attempts to integrate natural language processing techniques with 
existing technologies. That task is complex in many aspects, depending on what is
aimed at exactly.

In this study, I will attempt to create a handwriting recognition for Japanese 
Kanji. That seems interesting, because Kanji is an morphemic writing system
with a large number of characters. Thus, handwriting recognition (HWR) follows 
different patterns than in alphabetical writing systems like Latin script.
The overall methodology of HWR system is similar to systems for Latin characters,
but the details of analysis vary strongly.

Studying Japanese language is a complex task, because a new learner has to get 
used to a new vocabulary that - coming from a European language - has very 
little in common with the vocabulary of his mother tongue, unlike in European 
languages where quite often there are several intersections. The learner also 
needs to learn a new grammar system. Broadly speaking, most of the central 
European languages follow a subject-verb-object (SVO) structure. 
Japanese follows a subject-object-verb (SOV) structure therefore creating 
additional difficulty, comparable with German reversed subclause structures that 
are a source of error for many learners of German. 
Yet, the most notable difference for a language learner with a central European 
mother tongue is of course the writing system. The Japanese writing system uses 
three different scripts. The so-called \emph{Kana} scripts \emph{Hiragana} and 
\emph{Katakana} are syllabic, each character represents a syllable. Each syllable
consists of either a vowel, a consonant and a vowel, or a consonant cluster and 
a vowel. Hiragana and Katakana represent roughly the same inventory of syllables
and both have around 40-50 characters that can be modified with diacritics in 
order to yield additional syllable representations. Therefore, these scripts are 
a hurdle, but relatively unproblematic, due to their limited number of 
characters. Besides, the two sets of \emph{Kana} characters look quite distinct, 
so the problem of confusing one character with another is limited to a relatively
short learning period of those characters.

The Kanji, on the contrary, form the largest part of a writing system that has 
around 3,000 characters, which are built up of around 200 subunits called 
\emph{Radicals}. 
One part of the complexity lies in the number of characters. 
The other part of the complexity lies in the general concept of representing an 
idea or concept with a character instead of representing the phonemes of the 
spoken language with graphemes in connection with some language specific 
pronunciation rules. 
Another difficulty lies in connecting the characters with their pronunciations. 
Most characters have multiple pronunciations and for a language learner, 
studying Japanese vocabulary takes at least twice as much effort compared to 
languages using a Latin or at least some kind of alphabetic writing system. 
The two tasks of learning the Kanji and studying the vocabulary together can 
epitomise a very high learning curve. A connected subordinated problem lies
in the fact that quite often subjectively 'simple' vocabulary comes with 
complex Kanji. Some e-learning applications have taken on that issue by creating 
a learning environment in which a learner can connect learning vocabulary with 
studying the Kanji characters.

%xxx: see santosh2009: the statement of need

\subsection{Integrating NLP and E-Learning}
In this project, we would like to approach the issue of studying Kanji in an e-learning application. The novelty about it is a handwriting recognition that gives the learner the ability to actually practise writing the Kanji, instead of the rather limited multiple choice recognition that most other applications use.

