%%% Local Variables: 
%%% mode: latex
%%% TeX-master: "../KanjiHWR"
%%% End: 

\chapter{Introduction}
\label{chap:introduction}
\pagestyle{headings} %this is a general setting for the document, 
                     %except the unnumbered chapters. That's why it's here.

\section{Motivation}
\label{sec:intro:motivation}

%xxx wiss. fragestellung
%wie sieht hwr in lernumgebung aus?
% requirements s. 19

%xxx - Why this section? 
%xxx   The purpose of this section is 
%xxx   It would be off purpose, if 
%xxx - What goes into this section?
%xxx   The main content of this section is 
%xxx   * if describing a problem: why is the problem relevant.
%xxx   * if describing a solution to a problem: what alternatives were
%xxx     there to solve it, why was this solution chosen? 
%xxx     what made it the best choice? was it the optimal solution?
%xxx - How will this section be structured and organised?
%xxx   The organisational structure of the section 
%xxx - In what style will it be written?
%xxx   The style of writing will be 
%xxx - Next action - what to write first?
%xxx   The next part to write is

%xxx motivation: ausprobieren, welche moeglichkeiten sich aus handschriften-
%erkennung ergeben. S. 15 kernfragen herausarbeiten und warum diese relevant sind
% dafuer notwendig eine abbildbare modellierung, die auf papier nicht existiert.
%ist nicht dasselbe wie introduction, die in das thema einfuehrt.

%was ist daran neu? was ist daran cool? warum wollen wir das machen?
%bringt das was?

In the history of Computational Linguistics there have been a several 
attempts to integrate natural language processing techniques with 
existing technologies. That task is complex in many aspects, depending on what is
aimed at exactly.

In this study, a handwriting recognition for Japanese Kanji will be created. 
That is interesting because Kanji the characters form a morphemic 
writing system with a large number of characters. Thus, handwriting 
recognition (HWR) of Kanji follows different patterns than in alphabetical 
writing systems like Latin script. The overall methodology of HWR system is 
similar to systems for Latin characters, but the details of analysis vary 
strongly.

Studying Japanese language is a complex task, because a new learner has to get 
used to a new vocabulary that - coming from a European language - has very 
little in common with the vocabulary of his mother tongue, unlike in European 
languages where quite often there are several intersections. The learner also 
needs to learn a new grammar system. Broadly speaking, most of the central 
European languages follow a subject-verb-object (SVO) structure. 
Japanese follows a subject-object-verb (SOV) structure. These create additional 
difficulty, comparable with German reversed subclause structures that 
are a source of error for many learners of German. 
Yet, the most notable difference for a language learner with a central European 
mother tongue is of course the writing system. The Japanese writing system uses 
three different scripts. The so-called \emph{Kana} scripts \emph{Hiragana} and 
\emph{Katakana} are syllabic, each character represents a syllable. Each syllable
consists of either a vowel, a consonant and a vowel, or a consonant cluster and 
a vowel. The syllables are called \emph{open}. Hiragana and Katakana represent 
roughly the same inventory of syllables and both have around 40-50 characters 
that can be modified with diacritics in order to yield additional syllable 
representations. Therefore, these scripts are a hurdle for a learner, but 
relatively unproblematic, due to their limited number of characters. 
Besides, the two sets of Kana characters look quite distinct, 
so the problem of confusing one character with another is limited to a relatively
short learning period of those two scripts.

The Kanji, on the contrary, form the largest part of a writing system that has 
around 3,000 characters, which are built up of around 200 subunits called 
\emph{radicals}. 
One part of the complexity lies in the number of characters. 
The other part lies in the general method of representing an 
idea or concept with a character instead of attempting to represent the sounds
with graphemes in connection with language specific pronunciation rules. 
Another difficulty lies in cognitively connecting the characters with their 
pronunciations. Most characters have multiple pronunciations and for a language 
learner, studying Japanese vocabulary takes at least twice as much effort 
compared to languages using a Latin or at least some kind of alphabetic writing 
system. The two tasks of learning the Kanji and studying the vocabulary 
together can epitomise a very high learning curve. A connected subordinated 
problem lies in the fact that quite often subjectively 'simple' vocabulary comes 
with complex Kanji. Some e-learning applications have taken on that issue by 
creating a learning environment in which a learner can connect learning 
vocabulary with studying the Kanji characters.

%xxx: see santosh2009: the statement of need

\subsection{Integrating NLP and E-Learning}
\label{sec:intro:integratingnlpandelearning}

In this project, the main focus lies on the creation of an analytical handwriting
recognition engine for Kanji. The advantages of such an engine can be applied to
e-learning applications. This is exemplified with a sample e-learning application
that includes specialised excercises for handwritten input.
The novelty about it is a handwriting recognition that gives the learner the 
ability to actually practise writing the Kanji and get a feedback, instead of 
the rather limited multiple choice recognition that most other applications use.

The integration of NLP and e-learning relies on the analytical handwriting
recognition. Since characters are split into sub units, the generated feedback 
about the input correctness can guide a user of the e-learning application 
precisely to the error. This is an advantage over a binary statement 
\emph{correct} or \emph{incorrect} that is directly based on recognition success.

The analytical handwriting recognition examines the individual parts of the
input and compares them to the individual parts of the database entries.
That way, the errors can be identified locally in respect to the full character.
The use of this is the improvement of the learning experience. The error messages
by the system have a one-to-one correspondence with the character 
structure that needs to be studied.