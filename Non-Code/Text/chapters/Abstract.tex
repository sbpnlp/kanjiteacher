%%% Local Variables: 
%%% mode: latex
%%% TeX-master: "../KanjiHWR"
%%% End: 

\chapter*{Summary}
\label{chap:summary}
\addcontentsline{toc}{chapter}{Summary}
\pagestyle{empty}

%\markboth{linker kopf2}{rechter kopf2}

% - Why this section? 
%   The purpose of this section is to give a short summary of the whole thesis.
  
%   It would be off purpose, if it started explaining any details. It is just an
%   executive summary.

% - What goes into this section?
%   The main content of this section is an executive summary of the main bits
%   and piecese of the thesis.

%   * if describing a problem: why is the problem relevant.
%     The problem is - kanji are difficult to learn.
%     repeating to write them helps. 
%     handwriting recognition is not a new technology
%     combining the two yields an machine teacher - an e-learning environment.

%   * if describing a solution to a problem: what alternatives were 
%     there to solve it, why was this solution chosen? 
%     executive summary! leave alternatives aside.
%     most importantly: we have a solution!
%     and the solution works.
    
% - How will this section be structured and organised?
%   The organisational structure of the section will contain:

%   1. problem description
%   2. solution
%   3. yes, the solution works

% - In what style will it be written?
%   The style of writing will be consise. very consise.
%   around half a page
% - Next action - what to write first?
%   The next part to write is the abstract. there are some lines already,
%   apply structure 1,2,3 and finalise.

In this work I present and application that uses state of the art 
Chinese/Japanese handwriting recognition methods in order to provide 
an Kanji teaching application with an error correction.

Conceptually, the application is an e-learning environment for Japanese 
characters, intended for the foreign learner of the Japanese language. 
In order to provide more than a  multiple choice method, like most other 
systems, the application contains a handwriting recognition engine that can
be used preferably with a hand-held device like a PDA, but generally any 
stylus input device.

%\end{abstract}
%this is an example for an abstract of a research paper.
%look at other abstracts here:
%C:\Diplom.old\sampleDAs

  % abstract =     {In this paper, we give a comprehensive description
  %                 of our writer-independent online handwriting
  %                 recognition system frog on hand. The focus of this
  %                 work concerns the presentation of the
  %                 classification/training approach, which we call
  %                 cluster generative statistical dynamic time warping
  %                 (CSDTW). CSDTW is a general, scalable, HMM-based
  %                 method for variable-sized, sequential data that
  %                 holistically combines cluster analysis and
  %                 statistical sequence modeling. It can handle general
  %                 classification problems that rely on this sequential
  %                 type of data, e.g., speech recognition, genome
  %                 processing, robotics, etc. Contrary to previous
  %                 attempts, clustering and statistical sequence
  %                 modeling are embedded in a single feature space and
  %                 use a closely related distance measure. We show
  %                 character recognition experiments of frog on hand
  %                 using CSDTW on the UNIPEN online handwriting
  %                 database. The recognition accuracy is significantly
  %                 higher than reported results of other handwriting
  %                 recognition systems. Finally, we describe the
  %                 real-time implementation of frog on hand on a Linux
  %                 Compaq iPAQ embedded device.},


\chapter*{Zusammenfassung}
\label{chap:zusammenfassung}
\addcontentsline{toc}{chapter}{Zusammenfassung}
\pagestyle{empty}
\thispagestyle{empty}

In this work I present and application that uses state of the art 
Chinese/Japanese handwriting recognition methods in order to provide 
an Kanji teaching application with an error correction.

Conceptually, the application is an e-learning environment for Japanese 
characters, intended for the foreign learner of the Japanese language. 
In order to provide more than a  multiple choice method, like most other 
systems, the application contains a handwriting recognition engine that can
be used preferably with a hand-held device like a PDA, but generally any 
stylus input device.
