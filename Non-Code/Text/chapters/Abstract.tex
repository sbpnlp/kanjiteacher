%%% Local Variables: 
%%% mode: latex
%%% TeX-master: "../KanjiHWR"
%%% End: 

\chapter*{Summary}
\label{chap:summary}
\addcontentsline{toc}{chapter}{Summary}
\pagestyle{empty}

%\markboth{linker kopf2}{rechter kopf2}

% - Why this section? 
%   The purpose of this section is to give a short summary of the whole thesis.
  
%   It would be off purpose, if it started explaining any details. It is just an
%   executive summary.

% - What goes into this section?
%   The main content of this section is an executive summary of the main bits
%   and piecese of the thesis.

%   * if describing a problem: why is the problem relevant.
%     The problem is - Kanji are difficult to learn.
%     repeating to write them helps. 
%     handwriting recognition is not a new technology
%     combining the two yields an machine teacher - an e-learning environment.

%   * if describing a solution to a problem: what alternatives were 
%     there to solve it, why was this solution chosen? 
%     executive summary! leave alternatives aside.
%     most importantly: we have a solution!
%     and the solution works.
    
% - How will this section be structured and organised?
%   The organisational structure of the section will contain:

%   1. problem description
%   2. solution
%   3. yes, the solution works

% - In what style will it be written?
%   The style of writing will be consise. very consise.
%   around half a page
% - Next action - what to write first?
%   The next part to write is the abstract. there are some lines already,
%   apply structure 1,2,3 and finalise.

In this work, a Japanese handwriting recognition system is being developed.
The system is integrated into an e-learning environment in order to provide 
a Kanji teaching application with automated error correction.
Conceptually, the application is an e-learning environment for Japanese 
characters, intended for the western learner of the Japanese language. 
Most e-learning systems of Japanese Kanji provide only a multiple choice method
for the learner to reproduce characters. The present prototype offers the
ability to enter characters with a stylus on a touch screen system.

The study seeks to determine to what extent it is possible to use modern NLP 
methods for language learning. While other studies mainly focus on grammatical
correction, this application is targeted on the Kanji characters. It will be 
examined if a handwriting recognition engine can generate informed feedback,
suitable for a learner. Additionally, the study examines if that feedback helps 
obtaining the ability to actively reproduce the Kanji characters.

The prototype developed in this work combines e-learning methods with natural
language processing applied to the Japanese script. In order to prepare the 
task of creating an interdisciplinary software that spans across the aforementioned 
fields of study, the work reviews the structure of the Japanese script, 
the current state of the art in handwriting recognition methods and e-learning 
techniques.
The recognition engine implements a structural approach to Kanji character 
identification. The recognition performs partial analysis of substructures
and binds the recognised elements together to form a character.
Because of the structural approach it becomes possible to create an informed
error recognition that considers linguistic units of the Kanji characters.

The result of the study is

\chapter*{Zusammenfassung}
\label{chap:zusammenfassung}
\addcontentsline{toc}{chapter}{Zusammenfassung}

In dieser Arbeit wird eine Handschriftenerkennung für japanische Kanji 
entwickelt. Der Handschriftenerkenner ist in eine E-Learning-Umgebung integriert und
liefert eine automatisch generierte Fehlerkorrektur für Lernende.

Das System ist in konzeptioneller Hinsicht eine E-Learning-Anwendung für das 
Erlernen der japanischen Schrift. Letztere weist aufgrund ihrer morphemischen 
Struktur einen hohen Komplexitätsgrad auf und benötigt daher besonderen Lernaufwand.
Die meisten E-Learning-Systeme für asiatische Schriftzeichen bieten Zeichenabfrage
als Multiple-Choice an, da die Eingabe der Zeichen für einen Lernenden sonst 
ein technisches Problem darstellen würde.
Der in dieser Arbeit erstellte Prototyp bietet die Möglichkeit 
zur handschriftlichen Eingabe von Zeichen auf einer dafür geeigneten 
Bildschirmoberfläche. Das ist ein Alleinstellungsmerkmal unter den 
E-Learning-Anwendungen für die japanische Sprache.

Die Studie untersucht, inwieweit es im Bereich des Schrifterwerbs möglich ist, 
NLP und Lernmethoden zusammenzubringen. Dabei wird nicht mit Parsing-Methoden die 
grammatische Struktur der Sprache untersucht, sondern vielmehr die interne Struktur 
der Kanji zugrunde gelegt und durch einen Handschriftenerkenner erfasst. 
Dabei sollen Schreibfehler strukturell erkannt werden. Intelligentes Feedback soll 
dem Lernenden dabei helfen, die Fähigkeit der aktiven Reproduktion der Kanji zu 
erwerben.

Da der Prototyp eine disziplinübergreifende Software ist, 
die in den Bereichen Handschriftenerkennung und E-Learning angesiedelt ist,
wird in der vorliegenden Arbeit der Forschungsstand der beiden untersucht.
Weiterhin wird die Struktur der japanischen Schrift linguistisch analysiert und
dargestellt. Die Kombination der drei Disziplinen in einer Studie führt dazu, 
dass die Substrukturen der Kanji überhaupt programmatisch analysiert werden können,
wodurch die Fehlererkennung ermöglicht wird.


\chapter*{Danksagungen}
\label{chap:danksagungen}
\addcontentsline{toc}{chapter}{Danksagungen}

Ich möchte mich bei allen bedanken, die mir bei der Erstellung dieser Arbeit 
geholfen haben. Jeder hat einen wertvollen Beitrag geleistet, 
den ich sehr zu schätzen weiß. Besonders bedanken möchte ich mich bei 
Dr. Tilman Becker, der mir über die gesamte Zeit wertvolle Hinweise gegeben hat, mit dem ich 
hochinteressante fachliche Diskussionen hatte, die diese Arbeit vorangetrieben
und verbessert haben. Weiterhin bedanke ich mich bei Prof.\ Wahlster, in dessen 
Abteilung \emph{Intelligente Benutzerschnittstellen} am DFKI ich längere Zeit als HiWi 
und als Diplomand tätig war, für das Ermöglichen dieser Diplomstudie.
Mein besonderer Dank gilt allen Korrekturlesern, die mich vor dem ein oder anderen
Rechtschreibfehler und inhaltlichen Unstimmigkeiten bewahrt haben, sowie allen,
die mir mit Rat und Tat zur Seite gestanden haben. Diejenigen, 
die ich vergessen habe, mögen mir verzeihen.
Matthew Brown,
Özgür Demir,
Jacky Freiheit,
Marco Geiger,
Sebastian Germesin,
Matthias an der Heiden,
Gerd Herzog,
Thomas Kleinbauer,
Matthias Kwaterski, 
Hartmut Nebe,
Tino Ortega-Gomez,
Peter Poller,
Sigrid Poggel,
Klaus Poggel,
Gerhard Sonnenberg,
Rainer Stahlmann,
Anke Steffen,    
Hideki Yamaguchi.  
Alle in der Arbeit verbleibenden Fehler sind natürlich meine eigenen.