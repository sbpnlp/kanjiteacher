%%% Local Variables: 
%%% mode: latex
%%% TeX-master: "../KanjiHWR"
%%% End: 

\chapter{Conceptual Design of Kanji-Coach}
\label{chap:conceptualdesignofkanjicoach}

\section{Requirements of a Kanji Teaching E-Learning Application}
\label{sec:concept:requirements}

\subsection{General Considerations}
\label{sec:concept:generalconsiderations}

In order to create a concept for a Kanji teaching application,
a number of different aspects need to be taken into consideration.
These aspects emerge from the academic background concerning the Japanese script,
pedagocical and didactic knowledge about teaching languages and general
conceptions of e-learning applications.

Many efforts in designing e-learning applications are focused around the
teacher's view on learning. For designing an e-learning application that
is useful to students, the students view needs be taken into 
account~\shortcite{Alexander2007}. 
\shortciteANP{Ivashin2009}~\citeyear{Ivashin2009} critisises the technical 
dominance in e-learning and e-teaching processes, as the conceptual software
designs are not always supporting the didactic purpose of the software.
Therefore, the user view should be taken into account when conceptually designing
an e-learning application.
The requirement of a \emph{user-focused design} follows directly from this view.

For online e-learning, it is a known that readers only scan the textual 
information displayed. Therefore it is not useful to provide a user with 
large blocks of text, but rather with smaller chunks that encourage 
skimming over~\shortcite{Hamid2001}. It can be expected that the fact
that an e-learning process happens online does not greatly affect the user 
behaviour. Therefore, the observations made for online e-learning can
probably be applied to offline desktop application based e-learning.
The requirement of \emph{keeping textual information short and concise} derives
from the observation.

If e-learning is considered as a learning method in higher education, 
\emph{blended learning} seems to be the most suitable form of e-learning.
That means, combining classroom activities with e-learning 
methods~\shortcite{Hettinger2008,Kahiigi2008}. Language learning is not 
necessarily considered as higher education. In case of studying Japanese,
with its specific difficulties in language and the script, language learning
is taken to an intellectual level that is at least close to higher education.
Therefore, an e-learning application for any aspect of the Japanese language 
should not have the pretensions of posessing the capability 
to \emph{teach Japanese}. Japanese is a complex language with a complex script.
Therefore, e-learning applications should aim at supporting a learner's 
classroom efforts of studying the language. The requirement of 
\emph{focussing on a specific language aspect} can be drawn from this reasoning.
The prototype system designed in this work does not aim at being a complete
system, but rather offers individual learning components from which a user can 
choose what type of learning and which component best supports his study.

\subsection{Classification of a Kanji Teaching Application}
\label{sec:concept:classificationofakanjiteachingapplication}
%xxx What follows from E-Learning aspects of the whole thing?
%classification as an e-learning application.
%what type of application is it?
%which learning paradigm is followed?

In section~\ref{sec:elearn:classification} different types of e-learning systems
have been discussed. In the course of designing a prototype system, 
design choices need to be made.
The design choice for the prototype will be an offline e-learning system,
that runs on a desktop PC.
This design choice does not follow a conceptual requirement, in fact it ignores
\shortciteANP{Ivashin2009}'s~\citeyear{Ivashin2009} criticism of technical 
dominance in e-learning systems. The choice is a purely technical choice, 
yet, it is driven by a conceptual requirement.
The purpose of the e-learning environment is to test to what extend a handwriting
recognition can help study the Kanji. In order to study that research question,
the handwriting recognition needs to be implemented and integrated with the
e-learning system. Thus, the design choice for an offline system was 
inevitable in the sense that the technical limitations of on-line applications
form an obstacle for pen input of characters and fast recognition procedures.

In the definition given by~\shortcite{Richert2007} the prototype is a \emph{computer based training} (CBT) system, as it does not use the Internet for 
communication or a webserver for storage. Another criterion for an offline 
systems is that they are offered on CD-ROM or floppy disk.
That criterion can be regarded completly obsolete, as it refers to specific 
storage media. Even if a higher level of abstraction is used to describe the
criterion, it is still obsolete, since \emph{a passive storage medium} is not 
necessary to describe what the criterion actually tries to define.
The criteria concerning communication and data storage are useful to confine
different types of e-learning applications.
Additionally, \emph{installability} can be used as a criterion for offline 
e-learning systems. \emph{Installability} here refers to 
\emph{the possibility to install a software on a computer system}, 
not the \emph{ease of installation}, 
which is defined in ISO9126 as \emph{installability} as 
well~\shortcite{Chua2004}. 
The ISO9126 type of installability will be taken into account during the software
evaluation, which is reported in chapter~\ref{chap:implementationevaluation}.

Concerning the level of interactivity described in 
section~\ref{sec:elearn:interactivity} the prototype designed in this work is 
aimed at a level higher than 
level~\emph{(\ref{elearn:class:changingcontent})~Changing the content of a 
component}. It is targeted between the 
levels~\emph{(\ref{elearn:class:generateobjects})~Generating objects or 
the content of a representation} 
and~\emph{(\ref{elearn:class:constructivemanipulative})~Constructive and 
manipulative actions through situation-dependent feedback}.
Concretely, a user can:
\begin{itemize}
 \item Change the ideal shape of a character by storing a new gold standard.
 \item Create new characters and their descriptions
 \item Receive situation-dependent feedback even on the newly created characters,
       due to the nature of the error recognition algorithm that evaluates
       mathematically the distance between a gold standard character and
       a an input.
       Additionally, characters are analysed structurally, therefore new 
       characters added by the user will automatically be classified and arranged
       among the other characters in the database of the system.
\end{itemize}
Thus, based on the levels of interactivity~\shortcite{Richert2007},
it can be concluded that the prototype provides a very high level of 
interaction. The levels serve as an evaluation measurement for the quality of 
e-learning applications.

\subsection{Conceptual Issues for E-Learning of Kanji}
\label{sec:elearn:conceptualissuesforelearningofkanji}

%xxx state that it is not a vocabulary trainer, but a kanji trainer only.

\shortcite{Stahlmann2004} spezielle aspekte bezueglich han-trainer pro


\section{Approaching the Specific Difficulties of the Japanese Script}
\label{sec:concept:tacklingdifficulties}

%xxx Use difficulties described in Japanese section and describe
%xxx how they are cared for in this application.


\subsection{Character Learning Aspects}
\label{sec:concept:charaterlearningaspects}


- greife typische probleme der lerner auf 
  (siehe japanischkapite \ref{sec:japanesedifficulties})
  s. 11 hinten kurze auflistung. geschichten?


\subsubsection{Character Repetition}
\label{sec:concept:characterrepetition} %label in use already.

In section~\ref{sec:elearn:elearningoflanguages} the pure repetition of 
grammatical structures as a learning method has been critisised.
The system should account for that by not just forcing the user to
reproduce fixed structures. In fact, it should leave room for creativity.
Zum Beispiel - radikale vorgeben und zeichen schreiben lassen.
und ganz generell: toleranzgrenzen erlauben kreativitaet allein schon deswegen,
weil selbst der zeichenstift benutzt wird.


\section{Integration of HWR Into the Learning Process}
\label{sec:concept:integrationofhwrintolearning}

%xxx How do the HWR and the learning process interplay conceptually?
%xxx In what form do the corrections come up?
%xxx What kind of error recognition is there?
%xxx character input: timeout for learning - no button in handwriting data input view

Wichtige Fragestellung: Wie sieht eine HWR in Lernumgebung aus?
Mappe S. 19
genaue anforderungen s. 19!
waehle bestimmte architekture unter moeglichen ansaetzen.


welche art von character recognition muss geleistet werden?

was sind die moeglichkeiten (im vergleich zu anderen produkten),
die sich durch eine HWR ergeben?
wie kann man die ausschoepfen? s. 16 unten und s. 15

Error Recognition
what type of errors?
semantical errors? cow vs sheep vs pig
phonological errors (readings) kanji that sound the same.
theoretically: compounds - for the kanji readings.
heft: s. 52

- compare with normal paper-based learning of kanji
- compare with other kanji-learning systems
klare abgrenzung von skritter.
s. 51 unten im heft.


\section{Handling Errors}
\label{sec:concept:handlingerrors}

%xxx why this section? what is its purpose?
%because it is one of the crucial novelties of the system to provide an
%error handling like this, therefore it must be reported.

%what is in this section:
%how to deal with errors conceptually
%what types of errors are there and how they can be handled
%outcome visioning: educational aspects / the e-learning view
%how? how will it be structured?
%next action - what to write first?

\subsection{Motivation for Error Recognition}
\label{sec:concept:motivationforerrorrecognition}


\subsection[Sources of Error]{Possible Sources of Error When Writing Japanese Characters}
\label{sec:concept:sourcesoferror}

error handling, see page 58.


See notes on paper, seite 58
- for example stroke number and stroke sequence
- length of strokes

- stroke velocity

\section{Use Cases}
\label{sec:concept:usecases}

siehe 'screenshot' - grafiken von s. 2 - 9
auch: was kann man aus e-learning machen?
welche (technischen) moeglichkeiten sind eroeffnet,
insbesondere auch durch handschriftenerkennung?

idee: schoenschreibekurs, bei dem einzelne striche
gesondert geuebt werden.

