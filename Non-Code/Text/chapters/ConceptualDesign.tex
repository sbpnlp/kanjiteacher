%%% Local Variables: 
%%% mode: latex
%%% TeX-master: "../KanjiHWR"
%%% End: 

\chapter{Conceptual Design of Kanji-Coach}
\label{chap:conceptualdesignofkanjicoach}

\section{General Requirements}
\label{sec:concept:generalrequirements}

%xxx What follows from E-Learning aspects of the whole thing?
%xxx state that it is not a vocabulary trainer, but a kanji trainer only.

\subsection{Character Learning Aspects}
\label{sec:concept:charaterlearningaspects}

\subsubsection{Character Repetition}
\label{sec:concept:characterrepetition} %in use already.


\section{Tackling the Specific Difficulties of the Japanese Script}
\label{sec:concept:tacklingdifficulties}

%xxx Use difficulties described in Japanese section and describe
%xxx how they are cared for in this application.

\section{Integration of HWR Into the Learning Process}
\label{sec:concept:integrationofhwrintolearning}

%xxx How do the HWR and the learning process interplay conceptually?
%xxx In what form do the corrections come up?
%xxx What kind of error recognition is there?
%xxx character input: timeout for learning - no button in handwriting data input view


\section{Use Cases}
\label{sec:concept:usecases}

