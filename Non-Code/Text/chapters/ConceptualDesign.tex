%%% Local Variables: 
%%% mode: latex
%%% TeX-master: "../KanjiHWR"
%%% End: 

\chapter{Conceptual Design of Kanji-Coach}
\label{chap:conceptualdesignofkanjicoach}

- Why this section? 
  The purpose of this section is 
  It would be off purpose, if 
- What goes into this section?
  The main content of this section is 
  * if describing a problem: why is the problem relevant.
  * if describing a solution to a problem: what alternatives were
    there to solve it, why was this solution chosen? 
    what made it the best choice? was it the optimal solution?
- How will this section be structured and organised?
  The organisational structure of the section 
- In what style will it be written?
  The style of writing will be 
- Next action - what to write first?
  The next part to write is


\section{General Requirements}
\label{sec:concept:generalrequirements}

%xxx What follows from E-Learning aspects of the whole thing?
%xxx state that it is not a vocabulary trainer, but a kanji trainer only.
Wichtige Fragestellung: Wie sieht eine HWR in Lernumgebung aus?
Mappe S. 19
genaue anforderungen s. 19!
waehle bestimmte architekture unter moeglichen ansaetzen.

s. 12 beachten: WICHTIG: lernkomponenten, muss kein ganzes system sein.



\subsection{Character Learning Aspects}
\label{sec:concept:charaterlearningaspects}


- greife typische probleme der lerner auf 
  (siehe japanischkapite \ref{sec:japanesedifficulties})
  s. 11 hinten kurze auflistung. geschichten?

\subsubsection{Character Repetition}
\label{sec:concept:characterrepetition} %label in use already.


\section{Tackling the Specific Difficulties of the Japanese Script}
\label{sec:concept:tacklingdifficulties}

%xxx Use difficulties described in Japanese section and describe
%xxx how they are cared for in this application.

\section{Integration of HWR Into the Learning Process}
\label{sec:concept:integrationofhwrintolearning}

%xxx How do the HWR and the learning process interplay conceptually?
%xxx In what form do the corrections come up?
%xxx What kind of error recognition is there?
%xxx character input: timeout for learning - no button in handwriting data input view

welche art von character recognition muss geleistet werden?

was sind die moeglichkeiten (im vergleich zu anderen produkten),
die sich durch eine HWR ergeben?
wie kann man die ausschoepfen? s. 16 unten und s. 15

Error Recognition
what type of errors?
semantical errors? cow vs sheep vs pig
phonological errors (readings) kanji that sound the same.
theoretically: compounds - for the kanji readings.
heft: s. 52

- compare with normal paper-based learning of kanji
- compare with other kanji-learning systems
klare abgrenzung von skritter.
s. 51 unten im heft.



\section{Use Cases}
\label{sec:concept:usecases}

siehe 'screenshot' - grafiken von s. 2 - 9
auch: was kann man aus e-learning machen?
welche (technischen) moeglichkeiten sind eroeffnet,
insbesondere auch durch handschriftenerkennung?

idee: schoenschreibekurs, bei dem einzelne striche
gesondert geuebt werden.


\section{HWR Applied to E-learning of Japanese Kanji}
\subsection{Integration of HWR Into E-Learning Application}
educational aspects / the e-learning view

\section{Handling Errors}
\label{sec:concept:handlingerrors}

%xxx why this section? what is its purpose?
%because it is one of the crucial novelties of the system to provide an
%error handling like this, therefore it must be reported.

%what is in this section:
%how to deal with errors conceptually
%what types of errors are there and how they can be handled
%outcome visioning: educational aspects / the e-learning view
%how? how will it be structured?
%next action - what to write first?


\subsection[Sources of Error]{Possible Sources of Error When Writing Japanese Characters}
\label{sec:concept:sourcesoferror}

error handling, see page 58.


See notes on paper, seite 58
- for example stroke number and stroke sequence
- length of strokes

- stroke velocity