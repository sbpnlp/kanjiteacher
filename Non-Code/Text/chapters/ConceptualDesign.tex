%%% Local Variables: 
%%% mode: latex
%%% TeX-master: "../KanjiHWR"
%%% End: 

\chapter{Conceptual Design of Kanji-Coach}
\label{chap:conceptualdesignofkanjicoach}

\section{General Requirements}
\label{sec:concept:generalrequirements}

%xxx What follows from E-Learning aspects of the whole thing?
%xxx state that it is not a vocabulary trainer, but a kanji trainer only.
Wichtige Fragestellung: Wie sieht eine HWR in Lernumgebung aus?
Mappe S. 19
genaue anforderungen s. 19!
waehle bestimmte architekture unter moeglichen ansaetzen.

s. 12 beachten: WICHTIG: lernkomponenten, muss kein ganzes system sein.



\subsection{Character Learning Aspects}
\label{sec:concept:charaterlearningaspects}


- greife typische probleme der lerner auf 
  (siehe japanischkapite \ref{sec:japanesedifficulties})
  s. 11 hinten kurze auflistung. geschichten?

\subsubsection{Character Repetition}
\label{sec:concept:characterrepetition} %label in use already.


\section{Tackling the Specific Difficulties of the Japanese Script}
\label{sec:concept:tacklingdifficulties}

%xxx Use difficulties described in Japanese section and describe
%xxx how they are cared for in this application.

\section{Integration of HWR Into the Learning Process}
\label{sec:concept:integrationofhwrintolearning}

%xxx How do the HWR and the learning process interplay conceptually?
%xxx In what form do the corrections come up?
%xxx What kind of error recognition is there?
%xxx character input: timeout for learning - no button in handwriting data input view

welche art von character recognition muss geleistet werden?

was sind die moeglichkeiten (im vergleich zu anderen produkten),
die sich durch eine HWR ergeben?
wie kann man die ausschoepfen? s. 16 unten und s. 15

Error Recognition
what type of errors?
semantical errors? cow vs sheep vs pig
phonological errors (readings) kanji that sound the same.
theoretically: compounds - for the kanji readings.
heft: s. 52

- compare with normal paper-based learning of kanji
- compare with other kanji-learning systems
klare abgrenzung von skritter.
s. 51 unten im heft.



\section{Use Cases}
\label{sec:concept:usecases}

siehe 'screenshot' - grafiken von s. 2 - 9
auch: was kann man aus e-learning machen?
welche (technischen) moeglichkeiten sind eroeffnet,
insbesondere auch durch handschriftenerkennung?

idee: schoenschreibekurs, bei dem einzelne striche
gesondert geuebt werden.

