
%%% Local Variables: 
%%% mode: latex
%%% TeX-master: "../KanjiHWR"
%%% End: 

\chapter{On-Line Handwriting Recognition}
\label{chap:onlinehwr}

%\begin{itemize}
%\item Warum: Um Ueberblick ueber HWR-Techniken zu verschaffen
%\item Nutzen: Leser kann sich in HWR-Materie eindenken.
%\item Was: Allg. Info ueber HWR. "Wie geht das ueberhaupt?"
%\item Wie: Wiss. Report. / Zusammenfassung. Vergleich.
%\end{itemize}

\section{Introduction}
\label{sec:onlinehwrintroduction}

%xxx: see plamondon2000 intro
%xxx: see plamondon2000 2: different disciplines
%xxx: see santosh2009: introduction for related stuff

Handwriting is a very personal skill to individuals. It consists of graphical
marks on a surface, it can be used to identify a person, it has the main
purpose of communication. This is achivied by drawing letters or other 
\emph{graphemes}, which in turn represent parts of a language.
The characters have a certain basic shape, which must be recognisable
for a human in order for the communication process to function.
There are rules for the combination of letters, which have the ability - if
known to the reader - to help recognise a character or word.

Handwriting was developed as a means of communication and to expand one's own
memory. With the advent of each new technologies the question arose, 
if handwriting was going to survive. However, the opposite seems to be the 
truth: For example, the printing press increased the number of documents
available and therefore increased the number of people who learned to read
and write. Through the increased rate of alphabetisation, naturally there was
an increased use of handwriting as a means of communication.

In various situations handwriting seems much more practical than typing on a
keyboard. For instance children at school are using notepads and pencils or
ink pens, which are regarded as a better tool to teach writing by German 
teachers. Therefore it can be concluded that there is little danger of
the extinction of handwriting as a communication tool. In fact, as 
the length of handwritten messages decreases, the number of people using 
handwriting increases \shortcite{Plamondon2000}.

\section{Handwriting Features}
\label{sec:handwritingfeatures}

Any script of any language has a number of features. The fundamental 
characteristic of a script is that the differences between the features
of different characters are more decisive than the different features of 
drawing variants of the same letter in individual handwriting styles.
There might be exeptions, because \emph{0} and \emph{O} or \emph{1} and 
\emph{I} respectively, can be written alike. However, in those cases, 
context makes clear which one was intendet by the writer.
Despite the exception, written communication can only work with that
fundamental quality \shortcite{Tappert1990}.

\subsection{Handwriting Properties of Latin Script}
\label{sec:handwritingpropertieslatin}

In the latin script we have 26 letters, each of which has two variants, 
a capital and a lowercase variant. When writing a character 
in the latin script, there are four main areas, in which the character 
can reside. All characters have their main part between a top line and a 
ground line. There is also a middle line. Capital characters stretch out to use
the full space between the ground line and the top line, whereas lowercase 
characters usually use the space between the ground line and the middle line. 
Some lowercase characters (like lowercase \emph{b, d, f, h, k, l, t}) have an
ascender and use the area above the middle line as well, 
some lowercase characters have a descender and use the area below the ground 
line (like lowercase \emph{g, j, p, q, y}). In handwritten cursive script, 
there are writing variants where also some lowercase letters (\emph{f, z}) and
certain uppercase characters (\emph{G, J}) expand below the ground line.
For all latin-based alphabets, usually one character is finished before the 
next one starts, however, there are exceptions:
In cursive handwriting, the dots on \emph{i} and \emph{j} and the crosses
of \emph{t} might be delayed until the underlying portions of all 
the characters of the word are completed. 

XXX Graphic with exmaple of expanding cursive letters here.

\subsection{Handwriting Properties of East Asian Scripts}
\label{sec:handwritingpropertieseastasian}

% xxx: see tappert1990 IV handwriting properties and recognition problems
Generally, a handwriting is a formed of a number of strokes, that are drawn
in a time sequence. Opposed to the latin-based alphbets, consider Chinese and
Japanses script. Chinese has a larger alphabet, up to 50000 characters, 
3000-5000 of which are in active use. There are also two writing styles,
block style - which corresponds to printed characters in latin alphabets,
even if handwritten. The other style is cursive style. In block style the
individual parts of the character are usually written in proper stroke order,
and abide by the proper stroke number. In cursive style the characters are
written faster, with less care and don't necessarily abide to stroke
number or order. In fact, they are usually written with fewer strokes,
connecting some block-style strokes by using simpler radical 
shapes \shortcite{Tappert1990}.

In Japanese, three different scripts are in active use at the same time,
mixed and next to each other. They are called \emph{Hiragana} (\cjk{ひらがな}), 
\emph{Katakana} (\cjk{カタカナ}) and \emph{Kanji} (\cjk{漢字}).
Hiragana and Katakana are syllabic alphbets, each containing 46 characters
(see~\ref{sec:kana}), whereas Kanji are essentially the Chinese \emph{Hànzì} 
characters (\cjk{汉字}) as they were imported into the Japanese language 
(see~\ref{sec:ahorthistoryofjapanesewritingsystem}).

The different scripts can even be blended with each other within one word. 
Take for instance the verb \emph{taberu}  (\cjk{食べる} - \emph{to eat}). 
The first character is a kanji character, pronounced \emph{/ta/}, 
also bears the meaning of the word, the second and third characters 
are the hiragana characters \emph{be} and \emph{ru} which are there for 
conjugation only as well as for phonetic reasons. 
However, without them, the character \cjk{食} still bears the meaning of 
the concept \emph{eat}, but the character alone does not result in 
the verb \emph{taberu}.

\section{Automated Recognition of Handwriting}
\label{sec:autorecoofhandwriting}

\subsection{Short History of Handwriting recognition}
\label{sec:shorthistoryofhwr}

\emph{Handwriting recognition} (HWR) as a technological discipline performed 
by machines has been around for many years. The quality of the systems 
recognising handwriting has improved over the decades. It is the key 
technology to pen-based computer systems. The first research papers 
concerned with \emph{pattern recognition} on computers were published 
in the late 1950ies, \emph{Handwriting recognition} as an individual subject in 
the early 1960ies. \shortcite{Goldberg1915} describes in a US Patent
a machine that can recognise alphanumeric characters as early as 1915. 
However, despite the surprise of how early such a device was invented,
it should be taken into consideration that that was before the times 
of modern computers, therefore the methods he employs are quite different 
from the algorithms used after the advent of computers, more concretely, 
computers with screens. 

\shortcite{Tappert1990} describe in their review the development of 
handwriting recognition, which was a popular research topic in the early 
1970ies and then again in the 1980ies, due to the increased availability 
of pen-input devices.  Generally speaking, handwriting recognition (HWR) 
involves automatic conversion of handwritten text into a machine readable 
character encoding like ASCII or UTF-8. Typical HWR-environments include 
a pen or stylus that is used for the handwriting, a touch-sensitive surface, 
which the user writes on and a an application that interprets the strokes 
of the stylus on the surface and converts them into digital text. 
Usually, the writing surface captures the x-y coordinates of the stylus 
movement.

\subsection{Pattern Recognition Problems}
\label{sec:patternrecognitionproblems}

%xxx: was ist das generelle problem?
%xxx: welche probleme treten dabei auf?
xxx: related problems: see liujaegernakagawa2004 1.1
xxx: write something about the problems of different scripts here, 
xxx: too, as in tappert1990 IV: problems.

The general problem of \emph{pattern recognition} is to take a non-symbolic 
representation of some pattern, like mouse or pen coordinates and transform
it into a symbolic representation like a \emph{rectangle} with its coordinates,
or in the case of handwriting recognition, a character. 
There are several related problems, the recognition of equations, line 
drawnings and gestures symbols. The recognition of language symbols includes 
the different large alphabets of Chinese, the different scripts of Japanese, 
alphabetic scripts like Greek, or Arabic and other non-alphabetic scripts like
the Korean  Hangul, but also various writing styles of the latin-based 
alphabets, and diacritics that are used to denote pronunciation variants in 
different languages using latin script, like Turkish or Vietnamese.

\subsubsection{Similar characters}
\label{sec:similarcharacters}

XXX remake fig 2. in tappert1990 which he had stolen from his reference 240

There are several subproblems to the task of pattern recognition of a character.
Different styles of handwriting can be seen in xxx LABELOFGRAPHICABOVE. 
Towards the bottom of the graphic characters are harder to recognise. 
In the case of boxed discrete characters, segmentation is done for the 
machine by the user. Run-on discrete characters are easier to recognise 
than pure cursive handwriting, because there is a pen-up and 
pen-down between each character. In cursive handwriting, segmentation 
between the characters becomes a more difficult task. Some parts of the writing
may be delayed, like the crosses of \emph{t} or the dots on \emph{i} or \emph{j}.
Besides the segmentation, the discrimination of shapes is often not trivial.
Humans may or may not be able to decipher somebody else's handwriting and clearly
distinguish between, say \emph{U-V}, but most of the times, context helps with 
that task. Other characters have similar shapes, too, like \emph{C-L}, \emph{a-d}
and \emph{n-h}. Confusion can arise between characters and numbers like 
\emph{O-0}, \emph{I-1}, \emph{l-1}, \emph{Z-2}, \emph{S-5}, \emph{G-6}, 
\emph{I-1}, \emph{Z-2}. Lowercase and uppercase are hard to distinguish in the
cases of \emph{C-c}, \emph{K-k}, \emph{O-o}, others are mainly distinguished 
by their position relative to the base line of the rest of the text: 
\emph{P-p}, \emph{Y-y}. Therefore, context helps the human reader to identify
the correct character. This could be used as an advantage in automated pattern
recognition, as well. However, the other characters nearby would have to be
recognised first, which creates a (solvable) hen and egg problem.
In the Japanese scrpit the problem is taken to another dimension. 
For example \cjk{本}(root)-\cjk{木}(tree) is only a very simple example 
that might lead to confusion of the different symbols. However, due to the 
hierarchical organisation of the characters with their radicals
(see~\ref{sec:japanesewritingsystemtoday}) there are many more shapes that look
very much alike. From the shape recognition perspective, minor changes to the 
shape of a character can change its meaning drastically.
Compare \cjk{嘸}~(how, indeed), \cjk{撫}~stroke, pat), \cjk{蕪}~(turnip) and
\cjk{憮}~(disappointment). They all contain the radical 
\cjk{無}~(nothingness, none), which doesn't seem to have any semantic connection
with the characters, however, it can be seen as the main radical in those 
characters.











\subsection{Hardware requirements}
\label{sec:hardwarerequirements}
xxx: see santosh2009 basic tools / techniques / digitizer technology, 

Several different hardware commercial products are available in order to
capture the x-y coordinates of a stylus or pen. Graphics tablet like the
products of the Wacom Co., Ltd.\footnote{www.wacom.com} are popular input
devices for hand motions and hand gestures. The use of pen-like input devices 
has also been recommended, since 42\% of mouse users report feelings of 
weakness, stiffness and general discomfort in the wrist and hand when 
using the mouse for long periods \shortcite{Woods2002}. Moreover there are
PDAs and Tablet PCs, where the writing suface serves as an output device,
i.e. an display at the same time. New generation mobile phones also contain
touch-displays, but for those it is more common to be operated without a 
stylus. Those devices interpret user gestures, however the input is given 
directly with the users fingers. Another rather new development are real-ink 
digital pens. With those, a user can write on paper with real ink, and the pen
stores the movements of the pen-tip on the paper. The movements are transferred 
to a computer later. It can be expected that with technologies like Bluetooth
it may be possible to transfer those data in real-time, not delayed.

\subsection{Recognition vs Identification}
\label{sec:recognitionvsidentification}

Handwriting recognition is the task of transforming a spatial language 
representation into a symbolic representation. In the English language
(and many others) the symbolic representation is typically 8-bit ASCII.
However, with \emph{Unicode} being around for more than a decade now,
storage space on harddisks not being as much of an issue any more and
\emph{RAM} being readily available to the Gigabytes, it has become more 
common to use a \emph{UTF-8}  encoding, which is a variable-length character 
encoding for Unicode \shortcite{Unicode2000}.
Akin disciplines to handwriting recognition are 
\emph{handwriting identification}, which is the task of identifying the author
of a handwritten text sample from a set of writers, assuming that each
handwriting style can be seen as individual to the person who wrote it.
The task of \emph{signature verification} is to determine if a given signature
stems from the person who's name is given in the signature.
Thus, handwriting identification and verification can be used for 
analysis in the field of jurisdiction. They determine the individual features
of a handwritten sample of a specific writer and compare those
to samples given by a different or the same writer. By analysing those 
features one can find out if a piece of handwritten text is authentic or not.

\subsection{Interpretation of Handwriting}
\label{sec:interpretationofhandwriting}

Handwriting recognition and interpretation are trying to filter out the 
writer-specific variations and extract the text message only. This conversion
process can be a hard task, even for a human. Humans use context knowledge
in order to determine the likeliness of a certain message in a certain context.
For instance, a handwritten message on a shopping list that could be read
as \emph{bread} or \emph{broad} due to the similarities of the characters 
for 'e' and 'o' in some cursive handwriting styles, will be interpreted 
as \emph{bread}, since it is a much more likely interpretation in the 
shopping list domain. However, if the next word on the shopping list 
is \emph{beans}, the likelyhood for the interpretation of the first word
as \emph{broad} rises, because the collocation \emph{broad beans} is a
sequence that is likely on a shopping list, at least more likely than
having the interpretation \emph{bread} and then \emph{beans} without a
clear separation between the two. Even with non-handwritten, 
but printed characters, the human mind can be tricked because of the 
brain's ability to perform these interpretations within miliseconds 
without conscious thinking. An example of that are modern T-Shirt inscriptions
that state things like \emph{Pozilei} in a white font on a green ground 
(the German police colours in most federal states are green and white), 
which German native speakers usually read as \emph{Polizei} (police),
because that is the most likely interpretation.

\subsection{On-Line vs. Off-Line recogniton}
\label{sec:onlinevsoffline}

xxx: see plamondon2000 1.5. blend systems!
xxx: see santosh2009: off-line / on-line chapter

\emph{On-line} HWR means that the input is converted in \emph{real-time}, 
\emph{dynamically}, while the user is writing. This recognition can lag behind
the user's writing speed. \shortcite{Tappert1990} report average writing rates 
of 1.5-2.5 characters/s for English alphanumerics or 0.2-2.5 characters/s for 
Chinese characters. In online systems, the data usually comes in as a sequence 
of coordinate points.

\emph{Off-line} HWR is the application of a HWR algorithm after the writing.
It can be performed at any time after the writing has been completed. That 
includes recognition of data transferred from the real-ink pens 
(see \ref{sec:hardwarerequirements}) to a computing device after the writing
has been completed. The stadard case of off-line HWR, however, is a subset of
optical character recognition (OCR). An scanner tranfers the physical image 
on paper into a bitmap, the character recognition is performed on the bitmap.
An OCR system can recognise several hundred characters per second.

On-line devices have the dynamic information of the writing, since each point 
coordinate is captured at a specific point of time. Also, the system know the
input stroke sequence, their direction and speed of writing. All these 
information can be an advantage for an on-line system, however, off-line systems
have used algorithms of line-thinning, such that the data consits of point
coordinates, similar to the input of online systems \shortcite{Tappert1990}.

\section{A Typical On-Line HWR application}
\label{sec:atypicalonlinehwrapplication}

A typical HWR application has several parts that follow up on each other in a
procedural fashion. 
\begin{itemize}
\item \textbf{Data capturing}: The data is captured through an input device 
  like a writing surface and a stylus.
\item \textbf{Preprocessing}: The data is segmented, noise reduction like smoothing and filtering are applied.
\item \textbf{Character Recognition}: Feature analysis, stroke matching, time, 
direction and curve matching.
\end{itemize}

\subsection{Data capturing}
\label{sec:datacapturing}

how is the data captured? what format?
hardware?
xxx: see plamondon2000 1.4.
xxx: see santosh2009 sampling

\subsection{Preprocessing}
\label{sec:preprocessing}

xxx: see santosh2009 pre-processing
xxx: see tappert1990 preprocessing: segmentation, noise reduction.
xxx: see santosh2009 noise elimination
xxx: see santosh2009 normalization
xxx: see santosh2009 repetition removal

\subsection{Character Recognition}
\label{sec:characterrecognition}

xxx: see tappert1990 VI shape recognition.
xxx: see plamondon2000: 3.1.1 different models
xxx: see all the substroke stuff, santosh, shimodaira2003, nakai2003: very short, properly in OLCCR

\subsection{Postprocessing}
\label{sec:postprocessing}

xxx: what happens after the recognition process?
xxx: see tappert1990 again in postprocessing chapter.

\section{HWR of Hanzi and Kanji}
\begin{itemize}
\item Warum: Um einen Ueberblick ueber HWR-Techniken fuer Japanische 
  Schriftzeichen und verschiedene Herangehensweisen zu verschaffen.
\item Nutzen: Leser kann sich ein Bild darueber verschaffen,
  in welchem Kontext sich die Applikation bewegt.
\item Was: research different approaches, see what the focus on, 
  what their specialty is and report about them. Take different specialist 
  papers and compare them.
\item Wie: Wiss. Report. / Zusammenfassung. Vergleich. 
\end{itemize}

\subsection{The current State-of-the-Art in Japanese and Chinese Character Recognition}
From the 1990s onwards, On-Line Japanese and Chinese Character Recognition 
(OJCCR) systems have been aiming at loosening the restrictions imposed on 
the writer when using an OJCCR system. Their focus shifted from recognition 
of block style script ('regular' script) to fluent style script, 
which is also called 'cursive' style. Accuracies of up to about 95\% are
achieved in the different systems.

\cite{Nakagawa2008} report their recent results of online Japanese 
handwriting recognition and its applications. Their article gives 
important insights into character modeling, which are employed in 
this application.

xxx: bla.  says the opposite. \cite{ChenLee1996} oder auch 
xxx: \cite{Nakagawa2008} und \cite{Nakai2003} 
xxx: zu guter letzt: \cite{Santosh2009}

\subsection{Overview of a typical OJCCR system}

xxx:  \cite{LiuJaegerNakagawa2004} have said: 

xxx:  graphic:
handwritten -> character segmentation -> ... -> character codes
see fig. 3 of liujaegernakagawa2004

Broadly speaking, from an abstract viewpoint, typical handwriting recognition 
systems for Chinese and Japanese characters have the same structure like the
systems for latin-based alphbets. The process begins with \emph{Character 
segmentation}, goes on with \emph{Preprocessing}, \emph{Pattern descrpition}, 
\emph{Pattern recognition} and ends with \emph{Contextual processing}, 
if applicable. However, there are differences to the standard process, due to 
the nature of the Chinese characters (see \ref{sec:handwritingpropertieseastasian}).
Especially the pattern representation is divers in the different OJCCR systems,
whereas it is naturally more alike in the systems focussing on latin characters.
This is due to the fact that the latin alphabet is rather small, but has more
variation concerning writing style, whereas the Chinese alphabet has a larger 
inventory of characters, but less variation in how to write a character - at
least - it is widely agreed upon a 'proper' stroke sequence for a character,
even across country borders.

xxx: liujaeger2004: 4.1. structural representation. statistical representation.

xxx: liujaeger2004: character classification
xxx: liujaeger2004: very short: contextual processing.




