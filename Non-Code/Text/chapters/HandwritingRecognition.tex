%%% Local Variables: 
%%% mode: latex
%%% TeX-master: "../KanjiHWR"
%%% End: 

\chapter{On-Line Handwriting Recognition}
\label{chap:onlinehwr}

%\begin{itemize}
%\item Warum: Um Ueberblick ueber HWR-Techniken zu verschaffen
%\item Nutzen: Leser kann sich in HWR-Materie eindenken.
%\item Was: Allg. Info ueber HWR. "Wie geht das ueberhaupt?"
%\item Wie: Wiss. Report. / Zusammenfassung. Vergleich.
%\end{itemize}

\section{Introduction}
\label{sec:onlinehwrintroduction}

xxx: see plamondon2000 2: different disciplines
xxx: see santosh2009: introduction for related stuff

\subsection{The Art of Handwriting}
\label{sec:theartofhandwriting}

Handwriting is a very personal skill to individuals. It consists of graphical
marks on a surface, it can be used to identify a person, it has the main
purpose of communication. This is achivied by drawing letters or other 
\emph{graphemes}, which in turn represent parts of a language.
The characters have a certain basic shape, which must be recognisable
for a human in order for the communication process to function.
There are rules for the combination of letters, which have the ability - if
known to the reader - to help recognise a character or word.

Handwriting was developed as a means of communication and to expand one's own
memory. With the advent of each new technologies the question arose, 
if handwriting was going to survive. However, the opposite seems to be the 
truth: For example, the printing press increased the number of documents
available and therefore increased the number of people who learned to read
and write. Through the increased rate of alphabetisation, naturally there was
an increased use of handwriting as a means of communication.

In various situations handwriting seems much more practical than typing on a
keyboard. For instance children at school are using notepads and pencils or
ink pens, which are regarded as a better tool to teach writing by German 
teachers. Therefore it can be concluded that there is little danger of
the extinction of handwriting as a communication tool. In fact - it seems, as 
the length of handwritten messages decreases, the number of people using handwritin increases \shortcite{Plamondon2000}.

\subsection{Automated Recognition of Handwriting}
\label{sec:autorecoofhandwriting}

Handwriting recognition as a technological discipline performed by machines
has been around for many years. The quality of the systems recognising 
handwriting has improved over the decades. It is the key technology to 
pen-based computer systems. 
The first research papers concerned with 
\emph{pattern recognition} on computers were published in the late 1950ies, 
\emph{Handwriting recognition} as an individual subject in 
the early 1960ies. \shortcite{Goldberg1915} describes in a US Patent
a machine that can recognise alphanumeric characters as early as 1915. 
However, despite the surprise how early such a device was invented,
it should be taken into consideration that that was before the times 
of modern computers, therefore the methods he employs are quite different 
from the algorithms used after the advent of computers, more concretely, 
computers with screens.

\shortcite{Tappert1990} describe in their review the development of 
handwriting recognition, which was a popular research topic in the early 
1970ies and then again in the 1980ies, due to the increased availability 
of pen-input devices. 
Generally speaking, handwriting recognition (HWR) involves automatic 
conversion of handwritten text into a machine readable character encoding
like ASCII or UTF-8. Typical HWR-environments include a pen or stylus that 
is us ed for the handwriting, a touch-sensitive surface, which the user 
writes on and a an application that interprets the strokes of the stylus on
the surface and converts them into digital text. Usually, the writing 
surface captures the x-y coordinates of the styus movement. 

\section{Methods of Handwriting Recognition Systems}
\label{sec:methodsofhwrsystems}
xxx: see tappert1990 IV handwriting properties and recognition problems
xxx: was ist das generelle problem?
xxx: welche probleme treten dabei auf?
xxx: related problems: see liujaegernakagawa2004 1.1
xxx: write something about the problems of different scripts here, too, as in tappert1990 IV: problems.

\section{Hardware requirements}
\label{sec:hardwarerequirements}
xxx: see santosh2009 basic tools / techniques / digitizer technology, 

Several different hardware commercial products are available in order to
capture the x-y coordinates of a stylus or pen. Graphics tablet like the
products of the Wacom Co., Ltd.\footnote{www.wacom.com} are popular input
devices for hand motions and hand gestures. The use of pen-like input devices 
has also been recommended, since 42\% of mouse users report feelings of 
weakness, stiffness and general discomfort in the wrist and hand when 
using the mouse for long periods \shortcite{Woods2002}. Moreover there are
PDAs and Tablet PCs, where the writing suface serves as an output device,
i.e. an display at the same time. New generation mobile phones also contain
touch-displays, but for those it is more common to be operated without a 
stylus. Those devices interpret user gestures, however the input is given 
directly with the users fingers. Another rather new development are real-ink 
digital pens. With those, a user can write on paper with real ink, and the pen
stores the movements of the pen-tip on the paper. The movements are transferred 
to a computer later. It can be expected that with technologies like Bluetooth
it may be possible to transfer those data in real-time, not delayed.

\section{On-Line vs. Off-Line recogniton}
\label{sec:onlinevsoffline}

xxx: see plamondon2000 1.5. blend systems!
xxx: see santosh2009: off-line / on-line chapter

\emph{On-line} HWR means that the input is converted in \emph{real-time}, 
\emph{dynamically}, while the user is writing. This recognition can lag behind
the user's writing speed. \shortcite{Tappert1990} report average writing rates 
of 1.5-2.5 characters/s for English alphanumerics or 0.2-2.5 characters/s for 
Chinese characters. In online systems, the data usually comes in as a sequence 
of coordinate points.

\emph{Off-line} HWR is the application of a HWR algorithm after the writing.
It can be performed at any time after the writing has been completed. That 
includes recognition of data transferred from the real-ink pens 
(see \ref{sec:hardwarerequirements}) to a computing device after the writing
has been completed. The stadard case of off-line HWR, however, is a subset of
optical character recognition (OCR). An scanner tranfers the physical image 
on paper into a bitmap, the character recognition is performed on the bitmap.
An OCR system can recognise several hundred characters per second.

On-line devices have the dynamic information of the writing, since each point 
coordinate is captured at a specific point of time. Also, the system know the
input stroke sequence, their direction and speed of writing. All these 
information can be an advantage for an on-line system, however, off-line systems
have used algorithms of line-thinning, such that the data consits of point
coordinates, similar to the input of online systems \shortcite{Tappert1990}.

\section{Typical On-Line HWR application}
\label{sec:typicalonlinehwrapplication}

A typical HWR application has several parts that follow up on each other in a
procedural fashion. 
\begin{itemize}
\item \textbf{Data capturing}: The data is captured through an input device 
  like a writing surface and a stylus.
\item \textbf{Preprocessing}: The data is segmented, noise reduction like smoothing and filtering are applied.
\item \textbf{Character Recognition}: Feature analysis, stroke matching, time, 
direction and curve matching.
\end{itemize}

\subsection{Data capturing}
\label{sec:datacapturing}

how is the data captured? what format?
hardware?
xxx: see plamondon2000 1.4.
xxx: see santosh2009 sampling

\subsection{Preprocessing}
\label{sec:preprocessing}

xxx: see santosh2009 pre-processing
xxx: see tappert1990 preprocessing: segmentation, noise reduction.
xxx: see santosh2009 noise elimination
xxx: see santosh2009 normalization
xxx: see santosh2009 repetition removal

\subsection{Character Recognition}
\label{sec:characterrecognition}

xxx: see tappert1990 VI shape recognition.
xxx: see plamondon2000: 3.1.1 different models
xxx: see all the substroke stuff, santosh, shimodaira2003, nakai2003: very short, properly in OLCCR

\subsection{Postprocessing}
\label{sec:postprocessing}

xxx: what happens after the recognition process?
xxx: see tappert1990 again in postprocessing chapter.

\section{HWR of Hanzi and Kanji}
\begin{itemize}
\item Warum: Um einen Ueberblick ueber HWR-Techniken fuer Japanische 
  Schriftzeichen und verschiedene Herangehensweisen zu verschaffen.
\item Nutzen: Leser kann sich ein Bild darueber verschaffen,
  in welchem Kontext sich die Applikation bewegt.
\item Was: research different approaches, see what the focus on, 
  what their specialty is and report about them. Take different specialist 
  papers and compare them.
\item Wie: Wiss. Report. / Zusammenfassung. Vergleich. 
\end{itemize}

\subsection{The current State-of-the-Art in Japanese and Chinese Character Recognition}
From the 1990s onwards, On-Line Japanese and Chinese Character Recognition 
(OJCCR) systems have been aiming at loosening the restrictions imposed on 
the writer when using an OJCCR system. Their focus shifted from recognition 
of block style script ('regular' script) to fluent style script, 
which is also called 'cursive' style. Accuracies of up to about 95\% are
achieved in the different systems.


xxx: bla.  says the opposite. \cite{ChenLee1996} oder auch 
xxx: \cite{Nakagawa2008} und \cite{Nakai2003} 
xxx: zu guter letzt: \cite{Santosh2009}

\subsection{Overview of a typical OJCCR system}

xxx:  \cite{LiuJaegerNakagawa2004} have said: 

xxx:  graphic:
handwritten -> character segmentation -> ... -> character codes
see fig. 3 of liujaegernakagawa2004

Broadly speaking, from an abstract viewpoint, typical handwriting recognition 
systems for Chinese and Japanese characters have the same structure like the
systems for latin-based alphbets. The process begins with \emph{Character 
segmentation}, goes on with \emph{Preprocessing}, \emph{Pattern descrpition}, 
\emph{Pattern recognition} and ends with \emph{Contextual processing}, 
if applicable. However, there are differences to the standard process, due to 
the nature of the Chinese characters (see \ref{sec:methodsofhwrsystems}).
Especially the pattern representation is divers in the different OJCCR systems,
whereas it is naturally more alike in the systems focussing on latin characters.
This is due to the fact that the latin alphabet is rather small, but has more
variation concerning writing style, whereas the Chinese alphabet has a larger 
inventory of characters, but less variation in how to write a character - at
least - it is widely agreed upon a 'proper' stroke sequence for a character,
even across country borders.

xxx: liujaeger2004: 4.1. structural representation. statistical representation.

xxx: liujaeger2004: character classification
xxx: liujaeger2004: very short: contextual processing.




