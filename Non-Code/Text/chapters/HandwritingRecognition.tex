\chapter{Handwriting Recognition}
\section{General methodologies of HWR}
\begin{itemize}
\item Warum: Um Ueberblick ueber HWR-Techniken zu verschaffen
\item Nutzen: Leser kann sich in HWR-Materie eindenken.
\item Was: Allg. Info ueber HWR. "Wie geht das ueberhaupt?"
\item Wie: Wiss. Report. / Zusammenfassung. Vergleich.
\end{itemize}

\section{HWR of Hanzi and Kanji}
\begin{itemize}
\item Warum: Um einen Ueberblick ueber HWR-Techniken fuer Japanische 
  Schriftzeichen und verschiedene Herangehensweisen zu verschaffen.
\item Nutzen: Leser kann sich ein Bild darueber verschaffen,
  in welchem Kontext sich die Applikation bewegt.
\item Was: research different approaches, see what the focus on, 
  what their specialty is and report about them
\item Wie: Wiss. Report. / Zusammenfassung. Vergleich.
\end{itemize}

\begin{CJK}
  naesot
\end{CJK}

\begin{CJK}
  super!

\end{CJK}
\subsection{The current State-of-the-Art in Japanese and Chinese Character Recognition}
From the 1990s onwards, On-Line Japanese and Chinese Character Recognition 
(OJCCR) systems have been aiming at loosening the restrictions imposed on 
the writer when using an OJCCR system. Their focus shifted from recognition 
of block style script ('regular' script) to fluent style script, 
which is also called 'cursive' style. Accuracies of up to about 95\% are
achieved in the different systems. \cite{LiuJaegerNakagawa2004} have said: 
bla. \cite{Tappert1990} says the opposite. \cite{ChenLee1996} oder auch 
\cite{Nakagawa2008} und \cite{Nakai2003} zu guter letzt: \cite{Santosh2009}

\subsection{Overview of a typical OJCCR system}

\subsection{Typical HWR application}


