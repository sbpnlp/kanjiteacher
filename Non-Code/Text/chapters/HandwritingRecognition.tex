
\chapter{On-Line Handwriting Recognition}
\label{chap:onlinehwr}
  
\section{General Methods}
\label{sec:generalmethods}

\begin{itemize}
\item Warum: Um Ueberblick ueber HWR-Techniken zu verschaffen
\item Nutzen: Leser kann sich in HWR-Materie eindenken.
\item Was: Allg. Info ueber HWR. "Wie geht das ueberhaupt?"
\item Wie: Wiss. Report. / Zusammenfassung. Vergleich.
\end{itemize}

Handwriting recognition has been around for many years. The first research
papers concerned with pattern recognition on computers were published in
the late 1950ies, Handwriting recognition as an individual subject in 
the early 1960ies. \shortcite{Goldberg1915} describes a machine that can
recognise alphanumeric characters in a US Patent as early as 1915, 
however that was before the times of modern computers, 
therefore the methods he employs are different from the algorithms used
after the advent of computers, more concretely, computers with screens.

Generally speaking, handwriting recognition (HWR) involves automatic 
conversion of handwritten text into a machine readable character encoding
like ASCII or UTF-8. Typical HWR-environments include a pen or stylus that is us
ed for the handwriting, a touch-sensitive surface, which the user writes on
and a an application that interprets the strokes of the stylus on the surface
and converts them into characters.







shortcite: \shortcite{Tappert1990} 

shortciteA: \shortciteA{Tappert1990} 

%fullcite: \fullcite{Tappert1990} 

%fullsiteAN: \fullciteA{Tappert1990} 

cite: \cite{Tappert1990} 

citeyear: \citeyear{Tappert1990}

describe in their review the development of handwriting recognition, which was a popular research topic in the early 1970ies and then again in the 1980ies, due to the increased availability of pen-input devices.
xxx



\section{HWR of Hanzi and Kanji}
\begin{itemize}
\item Warum: Um einen Ueberblick ueber HWR-Techniken fuer Japanische 
  Schriftzeichen und verschiedene Herangehensweisen zu verschaffen.
\item Nutzen: Leser kann sich ein Bild darueber verschaffen,
  in welchem Kontext sich die Applikation bewegt.
\item Was: research different approaches, see what the focus on, 
  what their specialty is and report about them
\item Wie: Wiss. Report. / Zusammenfassung. Vergleich.
\end{itemize}

\subparagraph{the only subpara}

and what does this look like?

\subsection{The current State-of-the-Art in Japanese and Chinese Character Recognition}
From the 1990s onwards, On-Line Japanese and Chinese Character Recognition 
(OJCCR) systems have been aiming at loosening the restrictions imposed on 
the writer when using an OJCCR system. Their focus shifted from recognition 
of block style script ('regular' script) to fluent style script, 
which is also called 'cursive' style. Accuracies of up to about 95\% are
achieved in the different systems. \cite{LiuJaegerNakagawa2004} have said: 
bla.  says the opposite. \cite{ChenLee1996} oder auch 
\cite{Nakagawa2008} und \cite{Nakai2003} zu guter letzt: \cite{Santosh2009}

\subsection{Overview of a typical OJCCR system}

\subsection{Typical HWR application}

