
%%% Local Variables: 
%%% mode: latex
%%% TeX-master: "../KanjiHWR"
%%% End: 

\chapter{On-Line Handwriting Recognition}
\label{chap:onlinehwr}
  
\section{General Methods}
\label{sec:generalmethods}

%\begin{itemize}
%\item Warum: Um Ueberblick ueber HWR-Techniken zu verschaffen
%\item Nutzen: Leser kann sich in HWR-Materie eindenken.
%\item Was: Allg. Info ueber HWR. "Wie geht das ueberhaupt?"
%\item Wie: Wiss. Report. / Zusammenfassung. Vergleich.
%\end{itemize}

Handwriting recognition has been around for many years. The first research
papers concerned with pattern recognition on computers were published in
the late 1950ies, Handwriting recognition as an individual subject in 
the early 1960ies. \shortcite{Goldberg1915} describes a machine that can
recognise alphanumeric characters in a US Patent as early as 1915, 
however that was before the times of modern computers, 
therefore the methods he employs are different from the algorithms used
after the advent of computers, more concretely, computers with screens.

\shortcite{Tappert1990} describe in their review the development of 
handwriting recognition, which was a popular research topic in the early 
1970ies and then again in the 1980ies, due to the increased availability 
of pen-input devices. 
Generally speaking, handwriting recognition (HWR) involves automatic 
conversion of handwritten text into a machine readable character encoding
like ASCII or UTF-8. Typical HWR-environments include a pen or stylus that 
is us ed for the handwriting, a touch-sensitive surface, which the user 
writes on and a an application that interprets the strokes of the stylus on
the surface and converts them into digital text. Usually, the writing 
surface captures the x-y coordinates of the styus movement. 

\section{Hardware requirements}
\label{sec:hardwarerequirements}

Several different hardware commercial products are available in order to
capture the x-y coordinates of a stylus or pen. Graphics tablet like the
products of the Wacom Co., Ltd.\footnote{www.wacom.com} are popular input
devices for hand motions and hand gestures. The use of pen-like input devices 
has also been recommended, since 42\% of mouse users report feelings of 
weakness, stiffness and general discomfort in the wrist and hand when 
using the mouse for long periods \shortcite{Woods2002}. Moreover there are
PDAs and Tablet PCs, where the writing suface serves as an output device,
i.e. an display at the same time. New generation mobile phones also contain
touch-displays, but for those it is more common to be operated without a 
stylus. Those devices interpret user gestures, however the input is given 
directly with the users fingers. Another rather new development are real-ink 
digital pens. With those, a user can write on paper with real ink, and the pen
stores the movements of the pen-tip on the paper. The movements are transferred 
to a computer later. It can be expected that with technologies like Bluetooth
it may be possible to transfer those data in real-time, not delayed.

\section{On-Line vs. Off-Line recogniton}
\label{sec:onlinevsoffline}

\emph{On-line} HWR means that the input is converted in \emph{real-time}, 
\emph{dynamically}, while the user is writing. This recognition can lag behind
the user's writing speed. \shortcite{Tappert1990} report average writing rates 
of 1.5-2.5 characters/s for English alphanumerics or 0.2-2.5 characters/s for 
Chinese characters. In online systems, the data usually comes in as a sequence 
of coordinate points.

\emph{Off-line} HWR is the application of a HWR algorithm after the writing.
It can be performed at any time after the writing has been completed. That 
includes recognition of data transferred from the real-ink pens 
(see \ref{sec:hardwarerequirements}) to a computing device after the writing
has been completed. The stadard case of off-line HWR, however, is a subset of
optical character recognition (OCR). An scanner tranfers the physical image 
on paper into a bitmap, the character recognition is performed on the bitmap.
An OCR system can recognise several hundred characters per second.

On-line devices have the dynamic information of the writing, since each point 
coordinate is captured at a specific point of time. Also, the system know the
input stroke sequence, their direction and speed of writing. All these 
information can be an advantage for an on-line system, however, off-line systems
have used algorithms of line-thinning, such that the data consits of point
coordinates, similar to the input of online systems \shortcite{Tappert1990}.

\section{Typical On-Line HWR application}
\label{sec:typicalonlinehwrapplication}

A typical HWR application has several parts that follow up on each other in a
procedural fashion. 
\begin{itemize}
\item \textbf{Data capturing}: The data is captured through an input device 
  like a writing surface and a stylus.
\item \textbf{Preprocessing}: The data is segmented, noise reduction like smoothing and filtering are applied.
\item \textbf{Character Recognition}: Feature analysis, stroke matching, time, 
direction and curve matching.
\end{itemize}

\section{HWR of Hanzi and Kanji}
\begin{itemize}
\item Warum: Um einen Ueberblick ueber HWR-Techniken fuer Japanische 
  Schriftzeichen und verschiedene Herangehensweisen zu verschaffen.
\item Nutzen: Leser kann sich ein Bild darueber verschaffen,
  in welchem Kontext sich die Applikation bewegt.
\item Was: research different approaches, see what the focus on, 
  what their specialty is and report about them
\item Wie: Wiss. Report. / Zusammenfassung. Vergleich.
\end{itemize}

\subparagraph{the only subpara}

and what does this look like?

\subsection{The current State-of-the-Art in Japanese and Chinese Character Recognition}
From the 1990s onwards, On-Line Japanese and Chinese Character Recognition 
(OJCCR) systems have been aiming at loosening the restrictions imposed on 
the writer when using an OJCCR system. Their focus shifted from recognition 
of block style script ('regular' script) to fluent style script, 
which is also called 'cursive' style. Accuracies of up to about 95\% are
achieved in the different systems. \cite{LiuJaegerNakagawa2004} have said: 
bla.  says the opposite. \cite{ChenLee1996} oder auch 
\cite{Nakagawa2008} und \cite{Nakai2003} zu guter letzt: \cite{Santosh2009}

\subsection{Overview of a typical OJCCR system}


