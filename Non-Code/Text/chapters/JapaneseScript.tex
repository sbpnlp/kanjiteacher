%%% Local Variables: 
%%% mode: latex
%%% TeX-master: "../KanjiHWR"
%%% End: 

\chapter{Japanese Script}
\label{sec:japansescript}

The Japanese writing system has a long history. It goes back to around 800 A.D. 
In this chapter, the development of the script will be reviewed in 
section~\ref{sec:ahorthistoryofjapanesewritingsystem}.
In section~\ref{sec:modernjapanesewritingsystem} the present Japanese writing 
system will be demonstrated, with a focus on the Kanji in 
section~\ref{sec:compositionofkanji}. The difficulties of learning to write
Japanese will be illustrated in section~\ref{sec:writingjapanesedifficulties}.

\section{A Short History of the Japanese Script}
\label{sec:ahorthistoryofjapanesewritingsystem}

The historical development of the Japanese script ... aaahhh murks!
\subsection{Timeline}
\label{sec:developmenttimeline}

xxx: see \shortcite{Foljanty1984} 2.1.1-2.1.3
xxx: see wikipedia article

\subsection{Development}
\label{sec:developmentofjapanesescript}

\section{The Modern Japanese Writing System}
\label{sec:modernjapanesewritingsystem}

xxx: see \shortcite{Foljanty1984} 3.1
xxx: see \shortcite{Lange1922} p.64
xxx: see \shortcite{Tsujimura2007} for morphology stuff

xxx: aufbau des schriftsystems generell
xxx: Gemischtschreibung
xxx: Kurze erwaehnung der morphologie. Hiragana an verben zur konjugation.
     zusammenhang verben / nomen in kanji, 
xxx: uppercase / lowercase nicht vorhanden. etc.
xxx: see \url{http://japanese.about.com/library/weekly/aa070101a.htm}
xxx: see \url{http://www.csse.monash.edu.au/~jwb/cgi-bin/wwwjdic.cgi?1R}

\subsection{Kana \cjk{かな}}
\label{sec:kana}

xxx: see \shortcite{Foljanty1984} 2.2
xxx: see \shortcite{Lange1922} p.57ff

\subsubsection{Hiragana \cjk{ひらがな}}
\label{sec:hiragana}

\subsubsection{Katakana \cjk{カタカナ}}
\label{sec:katakana}

\subsection{Composition of the Kanji \cjk{漢字}}
\label{sec:compositionofkanji}

xxx: see \shortcite{Lange1922} p.64

\subsubsection{Graphemic Elements}
\label{sec:graphemicelements}

xxx: see \shortcite{Foljanty1984} 2.1.4.2

\subsubsection{Radicals}
\label{sec:radicals}

xxx: see \shortcite{Foljanty1984} 2.1.5
xxx: see \shortcite{Lange1922} p.85ff p.94ff

\subsubsection{Readings}
\label{sec:readings}

\subsection{Structure of the Japanese Writing System}
\label{sec:structureofwritingsystem}
Having demonstrated the Hiragana in~\ref{sec:hiragana}, the Katakana 
in~\ref{sec:katakana} and the Kanji in section~\ref{sec:compositionofkanji}, 
it is now possible to report about the structure of the writing system as such.

xxx: see \shortcite{Foljanty1984} 3.1-3.2

\subsection{Romaji \cjk{ロマジ}}
\label{sec:romaji}
xxx: see \shortcite{Foljanty1984} 4

\subsection{Machine Writing of Japanese}
\label{sec:machinewritingofjapanese}

xxx: see \shortcite{Lange1922} p. XII Stichwort Drucklegung
xxx: see \shortcite{Foljanty1984} 5
xxx: see MS IME description (technical report or something?!)
xxx: see section \ref{sec:hwrofhanziandkanji} for a description of research 
efforts in order to provide technology for using handwriting as an input method 
for Japanese.
xxx: see \shortcite{Grassmuck1997}


\section{Difficulties of Writing Japanese for Learners}
\label{sec:writingjapanesedifficulties}


xxx: find places for citations of the following paper (if not already done)
\shortcite{Foljanty1984}
\shortcite{Lange1922}
\shortcite{Katsuki2006a}
\shortcite{Katsuki2006b}
\shortcite{Haschke2008}
\shortcite{Tsujimura2007}