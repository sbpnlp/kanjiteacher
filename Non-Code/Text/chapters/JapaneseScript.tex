%%% Local Variables: 
%%% mode: latex
%%% TeX-master: "../KanjiHWR"
%%% End: 

\chapter{Japanese Script}
\label{sec:japansescript}

The Japanese writing system has a long history. It goes back to around 800 A.D. 
The Japanese script is in fact a writing system, as Japanese is denoted in 
a combination of three different scripts: \emph{Hiragana}, \emph{Katakana} and 
\emph{Kanji}. Kanji is a conceptual script, where each character bears the 
meaning of one or more semantic concepts and represents morphemes. 
Hiragana and Katakana are both syllabic scripts, and the individual characters do
not bear reference to concepts or even words, but merely to phonological units, 
usually two phonemes.

In this chapter, the development of the script will be reviewed in 
section~\ref{sec:ahorthistoryofjapanesewritingsystem}.
In section~\ref{sec:modernjapanesewritingsystem} the current Japanese writing 
system will be exemplified, with a focus on the Kanji in 
section~\ref{sec:compositionofkanji}. Hiragana and Katakana will be reviewed in
section~\ref{sec:kana}, which centers around the Kana scripts. 
Machine processing of the different Japanese scripts and the difficulties that
go along will be demonstrated in sectinon~\ref{sec:machinewritingofjapanese}.
The difficulties of learning to use the Japanese script will be illustrated in 
section~\ref{sec:writingjapanesedifficulties}.

\section{A Short History of the Japanese Script}
\label{sec:ahorthistoryofjapanesewritingsystem}

The historical development of the Japanese script is tightly connected to the 
history of the Kanji characters. Kanji, in Japanese 
\cjk{漢字}~(eng.~lit.~\emph{Han~characters}) refers to the 'characters of the 
han', meaning Chinese. yyy

\subsection{Timeline}
\label{sec:developmenttimeline}

xxx: see \shortcite{Foljanty1984} 2.1.1-2.1.3
xxx: see wikipedia article

\subsection{Development}
\label{sec:developmentofjapanesescript}

\section{The Modern Japanese Writing System}
\label{sec:modernjapanesewritingsystem}

xxx: see \shortcite{Foljanty1984} 3.1
xxx: see \shortcite{Lange1922} p.64
xxx: see \shortcite{Tsujimura2007} for morphology stuff

xxx: aufbau des schriftsystems generell
xxx: Gemischtschreibung
xxx: Kurze erwaehnung der morphologie. Hiragana an verben zur konjugation.
     zusammenhang verben / nomen in kanji, 
xxx: uppercase / lowercase nicht vorhanden. etc.
xxx: see \url{http://japanese.about.com/library/weekly/aa070101a.htm}
xxx: see \url{http://www.csse.monash.edu.au/~jwb/cgi-bin/wwwjdic.cgi?1R}

\subsection{Kana \cjk{かな}}
\label{sec:kana}

xxx: see \shortcite{Foljanty1984} 2.2
xxx: see \shortcite{Lange1922} p.57ff

\subsubsection{Hiragana \cjk{ひらがな}}
\label{sec:hiragana}

\subsubsection{Katakana \cjk{カタカナ}}
\label{sec:katakana}

\subsection{Composition of the Kanji \cjk{漢字}}
\label{sec:compositionofkanji}

xxx: see \shortcite{Lange1922} p.64

\subsubsection{Graphemic Elements}
\label{sec:graphemicelements}

xxx: see \shortcite{Foljanty1984} 2.1.4.2

\subsubsection{Radicals}
\label{sec:radicals}

xxx: see \shortcite{Foljanty1984} 2.1.5
xxx: see \shortcite{Lange1922} p.85ff p.94ff

\subsubsection{Readings}
\label{sec:readings}

\subsection{Structure of the Japanese Writing System}
\label{sec:structureofwritingsystem}
Having demonstrated the Hiragana in~\ref{sec:hiragana}, the Katakana 
in~\ref{sec:katakana} and the Kanji in section~\ref{sec:compositionofkanji}, 
it is now possible to report about the structure of the writing system as such.

xxx: see \shortcite{Foljanty1984} 3.1-3.2

\subsection{Romaji \cjk{ロマジ}}
\label{sec:romaji}
xxx: see \shortcite{Foljanty1984} 4

\subsection{Machine Writing of Japanese}
\label{sec:machinewritingofjapanese}

Machine processing of the Japanese scripts has been an issue, ever since humans
started to automate their writing.

xxx: see \shortcite{Lange1922} p. XII Stichwort Drucklegung
xxx: see \shortcite{Foljanty1984} 5
xxx: see MS IME description (technical report or something?!)
xxx: see section \ref{sec:hwrofhanziandkanji} for a description of research 
efforts in order to provide technology for using handwriting as an input method 
for Japanese.
xxx: see \shortcite{Grassmuck1997}


\section{Difficulties of Writing Japanese for Learners}
\label{sec:writingjapanesedifficulties}


xxx: find places for citations of the following paper (if not already done)
\shortcite{Foljanty1984}
\shortcite{Lange1922}
\shortcite{Katsuki2006a}
\shortcite{Katsuki2006b}
\shortcite{Haschke2008}
\shortcite{Tsujimura2007}