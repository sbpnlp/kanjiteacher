%%% Local Variables: 
%%% mode: latex
%%% TeX-master: "../KanjiHWR"
%%% End: 

\chapter{Japanese Script}
\label{sec:japansescript}

The Japanese writing system has a long history. It goes back to around 800 A.D. 
The Japanese script is in fact a writing system, as Japanese is denoted in 
a combination of three different scripts: \emph{Hiragana}, \emph{Katakana} and 
\emph{Kanji}. Kanji is a conceptual script, where each character bears the 
meaning of one or more semantic concepts and represents morphemes. 
Hiragana and Katakana are both syllabic scripts, and the individual characters do
not bear reference to concepts or even words, but merely to phonological units, 
usually two phonemes.

In this chapter, the development of the script will be reviewed in 
section~\ref{sec:ashorthistoryofjapanesewritingsystem}.
In section~\ref{sec:modernjapanesewritingsystem} the current Japanese writing 
system will be exemplified, with a focus on the Kanji in 
section~\ref{sec:compositionofkanji}. Hiragana and Katakana will be reviewed in
section~\ref{sec:kana}, which centers around the Kana scripts. 
Machine processing of the different Japanese scripts and the difficulties that
go along will be demonstrated in section~\ref{sec:machinewritingofjapanese}.
The difficulties of learning to use the Japanese script will be illustrated in 
section~\ref{sec:writingjapanesedifficulties}.

\section{A Short History of the Japanese Script}
\label{sec:ashorthistoryofjapanesewritingsystem}

The historical development of the Japanese script is tightly connected to the 
history of the Kanji characters. Kanji, in Japanese 
\cjk{漢字}~(Jap. pron. \cjk{カンジ} / kanji; Eng.~lit.~\emph{Han~characters}) 
refers to the 'characters of the Han', meaning the Han Dynasty 
(206 B.C.-220 A.D.; simplified Chinese: \cjk{汉朝}; traditional Chinese: 
\cjk{漢朝}) \shortcite{Foljanty1984}. In Mandarin the same characters are 
referred to as \emph{Hànzì} (simplified Chinese: \cjk{汉字}; 
trad. Chinese: \cjk{漢字}).
Note, that the first character \cjk{漢} (Chin. 'han', Jap. 'kan', Eng. 'Han') 
of both the words \emph{Han dynasty} and \emph{Kanji} is identical in Japanese 
and traditional Chinese, even though it has a different pronunciation in the 
Chinese and Japanese language. In traditional Chinese the character with
the same meaning (\cjk{汉}) has a different shape. This apparent oddity will be 
explained in greater detail in 
section~\ref{sec:historicaldevelopmentofjapanesescript}.

\subsection{Historical Development}
\label{sec:historicaldevelopmentofjapanesescript}
%xxx: see \shortcite{Foljanty1984} 2.1.1-2.1.3
%xxx: see wikipedia article
%xxx: see \shortcite{Grassmuck1997}
%xxx: see \shortcite{Chamberlain1982} for the Kojiki

\subsubsection{History of the Kanji}
\label{sec:historyofthekanji}

The Kanji script as developed and coined by the Han is in principle still valid 
today. It is used alone or in combination with phonetic spelling in China, Japan,
Taiwan, Hongkong. In Vietnam it was used before it was replaced with the 
Vietnamese alphabet (Viet.: 'quốc ngữ', Eng. lit. 'national language', Eng. 
'national script'), a script based on the Latin alphabet. In South Korea the Han 
characters were in use until they were replaced with 
Hangul~(Kor. with Han characters \cjk{韓國語}; Eng. 'Korean')~\shortcite{Foljanty1984}.

The Kanji characters were brought to Japan by Koreans living in Japan around 
300-400 A.D. Since the Kanji were used by the Koreans to write 
Hangul they also used it to write Japanese. There was no other Japanese script 
before that time. Reports about an original Japanese script called 
\emph{Jindai Moji} (Jap. \cjk{神代文字}; Eng. 'scripts of the age of the gods')
could not be proven. They are now assumed to be a political and speculative 
invention by Japanese Nationalists in the early 
19th.~century~\shortcite{Foljanty1984}. According to \shortcite{Lange1922}
the \emph{Kogo Shūi} (Jap. 古語拾遺; a historical record of the Inbe clan),
which was written around 800 B.C. denies the presence of a Japanese native
script before the introduction of the Han characters. However, the questions
seems irrelevant in the sence, that no longer text or document has been found,
written in that script.

In the Christian year 712 an ancestral act of writing was performed at 
Japanese emperor Temmu's court. Hieda~no~Are, a member of the guild of the 
\emph{kataribe} or reciters, basically a Japanese Griot, dictates the 
\emph{Kojiki} (Jap. \cjk{古事記}; Eng. 'Record of Ancient Matters') to 
Ō~no~Yasumaro. Ō~no~Yasumaro wrote the Kojiki, which is not the first written 
document found in Japan, however it is Japan's oldest attempt to write down 
spoken Japanese~\shortcite{Grassmuck1997,Chamberlain1982}.

At the time the Han characters were first used to write Japanese, 
they were already a developed script, more than 1,000 years old, 
as they stabilised to their modern form within the Han 
period~\shortcite{Grassmuck1997} 
\footnote{Also see timeline in section~\ref{sec:app:kanjitimeline}.}.
The first chinese characters were found on oracle bones from the Shang Dynasty
(Chinese \cjk{商朝}), which ruled over China some 500 to 600 years within the 
time period between 1600 B.C. and 1046 B.C.~\shortcite{Grassmuck1997,Guo2000}.

According to the Kojiki, a scholar called Wani (Jap. \cjk{王仁}) from Korea 
brought two foundational Chinese books to Japan, the \emph{Lunyu} 
(Simplified Chin. \cjk{论语}; trad. Chinese: \cjk{論語}; Eng. 'Analects'), 
also known as \emph{The Analects of Confucius} and 
the \emph{Qianziwen} (Chin./Jap. \cjk{千字文}; Jap. 
pron.~\cjk{センジブン}/senjibun; Eng. 'The Thousand Character Classic'),
which is a Chinese poem used as a primer for teaching Chinese characters to 
children. It contains exactly one thousand unique 
characters~\shortcite{Grassmuck1997}. The Chinese language comprehends more 
than 40,000 Hànzì characters lexicographically. Only around 25\% of those 
including about 250 \emph{Kokuji} (Jap. \cjk{国字}; Eng. 'national characters') 
are in Japanese dictionaries. Only around 2,000-3,000 of those are part of the 
common characters~\shortcite{Foljanty1984}. 

The Japanese Ministry of Education issued a list of 1,850 standard Kanji in 1946
under the name of \emph{Tōyō kanjihyō} (Jap.~\cjk{当用漢字表};
Eng.~'list of Kanji for general use'). The list of Tōyō Kanji was slightly 
revised and extended in 1981 and comprised 1,945 Kanji as the Jōyō Kanji
(Jap. \cjk{常用漢字}; Eng. 'often used Kanji')~\shortcite{Foljanty1984}.
As of 2010 a revised list of 2,131 characters is in official 
use~\shortcite{Noguchi2009}. 

In China there had been a spelling reform in the 1950s, affecting many of the
general use characters, resulting in simplified Chinese. In Japan, the Ministry
of Education issued it's own reform when the Tōyō kanji list was introduced.
However, the Japanese reform affected a smaller set of characters of only a few 
hundred and resulted in Shinjitai (Jap. shinjitai: \cjk{新字体}; 
Jap. kyūjitai: \cjk{新字體};
Jap. pron. \cjk{シンジタイ}/shinjitai; Eng. 'new character form'), 
which replaced the Kyūjitai 
(Jap. shinjitai: \cjk{旧字体}; Jap. kyūjitai: \cjk{舊字體}; 
Jap. pron.~\cjk{キュウジタイ}/kyūjitai; Eng. lit. 'old character forms'). 
This explains how some characters are still identical in traditional Chinese and 
Japanese, because they were not affected by any spelling reform, like the 
afforementioned \cjk{漢} (Jap. pron. \cjk{カン}/kan; Chin. pron. 'hàn'),
while other characters are different, like the simplified 
Chinese 'hàn': \cjk{汉}. Henceforth, and throughout this document, all Japanese characters are in the new character form shinjitai.

\subsubsection{Development of the Form of the Kanji}
\label{sec:developmentofformofkanji}


\section{The Modern Japanese Writing System}
\label{sec:modernjapanesewritingsystem}

xxx: see \shortcite{Foljanty1984} 3.1
xxx: see \shortcite{Lange1922} p.64
xxx: see \shortcite{Tsujimura2007} for morphology stuff
xxx: see \shortcite{Grassmuck1997}

xxx: aufbau des schriftsystems generell
xxx: Gemischtschreibung
xxx: Kurze erwaehnung der morphologie. Hiragana an verben zur konjugation.
     zusammenhang verben / nomen in kanji, 
xxx: uppercase / lowercase nicht vorhanden. etc.
xxx: see \url{http://japanese.about.com/library/weekly/aa070101a.htm}
xxx: see \url{http://www.csse.monash.edu.au/~jwb/cgi-bin/wwwjdic.cgi?1R}

\subsection{Kana \cjk{かな}}
\label{sec:kana}

xxx: see \shortcite{Foljanty1984} 2.2
xxx: see \shortcite{Lange1922} p.57ff

\subsubsection{Hiragana \cjk{ひらがな}}
\label{sec:hiragana}

\subsubsection{Katakana \cjk{カタカナ}}
\label{sec:katakana}

\subsection{Composition of the Kanji \cjk{漢字}}
\label{sec:compositionofkanji}

xxx: see \shortcite{Lange1922} p.64

\subsubsection{Graphemic Elements}
\label{sec:graphemicelements}

xxx: see \shortcite{Foljanty1984} 2.1.4.2

\subsubsection{Radicals}
\label{sec:radicals}

xxx: see \shortcite{Foljanty1984} 2.1.5
xxx: see \shortcite{Lange1922} p.85ff p.94ff

\subsubsection{Readings}
\label{sec:readings}

\subsection{Structure of the Japanese Writing System}
\label{sec:structureofwritingsystem}
Having demonstrated the Hiragana in~\ref{sec:hiragana}, the Katakana 
in~\ref{sec:katakana} and the Kanji in section~\ref{sec:compositionofkanji}, 
it is now possible to report about the structure of the writing system as such.

xxx: see \shortcite{Foljanty1984} 3.1-3.2

\subsection{Romaji \cjk{ロマジ}}
\label{sec:romaji}
xxx: see \shortcite{Foljanty1984} 4

\subsection{Machine Writing of Japanese}
\label{sec:machinewritingofjapanese}

Machine processing of the Japanese scripts has been an issue, ever since humans
started to automate their writing.

xxx: see \shortcite{Lange1922} p. XII Stichwort Drucklegung
xxx: see \shortcite{Foljanty1984} 5
xxx: see MS IME description (technical report or something?!)
xxx: see section \ref{sec:hwrofhanziandkanji} for a description of research 
efforts in order to provide technology for using handwriting as an input method 
for Japanese.
xxx: see \shortcite{Grassmuck1997}


\section{Difficulties of Writing Japanese for Learners}
\label{sec:writingjapanesedifficulties}


xxx: find places for citations of the following paper (if not already done)
\shortcite{Foljanty1984}
\shortcite{Lange1922}
\shortcite{Katsuki2006a}
\shortcite{Katsuki2006b}
\shortcite{Haschke2008}
\shortcite{Tsujimura2007}
\shortcite{Grassmuck1997}
