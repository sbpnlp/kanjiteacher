%%% Local Variables: 
%%% mode: latex
%%% TeX-master: "../KanjiHWR"
%%% End: 

\chapter{Notes}

This project uses state of the art Chinese/Japanese handwriting recognition 
methods in order to provide an Kanji teaching application with an error 
correction.

Conceptually, the application is an e-learning environment for Japanese 
characters, intended for the foreign learner of the Japanese language. 
In order to provide more than a  multiple choice method, like most other 
systems, the application contains a handwriting recognition engine that can
be used preferably with a hand-held device like a PDA, but generally any 
stylus input device.

The handwriting recognition method used is similar to the one proposed by Chen, 
J.-W. and S.-Y. Lee (1996) in their article "A Hierarchical Representation for 
the Reference Database of On-Line Chinese Character Recognition" in 
"Advances in Structural and Syntactical Pattern Recognition", Volume
1121 of "Lecture Notes in Computer Science", pp. 351--360. Berlin/Heidelberg, 
Germany: Springer.

\cite{Nakagawa2008} report their recent results of online Japanese 
handwriting recognition and its applications. Their article gives 
important insights into character modeling, which are employed in 
this application.


