%%% Local Variables: 
%%% mode: latex
%%% TeX-master: "../KanjiHWR"
%%% End: 

% \chapter{Outlook}
% \label{chap:futurework}
% xxx choose chapter title

\section{Future Work}
\label{sec:conclusion:futurework}

% - Why this section? 
%   The purpose of this section is to speculate about what could evolve from 
%   the research results presented in the thesis.
%   It would be off purpose, if it would become to long.
% - What goes into this section?
%   The main content of this section is a list of speculations and ideas.
%   * if describing a problem: why is the problem relevant.
%     That goes for each idea that will be presented.
% - How will this section be structured and organised?
%   The organisational structure of the section will be flat,
%   just a list of topics and a very short description for each one.
% - Next action - what to write first?
%   The next part to write is the actual list of ideas in short,
%   to I can fill it with something later.

% What can we do for other languages, using similar character sets?
The same concept of handwriting recognition could be applied to Chinese 
characters, a very similar concept could be applied to Korean characters
with a slightly modified database structure.


% This HWR engine as a product or service for other applications?
% What can be done with future devices?
% Idea: Port auf iPhone. Wieviel aufwand?
The analytical handwriting recognition engine can serve as a service for other 
applications. Instead of using a GUI on a desktop computer, new devices like
tablet PCs or the iPad could be deployed.

In the e-learning context, different devices could add to the possibilities
of the HWR-engine.
%With a multi-touch device?
In a multi-touch environment, the a learner could draw characters and
command the e-learning environment with his fingers.
It could be compared if the learning success is significantly higher, lower or
stays the same compared to using a stylus.
Additionally, an educational study could be conducted, comparing the learning 
success achieved with an e-learning environment to the learning success
using only pen and paper.
%With a pressure-sensitive device?
When using a pressure-sensitive device, the pressure-intensity could be used as
a recognition feature. However, a study would need to be conducted,
if pressure-intensity is a feature at all.
It would be plausible to assume that pressure intensity is even more individual 
to the writer than the shape of the drawn characters. 
Therefore, featuring pressure intensity in a HWR application could add more 
noise to the data. However, the differences in pressure intensity 
between different characters drawn by the same writer should be analysed.


% ISF - see section~\ref{sec:hwre:msisfformat} and section in 
% implementation chapter.
% %implementation - windows mobile 6 and ISF 
% %implementation - tablet PC and ISF

% \subsubsection{Character Repetition}
% \label{sec:concept:characterrepetition} %label in use already.

% In section~\ref{sec:elearn:elearningoflanguages} the pure repetition of 
% grammatical structures as a learning method has been critisised.
% The system should account for that by not just forcing the user to
% reproduce fixed structures. In fact, it should leave room for creativity.
% The system does not provide a free-drawing module in order to allow for
% total creativity. But, creativity is given through the use of the
% Handwriting input as such. A user can practice his own writing style.
% The tolerance thresholds given by the character recognition allow for a
% interpretation of the character shape, rather than a pure repetition.

% %Zum Beispiel - radikale vorgeben und zeichen schreiben lassen.
% %und ganz generell: toleranzgrenzen erlauben kreativitaet allein schon deswegen,
% %weil selbst der zeichenstift benutzt wird.


% Outlook:
% research e-learning.


% The user interface for character input with the stylus on a mobile device
% works similar to a drawing board. That is a more natural input than
