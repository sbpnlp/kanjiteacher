%%% Local Variables: 
%%% mode: latex
%%% TeX-master: "../KanjiHWR"
%%% End: 

\chapter{Outlook}

- Why this section? 
  The purpose of this section is 
  It would be off purpose, if 
- What goes into this section?
  The main content of this section is 
  * if describing a problem: why is the problem relevant.
  * if describing a solution to a problem: what alternatives were
    there to solve it, why was this solution chosen? 
    what made it the best choice? was it the optimal solution?
- How will this section be structured and organised?
  The organisational structure of the section 
- In what style will it be written?
  The style of writing will be 
- Next action - what to write first?
  The next part to write is

What can we do for other languages, using similar character sets?
This HWR as a service for other applications?

What can be done with future devices?
With a multi-touch device?
With a pressure-sensitive device?

Idea: Port auf iPhone. Wieviel aufwand?

See source of section~\ref{sec:datacapturing} for this paragraph here:
xxx: 
The reason for not using pressure intensity as a recognition feature might be 
that it is plausible to assume that pressure intensity is even more individual 
to the writer than the shape of the drawn characters. 
Therefore, featuring pressure intensity in a HWR application could add more 
noise to the data. However, the differences in pressure intensity 
between different characters drawn by the same writer should be analysed.
[xxx put the speculation about pressure intensity into outlook chapter? 
something along the lines of progress in device technology, 
subsection pressure intensity of tablets? Or maybe refer from here to
outlook chapter?] 
:xxx


ISF - see section~\ref{sec:hwre:msisfformat} and section in 
implementation chapter.
%implementation - windows mobile 6 and ISF 
%implementation - tablet PC and ISF



\subsubsection{Character Repetition}
\label{sec:concept:characterrepetition} %label in use already.

In section~\ref{sec:elearn:elearningoflanguages} the pure repetition of 
grammatical structures as a learning method has been critisised.
The system should account for that by not just forcing the user to
reproduce fixed structures. In fact, it should leave room for creativity.
The system does not provide a free-drawing module in order to allow for
total creativity. But, creativity is given through the use of the
Handwriting input as such. A user can practice his own writing style.
The tolerance thresholds given by the character recognition allow for a
interpretation of the character shape, rather than a pure repetition.

%Zum Beispiel - radikale vorgeben und zeichen schreiben lassen.
%und ganz generell: toleranzgrenzen erlauben kreativitaet allein schon deswegen,
%weil selbst der zeichenstift benutzt wird.

