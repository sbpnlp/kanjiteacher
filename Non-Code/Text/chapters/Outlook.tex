%%% Local Variables: 
%%% mode: latex
%%% TeX-master: "../KanjiHWR"
%%% End: 

% \chapter{Outlook}
% \label{chap:futurework}
% xxx choose chapter title

\section{Future Directions}
\label{sec:conclusion:futurework}

% - Why this section? 
%   The purpose of this section is to speculate about what could evolve from 
%   the research results presented in the thesis.
%   It would be off purpose, if it would become to long.
% - What goes into this section?
%   The main content of this section is a list of speculations and ideas.
%   * if describing a problem: why is the problem relevant.
%     That goes for each idea that will be presented.
% - How will this section be structured and organised?
%   The organisational structure of the section will be flat,
%   just a list of topics and a very short description for each one.
% - Next action - what to write first?
%   The next part to write is the actual list of ideas in short,
%   to I can fill it with something later.

%General improvement possibilities
\subsection{General Considerations for Improvement}
\label{sec:conclusion:generalconsiderationsforimprovement}
The performance results shown by the approach presented in this thesis
open up further research opportunities. The technical status of the recognition
suggests that the problem of analytical OLCCR is not solved yet.
In order to reach the recognition performance of non-analytical systems
the analytical methods should be improved.
In order to improve the recognition performance, one could combine multiple 
methods of analysis, probably even non-analytical full character recognition.
The area of pattern representation leaves room for improvement. 
The traditional structural representation mirrored in the XML format used
has an insufficiency because it does not provide additional information about
the characters. The feature-vector representation generated from the original
character's stroke sequences could be stored and modified alongside with
the actual traces.
Inspired by the research efforts of 
\shortciteANP{ChenLee1996}~\citeyear{ChenLee1996}, the database size could be
reduced by storing only parts of characters and their spatial relations
instead of full character representations. This has been considered 
a minor issue because of the ready availability of storage space.
However, that improvement could also increase recognition speed because
less patterns would need to be held in RAM and compared with the input.

Additionally, hybrid statistical-structural representations could help choosing
the best match. Both the stroke and between-strokes relationships can be
modelled statistically.


% What can we do for other languages, using similar character sets?
\subsection{Additional Research Possibilities}
\label{sec:conclusion:newresearchpossibilities}
The handwriting recognition engine developed in this thesis, can conceptually
be applied to other languages and character sets.
The same style of handwriting recognition can be applied to Chinese characters, 
in fact, on an abstract level Chinese and Japanese characters are identical,
despite their small language specific differences.
A fairly similar concept of analytical character recognition could be applied 
to Korean characters with a slightly modified database structure.
Korean characters bare the potential for an analytical handwriting recognition,
due to their structured composition. Yet, the composition for Korean characters
works in a different way than for Chinese and Japanese. A research hypothesis
for the analysis of Korean characters might be that an analytical recognition
system could have a higher accuracy, because there are are less combination 
possibilities for the substructures of Hangul.

% This HWR engine as a product or service for other applications?
% What can be done with future devices?
% Idea: Port auf iPhone. Wieviel aufwand?
A different field of additional research opportunities opened up by the
kind of analytical handwriting recognition engine developed during the
course of this thesis lies in the area of e-learning.
Because of the loosely coupled design and the SOAP interface to the engine
as a web service, the handwriting recognition can easily serve as a service 
for other applications. 
Instead of using a GUI on a desktop computer, new devices like tablet PCs or 
Apple Inc.'s iPad could be deployed.
In the e-learning context these different devices could add to the possibilities
of the HWR-engine.
For example, in a multi-touch environment, the a learner could draw characters 
and command the e-learning environment with his fingers instead of a stylus.
It could be compared if the learning success is significantly higher, lower or
stays the same compared to using a stylus.
Additionally, a fully-fledged educational study could be conducted, 
comparing the learning success achieved with an e-learning environment to the 
learning success using only pen and paper.
For a study like that there would have to be at least two considerably sized 
randomised groups of learners that would be compared with each other.

%With a pressure-sensitive device?
Another interesting research topic around the analytical handwriting recognition
could be pressure intensity.  When using a pressure-sensitive device, 
the pressure-intensity could be used as an additional recognition feature. 
Pressure-intensity offers to be a completely new feature that has not been 
exploited. However, a study would need to show if pressure-intensity 
is a feature at all. It would be plausible to assume that pressure intensity is 
very individual to the writer. 
Thus, featuring pressure intensity in a HWR application could add more 
noise to the data. However, at least the differences in pressure intensity 
between different characters drawn by the same writer should be analysed.
If pressure intensity is more individual to the writer than the shape of the 
drawn characters it could even lower recognition accuracy, 
but may be appropriate for writer identification using handwriting recognition.

% ISF - see section~\ref{sec:hwre:msisfformat} and section in 
% implementation chapter.
% %implementation - windows mobile 6 and ISF 
% %implementation - tablet PC and ISF

% \subsubsection{Character Repetition}
% \label{sec:concept:characterrepetition} %label in use already.

% In section~\ref{sec:elearn:elearningoflanguages} the pure repetition of 
% grammatical structures as a learning method has been critisised.
% The system should account for that by not just forcing the user to
% reproduce fixed structures. In fact, it should leave room for creativity.
% The system does not provide a free-drawing module in order to allow for
% total creativity. But, creativity is given through the use of the
% Handwriting input as such. A user can practice his own writing style.
% The tolerance thresholds given by the character recognition allow for a
% interpretation of the character shape, rather than a pure repetition.

