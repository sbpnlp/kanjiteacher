%%% Local Variables: 
%%% mode: latex
%%% TeX-master: "../KanjiHWR"
%%% End: 

\chapter{Japanese Language}
\label{chap:app:japaneselanguage}

% if you have free time, add a time line
% \section{Kanji Timeline} 
% \label{sec:app:kanjitimeline}

% %xxx - Why this section? 
% %xxx   The purpose of this section is 
% %xxx   It would be off purpose, if 
% %xxx - What goes into this section?
% %xxx   The main content of this section is 
% %xxx   * if describing a problem: why is the problem relevant.
% %xxx   * if describing a solution to a problem: what alternatives were
% %xxx     there to solve it, why was this solution chosen? 
% %xxx     what made it the best choice? was it the optimal solution?
% %xxx - How will this section be structured and organised?
% %xxx   The organisational structure of the section 
% %xxx - In what style will it be written?
% %xxx   The style of writing will be 
% %xxx - Next action - what to write first?
% %xxx   The next part to write is

% %xxx: make table here, containing these data

% 14th-11th centuries B.C.~or ca.~1200-1050 B.C.: First Chinese characters on Oracle bones. \shortcite{Guo2000}
% 206 B.C.~-~220 A.D.~Han dynasty, development of the \emph{Hànzì}
% 300-400 A.D.~The Hànzì characters were brought to Japan by Koreans.
% 712 A.D.~Kojiki was written


\section{Kana \cjk{かな}}
\label{sec:app:kana}

\subsection{Hiragana \cjk{ひらがな}}
\label{sec:app:hiragana}

Hiragana is a syllabic script. For a description of use and structure of
Hiragana see section~\ref{sec:hiragana}. A full table of the Hiragana script can 
be found in table~(\ref{table:fullhiragana}).
Table~(\ref{table:fullhiragana}) is adapted from \shortciteANP{Hadamitzky1995}~\citeyear{Hadamitzky1995}.

\begin{table}[htbp]
\begin{CJK}
  \begin{tabular}{c c c c c c c}
 &a&i&u&e&o&n\\
$\emptyset$&あ&い&う&え&お&ん\\
k&か&き&く&け&こ&\\
s&さ&し&す&せ&そ&\\
t&た&ち&つ&て&と&\\
n&な&に&ぬ&ね&の&\\
h&は&ひ&ふ&へ&ほ&\\
m&ま&み&む&ね&も&\\
y&や& &ゆ& &よ&\\
r&ら&り&る&れ&ろ&\\
w&わ& & & &を&\\
  \end{tabular}
\end{CJK}
\caption{The full set of Hiragana characters}
\label{table:fullhiragana}
\end{table}

\subsection{Katakana \cjk{カタカナ}}
\label{sec:app:katakana}

Katakana is the second syllabic script in use in the Japanese language. 
For a more detailed description of what it is used for, see 
section~\ref{sec:katakana}.
A full table of the Katakana script is depicted in 
table~(\ref{table:fullkatakana}).
Table~(\ref{table:fullkatakana}) is adapted from \shortciteANP{Hadamitzky1995}~\citeyear{Hadamitzky1995}.

\begin{table}[htbp]
\begin{CJK}
  \begin{tabular}{c c c c c c c}
 &a&i&u&e&o&n\\
$\emptyset$&ア&イ&ウ&エ&オ&ン\\
k&カ&キ&ク&ケ&コ&\\
s&サ&シ&ス&セ&ソ&\\
t&タ&チ&ツ&テ&ト&\\
n&ナ&ニ&ヌ&ネ&ノ&\\
h&ハ&ヒ&フ&ヘ&ホ&\\
y&ヤ& &ユ& &ヨ&\\
r&ラ&リ&ル&レ&ロ&\\
w&ワ& & & &ヲ&\\
  \end{tabular}
\end{CJK}
\caption{The full set of Katakana characters}
\label{table:fullkatakana}
\end{table}


%xxx idea: all characters in one huge picture?
%xxx generate picture with help of jim breens dictionary
%xxx maybe at least all characters the system can recognise?!
%xxx s. 15 rueckseite
