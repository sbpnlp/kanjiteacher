%%% Local Variables: 
%%% mode: latex
%%% TeX-master: "../KanjiHWR"
%%% End: 

\chapter{Implementation and Evaluation}
\label{chap:implementationevaluation}

see: ~\shortcite{Chua2004}
and: Lanzilotti_eLSE_Methodology_a_Systematic_Approach_to_the_Evaluation_of_e-Learning_Systems.pdf
find references - not in bibtex yet.

- Why this section? 
  The purpose of this section is 
  It would be off purpose, if 
- What goes into this section?
  The main content of this section is 
  * if describing a problem: why is the problem relevant.
  * if describing a solution to a problem: what alternatives were
    there to solve it, why was this solution chosen? 
    what made it the best choice? was it the optimal solution?
- How will this section be structured and organised?
  The organisational structure of the section 
- In what style will it be written?
  The style of writing will be 
- Next action - what to write first?
  The next part to write is

\section{Implementation Details}
Pointer auf CD und auf Appendix mit Beispielinteraktionen (diese mit Foto).
Screenshots.
Zahlen zur Erkennung - z.B. wie lange dauert es, ein zeichen zu erkennen?

wie wurden einzelne dinge realisiert, z.b. vectorielle funktionen?
was war neu?
klassen wie box / bounding box, technisch, alles was in HWREngine nicht behandelt
wurde.

abschnitt ueber optimierung.
optimierungszyklus inklusive ausprobieren beschreiben.
s. 51 rueckseite

s 49 rueckseite: interface-optimierung
entscheidungen herausstellen. 

s 27,28 vectorschnitt

s.11 iPhone - port of input app. checked out objective C and stuff!

see section~\ref{sec:hwre:writingsurfacegui}. describe what was difficult concerning the lists and bloody point objects.
performance issues! optimisation with try and error!

ISF - see section~\ref{sec:hwre:msisfformat}
%implementation - windows mobile 6 and ISF 
%implementation - tablet PC and ISF
% this nice feature could not be used, because it is only available from
%windows mobile 6 - not available!
%also for tablet PC - not available!

dead end of data format description:
how I first developed my own format and then found that
UPX was better.


%%%%%%%%
in \ref{sec:hwre:database} there is an undiscussed point:
   3. Description of the production of the lexicon.
      it was not just taken from j.b. but it was intervowen?! (verflochten) 
      with the trajectories. where did I get these from? 
      how many chars are in the two dictionaries

in section \ref{sec:concept:classificationinelearning} I claim that I am taking the ISO something 'installability' into account.
this should happen somewhere here.
from section \ref{sec:concept:classificationinelearning}
\begin{quote}
which is defined in ISO9126 as \emph{installability} as 
well~\shortcite{Chua2004}. 
The ISO9126 type of installability will be taken into account during the software
evaluation, which is reported in chapter~\ref{chap:implementationevaluation}.
  
\end{quote}



\section{Evaluation of the HWR}
\subsection{ Evaluation Metrics }
evaluation method: counting precision and recall
section about precision and recall - the odd numbers.
how can that be done honest and useful?
how can I get meaningful numbers at all?

s. 12 beachten: WICHTIG: eval kurz und qualitativ.

\subsection{DTW vs. 3-D DTW}
\label{sec:impl:dtwvs3ddtw} %this section already referenced.
This section al

\section{Evaluation of E-Learning Application with Integrated HWR}
qualitative auswertung, keine zahlen, sondern fragebogen.
fuehlt der lerner sich unterstuetzt? glaubt er, dass es schneller geht als ohne 
HWR? 
besser als auf papier?

\section{Evaluation of the Error Hints}
use cases, inwieweit helfen die fehlerhinweise?
geht das lernen dann wirklich schneller?
wie laesst sich der mehrwert bewerten?
system kann sagen: wo liegt die verwechslung? warum war das falsch?


