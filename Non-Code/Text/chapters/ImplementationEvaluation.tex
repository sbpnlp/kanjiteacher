%%% Local Variables: 
%%% mode: latex
%%% TeX-master: "../KanjiHWR"
%%% End: 

\chapter{Implementation and Evaluation}

why this section? 
what goes into this section?
  if describing a problem: why is the problem relevant.
  if describing a solution to a problem: what alternatives were
  there to solve it, why was this solution chosen? what made it the best
  choice? was it the optimal solution?
how will this section be structured and organised?
in what style will it be written?
next action - what to write first?



\section{Implementation Details}
Pointer auf CD und auf Appendix mit Beispielinteraktionen (diese mit Foto).
Screenshots.
Zahlen zur Erkennung - z.B. wie lange dauert es, ein zeichen zu erkennen?

wie wurden einzelne dinge realisiert, z.b. vectorielle funktionen?
was war neu?
klassen wie box / bounding box, technisch, alles was in HWREngine nicht behandelt
wurde.

abschnitt ueber optimierung.
optimierungszyklus inklusive ausprobieren beschreiben.
s. 51 rueckseite

s 49 rueckseite: interface-optimierung
entscheidungen herausstellen. 

s 27,28 vectorschnitt

s.11 iPhone - port of input app. checked out objective C and stuff!

\section{Evaluation of the HWR}
\subsection{ Evaluation Metrics }
evaluation method: counting precision and recall
section about precision and recall - the odd numbers.
how can that be done honest and useful?
how can I get meaningful numbers at all?

s. 12 beachten: WICHTIG: eval kurz und qualitativ.

\section{Evaluation of E-Learning Application with Integrated HWR}
qualitative auswertung, keine zahlen, sondern fragebogen.
fuehlt der lerner sich unterstuetzt? glaubt er, dass es schneller geht als ohne 
HWR? 
besser als auf papier?

\section{Evaluation of the Error Hints}
use cases, inwieweit helfen die fehlerhinweise?
geht das lernen dann wirklich schneller?
wie laesst sich der mehrwert bewerten?
system kann sagen: wo liegt die verwechslung? warum war das falsch?


