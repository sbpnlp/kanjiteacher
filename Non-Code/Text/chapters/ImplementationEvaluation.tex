%%% Local Variables: 
%%% mode: latex
%%% TeX-master: "../KanjiHWR"
%%% End: 

\chapter{Implementation and Evaluation}
\label{chap:implementationevaluation}

% %see: ~\shortcite{Chua2004}
% \section{Implementation Details}
% \label{sec:eval:implementationdetails}

% % Pointer auf CD und auf Appendix mit Beispielinteraktionen (diese mit Foto).
% %Screenshots.
% %Zahlen zur Erkennung - z.B. wie lange dauert es, ein zeichen zu erkennen?



% wie wurden einzelne dinge realisiert, z.b. vectorielle funktionen?
% was war neu?
% klassen wie box / bounding box, technisch, alles was in HWREngine nicht behandelt
% wurde.

% abschnitt ueber optimierung.
% optimierungszyklus inklusive ausprobieren beschreiben.
% s. 51 rueckseite

% s 49 rueckseite: interface-optimierung
% entscheidungen herausstellen. 

% s 27,28 vectorschnitt

% s.11 iPhone - port of input app. checked out objective C and stuff!

% see section~\ref{sec:hwre:writingsurfacegui}. describe what was difficult concerning the lists and bloody point objects.
% performance issues! optimisation with try and error!

% ISF - see section~\ref{sec:hwre:msisfformat}
% %implementation - windows mobile 6 and ISF 
% %implementation - tablet PC and ISF
% % this nice feature could not be used, because it is only available from
% %windows mobile 6 - not available!
% %also for tablet PC - not available!

% dead end of data format description:
% how I first developed my own format and then found that
% UPX was better.


% %%%%%%%%
% in \ref{sec:hwre:database} there is an undiscussed point:
%    3. Description of the production of the lexicon.
%       it was not just taken from j.b. but it was intervowen?! (verflochten) 
%       with the trajectories. where did I get these from? 
%       how many chars are in the two dictionaries


\section{Evaluation of the HWR}
\subsection{ Evaluation Metrics }
grundsaetzliches:
evaluation soll kurz und qualitativ sein.
plausible fakten muessen nicht unbedingt untermaurert werden.

baseline eval vs. topline eval.
technik ermoeglicht erst lernmethode, diese methode ist besser als andere methode.
kriterium fuer qualitative eval:
reproduzierbarkeit des experiments.

evaluation method: counting precision and recall
section about precision and recall - the odd numbers.
how can that be done honest and useful?
how can I get meaningful numbers at all?


s. 12 beachten: WICHTIG: eval kurz und qualitativ.

\section{E-Learning Module Evalutation}
\label{sec:eval:elearning}
Generally, there are two main directions in the evaluation methodology
for e-learning applications: the educationalist's approach and the 
software developer's approach.
Therefore, the evaluation of an e-learning system is a complex task
and requires optimisation work on the account of both the conceptual
designer of an e-learning application as well as the software developer.

The e-learning part of the prototype is a sample module that is used 
to exemplify one usage scenario of the HWR engine. 
It mainly shows plausibility of the approach. A detailed analysis of the
e-learning application using ISO9126~\shortcite{Chua2004} is not useful,
since the e-learning part of the software has not been optimised in any way.
The focus of the thesis is not to implement an e-learning application,
but to create an analytical handwriting recognition engine.
It might be a prospect for future 
work\footnote{See section~\ref{sec:conclusion:newresearchpossibilities}}
to optimise the e-learning part and build a fully-developed e-learning
application for Japanese characters, but that would be outside the scope of 
this thesis. For these reasons, there will not be an evaluation of the
e-learning module.



