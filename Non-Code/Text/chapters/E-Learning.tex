%%% Local Variables: 
%%% mode: latex
%%% TeX-master: "../KanjiHWR"
%%% End: 
\chapter{E-Learning}
\label{chap:elearning}
% xxx: s. 12 beachten: WICHTIG!
% S. 12: tilman sagt:
% Nicht Lernkonzepte vergleichen und evaluieren, 
% sondern sehen, was die Technik bietet und entsprechende use-cases darstellen.

\section{Introduction to E-Learning}
\label{sec:elearn:intro}

The term \emph{e-learning} refers to a number of different methods, concepts
and techniques. It is therefore difficult to confine the term sharply.
Thus, in literature, there are different definitions of what e-learning is
and what it is supposed to be.
\shortciteANP{Rosenberg2006} \citeyear{Rosenberg2006} defines e-learning as
follows:
  \begin{quote}
    \textbf{E-learning} is the use of Internet technologies to create and
    deliver a rich learning environment that includes a broad array of 
    instruction and information resources and solutions, the goal of which
    is to enhance individual and organizational performance.
  \end{quote}
\shortciteANP{Rosenberg2006} defines e-learning purely by terms of instruction
and information resources. Further, \emph{the use of Internet technologies} 
is seen as a necessary condition for e-learning. The definition does not 
take into account educational software.

\shortciteANP{Richert2007} \citeyear{Richert2007} critises the definition of
\shortciteANP{Rosenberg2006} because she sees no reason for such equality
of terms. She constitutes her view with the fact that electronic (learning) 
applications are not limited to the Internet. \shortciteANP{Richert2007} 
\citeyear{Richert2007} defines e-learning as:
  \begin{quote}
    Unter E-learning wird das computergestützte Lernen (vorwiegend von 
    Einzelpersonen) mit hypertextbasierten, multimedialen, interaktiven 
    Systemen verstanden, das zeit- und ortsunabhängig sowohl online als 
    auch offline erfolgen kann.
  \end{quote}
in English:
  \begin{quote}
    E-learning is defined as computer-aided learning (mainly by individuals)
    with hypertext- and multimedia-based interactive systems. The learning
    process can take place independent of time and location both online and 
    offline.
  \end{quote}
It is important to note that the term is broader than the definition of
\shortciteANP{Rosenberg2006}, but is restricted to \emph{learning systems}.
That means concretely that electronic media like dictionaries may be included
in e-learning systems as a tool, however, they can only form a part of a 
more general e-learning environment. Electronic media itself is not necessarily
understood as e-learning system.

%\section{E-Learning Methodology}
%\label{sec:elearn:methodology}

\section{Classification of E-Learning Systems} %was subsection of methodology
\label{sec:elearn:classification}

E-learning systems can be classified by their their degree of freedom for 
user interaction. On one end of the scale there are \emph{Drill-and-Practice} 
programs that do not allow for freedom of interaction. On the other end there 
are interactive programs allowing the user to interact and control the 
application. Judged by the definition of \shortciteANP{Richert2007} this 
classification does not seem very suitable~\shortcite{Richert2007}.

Another possiblility to classify e-learning systems is the the kind of storage
media used. This classification allows for a distinction between \emph{online} 
and \emph{offline} e-learning systems. \emph{Offline systems} are those systems
that are offered on passive storage media like floppy disk, CD-ROM.
Offline systems are usually called \emph{Computer Based Training} (CBS) systems.
\emph{Online systems} on the other hand are web server based systems that fall
under the category of \emph{Web Based Training} (WBS) 
systems~\shortcite{Richert2007}.

Additionally, \shortciteANP{Richert2007} \citeyear{Richert2007} defines
\emph{hybrid systems} that are CBT systems but use the Internet as a means of
communication with other learners.
Table~\ref{table:elearningsystems} shows the classification of e-learning systems
after~\shortcite{Richert2007}.
\begin{table}[htbp]
\begin{tabular}{|c|c|c|c|}
  \hline
  \multicolumn{2}{|c|}{} & \multicolumn{2}{|c|}{Using the WWW as storage medium} \\
  \cline{3-4}
  \multicolumn{2}{|c|}{} & Yes & No \\
  \hline
  \multirow{2}{*}{Using the Internet for communication} & No & WBT & CBT \\
  \cline{2-4}
   & Yes & Learning platforms & Hybrid CBT \\
  \hline
\end{tabular}
\caption{Classification of e-learning systems}
\label{table:elearningsystems}
\end{table}

\section{Technical Context of E-Learning}
\label{sec:elearn:technicalcontext}

\subsection{Multimedia Systems}
\label{sec:elearn:multimediasystems}

The term \emph{Multimedia} has several definitions. Simple versions of 
multimedia definitions state that multimedia refers to a combination of
different forms of information from several sources. Those forms can contain
textual information, graphic, video and audio. With a broad definition
of that kind any television news report could be regarded as multimedia.
\shortciteANP{Richert2007} \citeyear{Richert2007} understands \emph{multimedia}
more holistically than that. She sees multimedia as a technological concept 
that allows for the interaction of a user and a multiple media system.
More than one sensorial modality should be should be presented by the system.

\subsection{Classification of Interactivity}
\label{sec:elearn:interactivity}

\emph{Interactivity} can be defined in several steps. The concept of 
\emph{interaction} serves as a basis for the classification, because in a 
sociological sense there can, by definition, be no mutual interference
between man and machine. Interactivity in the sense of interaction comprises
the ability to access and control different functionalities of a software 
system~\shortcite{Richert2007}.

Six classes of interactivity can be described. They differ by their degree of
interaction between the user and a software system.
The gamut of interactivity is used to evaluate e-learning applications:
\begin{enumerate}
\item \textbf{View and absorb objects} \\
      The hypermedial components can be viewed and played by the user.
      The user can not further influence the components in any way.
\item \textbf{View and absorb multiple displays} \\
      Program components offer more than one display. For instance, a user
      could click on a picture and be shown a different one.
      No modification of components is possible.
\item \textbf{Varying the form of representation} \\
      On this level, users can gain the feeling they could actively influence
      the multimedia components. They can scale objects or view them from
      different perspectives. Users can influence the form of representation
      but not the content.
\item \textbf{Changing the content of a component - parameter or data 
      variation} \\
      Contents of a multimedia component are generated by the user. Users can
      input data or text. They can not change films or pictures.
      A usage example of that type could be the selection methods of statistics
      programs. Users can modify objects and the program yields different 
      results.
\item \textbf{Generating objects or the content of a representation} \\
      This mode of interaction is reached by applications that offer tools to
      create and change content. For example visualise thoughts with mindmaps,
      or render new forms and models.
\item \textbf{Constructive and manipulative actions through 
      situation-dependent feedback} \\
      On this level of interaction symbols can be manipulated and the result of
      the interpretation can be interpreted by the program.
      That allows for the generatoin of useful and context-sensitive feedback.
      User input can be evaluated by the application.
\end{enumerate}
The gamut is described after~\shortcite{Richert2007}.

\section{Pedagogical Context of E-Learning}
\label{sec:elearn:pedagogicalcontext}

The pedagogical context of e-learning is a crucial part of any e-learning
environment. The learning targets need to be defined and a conceptual design
of a software needs to be based on those.

\subsection{Learning}
\label{sec:elearn:learning}

The term \emph{learning} is of a complex nature. A definition of learning is
therefore never sharply confined. The definition of \emph{learning} by 
\shortciteANP{Lefrancois1994} \citeyear{Lefrancois1994} shows how broad
the term can be percieved:
\begin{quote}
  \emph{Lernen umfasst alle Verhaltensänderungen, 
        die aufgrund von Erfahrungen zustandekommen.}
\end{quote}
In English:
\begin{quote}
 Learning compasses all changes in behaviour that are based on experience.
\end{quote}
The changes in behaviour include those processes that do not aim at acquiring
information, but also those changes in behaviour of an unknown 
cause~\shortcite{Lefrancois1994}. According to~\shortcite{Richert2007},
this means the acquisition of competences of different kinds.

\subsubsection{Educational Objectives}
\label{sec:elearn:learningaims}



\subsubsection{Self-Driven Learning}
\label{sec:elearn:selfdrivenlearning}

\subsection{Intelligent Tutorial Systems}
\label{sec:elearn:intelligenttutorialsystems}

s 71KI, s72 tutorensysteme bei richert

\section{E-Learning of Languages}
\label{sec:elearn:elearningoflanguages}
strengthening of competences (s.96)
richert: s. 95
97ff

\section{E-Learning of Japanese Script}
\label{ser:elearn:elearningofjapanesescript}

\subsection{Conceptual Issues for E-Learning of Kanji}
\label{sec:elearn:conceptualissuesforelearningofkanji}

\subsection{Classification of a Kanji Teaching Application}
\label{sec:hwre:classificationofakanjiteachingapplication}




%\section{Japanese E-Learning Software}
%put all your bashing and criticism here
%e-learning software can be found here. zitiert von stahlmann.
%siehe wortsalat
%http://www.geotechnics.ch/Fritz/Schule/pages/lernsw/LernSW.htm


computer assisted language learning:
\shortcite{Bailey2009} %siehe seite 34 im heft

\shortcite{Zimmer2009}Bildung durch e-learning. allgemeine aspekte
\shortcite{Stahlmann2004} spezielle aspekte bezueglich han-trainer pro
\shortcite{Hettinger2008} wie kann man e-learning in der schule einsetzen?
e-learning: grundlagen, modelle, perspektiven

\shortcite{Richert2007} breite einfuehrung in e-learning theorie.

\shortcite{Seel2009} sehr breite allgemeine einfuehrung ins online-lernen
\shortcite{Ivashin2009}kritik an der technischen dominanz in elektronisch
unterstuetzten lern- und lehrprozessen.

\shortcite{Stark2002} comparison of two e-learning apps.
