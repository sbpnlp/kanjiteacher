%%% Local Variables: 
%%% mode: latex
%%% TeX-master: "../KanjiHWR"
%%% End: 
\chapter{E-learning}
\label{chap:elearning}


why this section? 
what goes into this section?
  if describing a problem: why is the problem relevant.
  if describing a solution to a problem: what alternatives were
  there to solve it, why was this solution chosen? what made it the best
  choice? was it the optimal solution?
how will this section be structured and organised?
in what style will it be written?
next action - what to write first?


\section{General E-Learning methods}
\section{E-Learning of languages}


in section 

xxx: s. 12 beachten: WICHTIG!

\section{E-Learning of Japanese}
\subsection{Conceptual issues}
\subsection{Japanese e-learning software}
put all your bashing and criticism here
%e-learning software can be found here. zitiert von stahlmann.
%siehe wortsalat
%http://www.geotechnics.ch/Fritz/Schule/pages/lernsw/LernSW.htm


shortcite nagata2002 not in bibtex yet
http://scholar.google.de/scholar?q=noriko+nagata+2002+BANZAI&hl=de&btnG=Suche&lr=

\shortcite{Bailey2009} %siehe seite 34 im heft
\shortcite{Zimmer2009}
\shortcite{Stahlmann2004}
