%%% Local Variables: 
%%% mode: latex
%%% TeX-master: "../KanjiHWR"
%%% End: 
\chapter{E-learning}
\label{chap:elearning}

\section{5 seiten max!}

% HWRE - last state:
% http://kanjiteacher.googlecode.com/svn-history/r167/Non-Code/Text/KanjiHWR.pdf

- Why this section? 
  The purpose of this section is 
  It would be off purpose, if 
- What goes into this section?
  The main content of this section is 
  * if describing a problem: why is the problem relevant.
  * if describing a solution to a problem: what alternatives were
    there to solve it, why was this solution chosen? 
    what made it the best choice? was it the optimal solution?
- How will this section be structured and organised?
  The organisational structure of the section 
- In what style will it be written?
  The style of writing will be 
- Next action - what to write first?
  The next part to write is

\section{General E-Learning methods}
\section{E-Learning of languages}


in section 

xxx: s. 12 beachten: WICHTIG!

\section{E-Learning of Japanese}
\subsection{Conceptual issues}
\subsection{Japanese e-learning software}
put all your bashing and criticism here
%e-learning software can be found here. zitiert von stahlmann.
%siehe wortsalat
%http://www.geotechnics.ch/Fritz/Schule/pages/lernsw/LernSW.htm


shortcite nagata2002 not in bibtex yet
%http://scholar.google.de/scholar?q=noriko+nagata+2002+BANZAI&hl=de&btnG=Suche&lr=

\shortcite{Bailey2009} %siehe seite 34 im heft
\shortcite{Zimmer2009}
\shortcite{Stahlmann2004}
\shortcite{Hettinger2008}
\shortcite{Richert2007}
\shortcite{Seel2009}


