%%%%%%%%%%%%%%%%%%%%%%%%%%%%%%%%%%%%%%%%%%%%%%%%%%%%%%%%%%%%%%%%%%%%%%%%%%%%%%%%
% Diploma Thesis
% Steven Buraje Poggel
%%%%%%%%%%%%%%%%%%%%%%%%%%%%%%%%%%%%%%%%%%%%%%%%%%%%%%%%%%%%%%%%%%%%%%%%%%%%%%%

\documentclass[a4]{book}
%\usepackage{a4wide}
\usepackage[margin=2cm]{geometry}
\usepackage{fontspec}
\usepackage[english]{babel}
\usepackage{url}
%\usepackage{minitoc}
%\usepackage[pdftex]{graphicx}
\usepackage{graphicx}
\usepackage{gb4e}
%\usepackage{pstricks}
\newcommand{\HRule}{\rule{\linewidth}{0.5mm}}
\usepackage{chicago}
\bibliographystyle{chicago}

\newenvironment{CJK}
{
  % \fontspec[Scale=0.9]{MS Mincho}
  \fontspec[Scale=0.9]{MingLiU} %MingLiU can also display Chinese characters as in 汉字 - which Mincho is not able to display, probably because it's not Japanese.
  \XeTeXlinebreaklocale "ja" 
  \XeTeXlinebreakskip = 0pt plus 1pt 
}{}

%\newcommand{\cjk}[1]{{\fontspec[Scale=0.9]{MS Mincho}#1}}
\newcommand{\cjk}[1]{{\fontspec[Scale=0.9]{MingLiU}#1}}
%\newcommand{\cjk}[1]{{\fontspec[Scale=0.9]{MingLiU}#1}} %MingLiU can also display Chinese characters as in 汉字 - which Mincho is not able to display, probably because it's not Japanese.

\setcounter{tocdepth}{3}
\setcounter{secnumdepth}{3}

\begin{document}

\input chapters/TitlepageContents.tex
%\input chapters/Abstract.tex
%\input chapters/Introduction.tex
\input chapters/JapaneseScript.tex
%\input chapters/HandwritingRecognition.tex
%\input chapters/E-Learning.tex
%\input chapters/ConceptualDesign.tex
%\input chapters/TechnicalDesign.tex
%\input chapters/HWREngine.tex
%\input chapters/ImplementationEvaluation.tex
%\input chapters/Conclusions.tex
%\input chapters/Outlook.tex
\appendix
%\input chapters/AppJapaneseScript.tex
\listoffigures
\listoftables
\bibliography{chapters/References,chapters/ReferencesE-Learning,chapters/ReferencesJapanese}
\end{document}
