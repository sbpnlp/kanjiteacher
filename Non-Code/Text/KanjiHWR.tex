%%%%%%%%%%%%%%%%%%%%%%%%%%%%%%%%%%%%%%%%%%%%%%%%%%%%%%%%%%%%%%%%%%%%%%%%%%%%%%%%
% Diploma Thesis
% Steven Buraje Poggel
%%%%%%%%%%%%%%%%%%%%%%%%%%%%%%%%%%%%%%%%%%%%%%%%%%%%%%%%%%%%%%%%%%%%%%%%%%%%%%%

\documentclass[a4paper]{book}
%\usepackage{a4wide}
\usepackage[margin=2cm]{geometry}
\usepackage{fontspec}
\usepackage[english]{babel}
\usepackage{multirow}
\usepackage{datetime}
\usepackage{url}
\usepackage{amsmath}
%\usepackage{minitoc}
%\usepackage[pdftex]{graphicx}
\usepackage{listings}
\usepackage{color} 
\usepackage{graphicx}
\usepackage{gb4e} %if there's a problem with new packages, try adding them before this one
%\usepackage{pstricks}
\newcommand{\HRule}{\rule{\linewidth}{0.5mm}}
\usepackage{chicago}
\bibliographystyle{chicago}

\title{An On-Line Japanese Handwriting Recognition System integrated into an E-Learning Environment for Kanji}
\author{\textbf{\large Diplomarbeit} \\
\textsc{ } \\
zur Erlangung des Grades \\
eines Diplom-Linguisten\\
der \\
Fachrichtung 4.7 Allgemeine Linguistik \\ 
der Universität des Saarlandes.}
\date{
Angefertigt von
Steven B.~Poggel\\
sbp@coli.uni-saarland.de \\
\textsc{ } \\
unter Leitung von \\
Prof.~Dr.~Dr.~h.c.~mult.~Wolfgang Wahlster \\
und \\
Dr.~Tilman Becker \\
\textsc{ } \\
Saarbrücken, den 31.03.2010}

%For language specifications of code environments
\newenvironment{xmlcode}
{
  \lstset{language=XML}
%  \fontspec{Courier New}
   \fontspec{GulimChe} %Using this font, so that CJK chars are supported in XML
}{}

\newenvironment{csharpcode}
{
  \lstset{language=[Sharp]C}
  \fontspec{Courier New}
}{}

% If any code snipped still uses "texcode" replace with xmlcode (fontspec 
% differs, because of CJK chars in XML)
% \newenvironment{texcode}
% {
%   \lstset{language=TeX}
%   \fontspec{Courier New}
% }{}

%For Japanese writing
\newenvironment{CJK}
{
% \fontspec[Scale=0.9]{SimSun}
%   \fontspec[Scale=0.9]{MS Mincho}
  \fontspec{MingLiU} %MingLiU can also display Chinese characters as in 汉字 - which Mincho is not able to display, probably because it's not Japanese.
  \XeTeXlinebreaklocale "ja" 
  \XeTeXlinebreakskip = 0pt plus 1pt 
}{}

%\newcommand{\cjk}[1]{{\fontspec[Scale=0.9]{MS Mincho}#1}}
\newcommand{\cjk}[1]{{\fontspec[Scale=1.0]{MingLiU}#1}} %MingLiU can also display Chinese characters as in 汉字 - which Mincho is not able to display, probably because it's not Japanese.

\setcounter{tocdepth}{3}
\setcounter{secnumdepth}{3}

\fontsize{11pt}{13}
\selectfont

\begin{document}

% CHECK1: "Fertig geschrieben", d.h. Rechtschreibpruefung und 
%         Freigabe zum Korrekturlesen
% CHECK2: Korrekturgelesen von jemand anderem
% CHECK3: Korrekturen eingebaut
% CHECK4: Endabnahme als individualkapitel
% CHECK5: Endabnahme im Gesamtzusammenhang der Arbeit
% CHECK6: Native speaker language check


% \input chapters/TitlepageContents.tex            %CHECK1 CHECK2
% \input chapters/Abstract.tex                     %CHECK1 CHECK2 CHECK3
% \input chapters/Introduction.tex                 %CHECK1 CHECK2 CHECK3
% \input chapters/JapaneseScript.tex               %CHECK1 CHECK2ab CHECK3a
% \input chapters/HandwritingRecognition.tex       %CHECK1 CHECK2
% \input chapters/E-Learning.tex                   %CHECK1 CHECK2
% \input chapters/ConceptualDesign.tex             %CHECK1 CHECK2 CHECK3
% \input chapters/TechnicalDesign.tex              %CHECK1 CHECK2
 \input chapters/HWREngine.tex                    %CHECK1 CHECK2 ToDo: continue with Tilman's corrections and find all xxx in the chapter.
% \input chapters/ImplementationEvaluation.tex     
% \input chapters/Conclusions.tex                   %CHECK1 CHECK2 CHECK3
% \input chapters/Outlook.tex                       %CHECK1
% \appendix
% \input chapters/AppJapaneseScript.tex            %CHECK1
% \input chapters/AppTechnicalities.tex           %CHECK1

% \listoffigures
% \lstlistoflistings
% \listoftables
% \bibliography{chapters/References,chapters/ReferencesE-Learning,chapters/ReferencesJapanese}
\chapter*{Document created on \today~at \xxivtime}
\end{document}
